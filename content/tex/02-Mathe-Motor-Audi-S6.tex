%ju 05-Jun-22 02-Mathe-Motor-Audi-S6.tex
\section{Motorberechnung - siehe Datenblatt Audi
S6}\label{motorberechnung-siehe-datenblatt-audi-s6}

Tabellenbuch (\textcite{bell:2021:tabellenbuchKfz} S. 32 - 33) FS
(\textcite{bell:2020:formelsammlung} S. 32 - 37)

\textbf{Aufgabe 1a Zylinderhubraum}

geg: $V_H = 4172~cm^3, z = 8$

ges: $V_h$

Formel: $V_H = V_h \cdot z \to V_h = \frac{V_H}{z}$

Lösung: $V_h = 521,5~cm^3$

\textbf{Aufgabe 1b Bohrung}

geg: $s = 9,3~cm, V_h = 521,5~cm^3$

ges: $d$

Formel:
$V_h = \frac{\pi \cdot d^2}{4} \cdot s \to d = \sqrt\frac{V_h \cdot 4}{\pi \cdot s}$

Lösung: $d = 8,4497~cm = 84,4969~mm$

\textbf{Aufgabe 1c Verdichtungsraum}

geg: $\epsilon = 11 : 1, V_h = 521,5~cm^3$

ges: $V_c$

Formel: $V_c = \frac{V_h}{\epsilon - 1}$

Lösung: $V_c = 52,15~cm^3$

\textbf{Aufgabe 1d Hubraumleistung in KW}

geg: $P_{eff} = 250~KW, V_H = 4172~cm^3 = 4,172~l$

ges: $P_H$

Formel: $P_H = \frac{P_{eff}}{V_H}$

Lösung: $P_H = 59,9233 KW/l$

\emph{spezifische Leistung} ($\to$ Literleistung, bessere
Vergleichbarkeit)

Umrechnungsfaktor $\boxed{1~PS = 0,735~KW \quad 1~KW = 1,36~PS}$

$\frac{81,4~PS/l}{1,36} = 59,85~KW$

\textbf{Aufgabe 1e}

geg: $M = 420~Nm, n = 3400~U/min$

ges: $P_{eff}$

Formel: $P_{eff} = \frac{M \cdot n}{9550}$

Lösung: $P_{eff} = 149,5288~KW$

\textbf{Aufgabe 1f Effektiven Kolbendruck bei maximaler Leistung}

geg: $P_{eff} = 250~KW, V_H = 4,172~l, n = 7000~U/min$

ges: $p_{eff}$

Formel: $p_{eff} = \frac{1200 \cdot P_{eff}}{V_H} \cdot n$

Lösung: $p_{eff} = 10,2726~bar$

\textbf{Aufgabe 1g mittlere Kolbengeschwindigkeit bei maximaler
Leistung}

geg: $s = 0,093~m, n = 7000~U/min$

ges: $v_m$

Formel: $v_m = \frac{s \cdot n}{30}$

Lösung: $v_m = 21,7~m/s$

(\emph{Standard} $v_m: \quad$ Otto = $9 - 16~m/s$, Diesel =
$8 - 14~m/s$, zwei Nullpunkte: OT, UT)

\textbf{Aufgabe 2 Motortyp nach Art der Motorsteuerung}

\begin{itemize}
\item
  >>double overhead camshaft<< (dohc)
\item
  zwei Nockenwellen über Zylinderkopf
\end{itemize}

\textbf{Aufgabe 3 Hub-Bohrung-Verhältnis}

Hub > Bohrung, $s > d$, $93~mm > 84,5~mm$ Langhuber

\emph{oder}

$\alpha = \frac{s}{d} = \frac{93}{84,5} = 1,1$

$\boxed{\alpha > 1 \quad \text{Langhuber}, \alpha = 1 \quad \text{Quadrathuber}, \alpha < 1 \quad \text{Kurzhuber}}$

\textbf{Aufgabe 4 elastischer Bereich}

Drehzahlbereich vom Maximalen Drehmoment zur Maximalen Leistung:
$3400 - 7000~U/min$
