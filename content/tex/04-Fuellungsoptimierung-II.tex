%ju 02-Jul-22 04-Fuellungsoptimierung-II.tex
\section{Wie beschreiben Sie die Dynamische
Aufladung?}\label{wie-beschreiben-sie-die-dynamische-aufladung}

\textbf{Ausgangslage} Ansaugen, Volumenvergrößerung, Druckdifferenz

Die \textbf{einströmenden Frischgase} werden am geschlossenen Ventil
\textbf{reflektiert} und an der bereits im Ansaugrohr stehenden
Luftmasse (Außenluft) erneut reflektiert und bewegt sich wieder auf das
EV zu und wenn jetzt das Ventil öffnet können die Frischgase schneller
in den Zylinder einströmen, weil die \textbf{Massenträgheit} einer
ruhenden Luftmasse nicht überwunden werden muss.

Wir machen uns hier die \textbf{kinetische Energie} der sich bereits
\textbf{in Bewegung gesetzten Luftmasse} zunutze, sodass der Beginn des
Einströmens kein Losbrechmoment der Luftmasse darstellt, sondern eine
schon in sich bewegten Luftmasse/Luftsäule zu nutzen und lässt damit das
\textbf{Einströmen schneller beginnen} und dadurch wird ein besserer
Füllungsgrad erreicht (Frischgasanteil steigt, mehr Kraftstoff $\to$
mehr Leistung und Drehmoment).

\subsection{Schwingsaugrohr}\label{schwingsaugrohr}

Variante

\begin{enumerate}
\item
  \textbf{Schaltsaugrohr} einfaches umschalten zwischen

  \begin{itemize}
  \item
    \textbf{lange Saugrohrlänge} und großes Sammlervolumen, große Massen
    (sind träge)

    \begin{itemize}
    \item
      \textbf{unteren Drehzahlbereich}
    \item
      Klappe geschlossen
    \end{itemize}
  \item
    \textbf{kurze Saugrohrlänge}, kleine Massen (sind agiler)

    \begin{itemize}
    \item
      \textbf{oberen Drehzahlbereich}
    \item
      Klappe offen
    \item
      Gassäule kann direkt aus dem Luftsammler in Richtung EV strömen
    \end{itemize}
  \end{itemize}
\item
  \textbf{Stufenlos regelbare Sauganlage}
\end{enumerate}

\subsection{Resonanzsaugrohr}\label{resonanzsaugrohr}

Beim Resonanzsaugrohr wird nicht der Weg (Saugrohrlänge) den die
Luftsäule durchlaufen muss, sondern deren Geschwindigkeit verändert.
Dies erreicht man durch Drehzahlabhängigen zu- und wegschalten einer
zusätzlichen Luftmasse im Ansaugrohr.

\begin{enumerate}
\item
  Im \textbf{oberen Drehzahlbereich} ist die Luftmasse $M_2$ durch die
  \textbf{Resonanzklappe} vom Saugrohr getrennt.

  \begin{itemize}
  \item
    Die \textbf{bewegte Luftmasse} entspricht einer relativ
    \textbf{kleinen Masse} $M_1$.
  \item
    Wodurch sie sehr \textbf{agil} ist und mit einer hohen Frequenz vom
    EV zur stehenden Außenluft zurück \textbf{reflektiert} werden kann.
  \end{itemize}
\item
  Im \textbf{unteren Drehzahlbereich} wird die \textbf{Resonanzklappe}
  geöffnet und damit die \textbf{zusätzliche Luftmasse} $M_2$
  aktiviert.

  \begin{itemize}
  \item
    Dadurch wird die Gesamtmasse $M_1 + M_2$ im Saugrohr erhöht,
    wodurch die \textbf{Geschwindigkeit der Luftsäule} abnimmt.
  \item
    Sodass sie die längere Zeit zwischen zwei Ventilöffnungen bei
    geringerer Drehzahl zur Verfügung steht, um das EV nach ihrer
    \textbf{Reflexion} mit der Außenluft wieder zu erreichen.
  \end{itemize}
\end{enumerate}

\subsection{Resonanz- und Schwingsaugrohr (keine
Prüfung)}\label{resonanz--und-schwingsaugrohr-keine-pruefung}

Bei einem 6-Zylinder-Reihenmotor werden die \emph{Zylindergruppen 1, 2,
3} und \emph{4, 5, 6} getrennt und damit hat man immer ein Ventil, was
sich öffnet und in der anderen Gruppe eins, was sich schließt.

\begin{enumerate}
\item
  Im \textbf{unteren Drehzahlbereich} ist die Umschaltklappe
  geschlossen:

  \begin{itemize}
  \item
    Bei der Befüllung der Zylinder 1, 2 und 3 wirkt der Raum vor den
    Zylindern 4, 5 und 6 als Resonanzraum und umgekehrt.
  \item
    Resonanzaufladung, hier schwingen die Luftmassen von rechts nach
    links.
  \end{itemize}
\item
  Im \textbf{oberen Drehzahlbereich} ist die Umschaltklappe geöffnet:

  \begin{itemize}
  \item
    Die Luft wird direkt angesaugt (kurzer Ansaugweg und hohe Frequenz
    der Gassäule).\\
  \item
    Für jeden einzelnen Zylinder lässt man diese Reflexionsphase
    durchlaufen.
  \end{itemize}
\end{enumerate}

\section{Fremdaufladung}\label{fremdaufladung}

Die Frischluft wird von einem Gebläse angesaugt und vor verdichtet und
mit einem Überdruck an den Motor geliefert.

\subsection{Abgasturbolader}\label{abgasturbolader}

Das \textbf{Turbinenrad} wird durch den Abgasstrom (bis zu
$320.000~min^{-1} = 5.333~s^{-1}$) beschleunigt. Dieses Turbinenrad
ist über eine \textbf{Welle} mit dem \textbf{Verdichterrad} verbunden,
das die Frischluft ansaugt und auf bis zu $2,2~bar$ verdichtet und an
den Motor liefert.

\textbf{Was versteht man unter Laufzeug?} Kombi von Turbinenrad, Welle
und Verdichterrad.

\subsubsection{Turbolader mit Bypass für
Ladedruckbegrenzung}\label{turbolader-mit-bypass-fuer-ladedruckbegrenzung}

\textbf{Warum Ladedruck begrenzen?}

\begin{itemize}
\item
  Klopfgrenze
\item
  Mechanische Überbelastung von Bauteilen
\end{itemize}

Ladedruckbegrenzung $\to$ Ladedruckregelventil (\textbf{Wastegate})
oder Bypassklappe

\subsubsection{VTG-Lader (Variable Turbinengeometrie, meist bei
Dieselmotoren)}\label{vtg-lader-variable-turbinengeometrie-meist-bei-dieselmotoren}

Konstanten Ladedruck und eine konstante Drehmomentkurve über einen
nahezu gesamten Drehzahlbereich.

Beim VTG-Lader sind vor dem Turbinenrad Leitschaufeln angeordnet, die
den Einlassquerschnitt abhängig von der Drehzahl anpassen.

\begin{enumerate}
\item
  Im \textbf{unteren Drehzahlbereich}

  \begin{itemize}
  \item
    d.h. bei einem kleinen Abgasstrom
  \item
    verstellen wir die \textbf{Leitschaufeln} so, dass der
    \textbf{Querschnitt} klein ist
  \item
    bei einer verhältnismäßig großen \textbf{Strömungsgeschwindigkeit}
  \item
    hier trifft der gesamte \textbf{Abgasstrom}
  \item
    auf das äußere Ende meines \textbf{Turbinenrades}, die wirksame
    Fläche wird größer
  \item
    großen \textbf{Hebelarm} und damit mehr Drehmoment
  \end{itemize}
\item
  Im \textbf{oberen Drehzahlbereich}

  \begin{itemize}
  \item
    verstellen wir die \textbf{Leitschaufeln} so, dass der
    \textbf{Querschnitt} groß ist
  \item
    hier trifft der gesamte \textbf{Abgasstrom}
  \item
    auf die Mitte meines \textbf{Turbinenrades}, die wirksame Fläche
    wird kleiner
  \item
    und damit haben wir den \textbf{gleichen Ladedruck} wie im unteren
    Drehzahlbereich
  \end{itemize}
\end{enumerate}

Damit der VTG-Lader auch in \textbf{Ottomotoren} eingebaut werden kann,
muss darauf geachtet werden, das die verbauten Materialien eine
dementsprechende thermische Belastbarkeit aushalten kann, um eben einen
reibungslosen Ablauf zu gewährleisten. Dieselmotoren haben eine
geringere Abgastemperatur.

\subsubsection{Registeraufladung
(Stufenaufladung)}\label{registeraufladung-stufenaufladung}

\begin{itemize}
\item
  Mitte 90er-Jahre, Audi RS2 und Porsche
\item
  kleiner und großer Turbolader sind in Reihe
\item
  Regelklappen für Abgasstromseite und Frischluftseite
\item
  Ladedruckbegrenzung $\to$ \textbf{Wastegate} (Bypassventil)
  stufenlose Ansteuerung über SG
\end{itemize}

\begin{enumerate}
\item
  \textbf{unteren Drehzahlbereich}:

  \begin{itemize}
  \item
    Regelklappen geschlossen
  \item
    \textbf{kleiner Turbo}

    \begin{itemize}
    \item
      bei einem kleinen Abgasstrom
    \item
      kommt schneller auf Drehzahl, agiler
    \item
      Warum? Durch geringere Massenträgheit
    \item
      bestimmt Ladedruck
    \end{itemize}
  \item
    \textbf{großer Turbo}

    \begin{itemize}
    \item
      dreht schon mal mit und arbeitet als Vorverdichter für den kleinen
      Lader
    \end{itemize}
  \end{itemize}
\item
  \textbf{mittleren Drehzahlbereich}:

  \begin{itemize}
  \item
    Regelklappen öffnen synchron
  \item
    verhindert Drossel Wirkung
  \end{itemize}
\item
  \textbf{oberen Drehzahlbereich}:

  \begin{itemize}
  \item
    Regelklappen voll offen
  \item
    \textbf{kleiner Turbo} läuft ohne Wirkung
  \item
    \textbf{großer Turbo} bei einem großen Abgasstrom, max. Fördern
  \end{itemize}
\end{enumerate}

Herstellername \emph{Twin-Turbo} - Bezeichnung nicht geschützt!

\subsubsection{Doppelaufladung}\label{doppelaufladung}

\begin{itemize}
\item
  zwei gleich große/kleine Turbolader sind parallel im Verbund
\item
  Ladedruckbegrenzung $\to$ \textbf{Wastegate} geöffnet
\end{itemize}

\begin{enumerate}
\item
  \textbf{unteren Drehzahlbereich}: Turbo 1 aktiv
\item
  \textbf{mittleren Drehzahlbereich}: Turbo 2 läuft an durch Öffnen
  eines Ventils, die vor verdichtete Luft wird zum Turbo 1 gefördert
\item
  \textbf{oberen Drehzahlbereich}: beide Turbo's aktiv
\end{enumerate}

Herstellername \emph{Bi-Turbo} - Bezeichnung nicht geschützt!

\subsubsection{Twin-Scroll-Lader}\label{twin-scroll-lader}

Bei einem 4 Zylinder Motor werden die \textbf{Abgasströme} der
\emph{Zylinder 1 und 4} sowie der \emph{Zylinder 2 und 3} in getrennten
Kanälen zur Turbine geleitet.

Durch Strömung-Impulse (Tick, Tick, \ldots{} immer abwechselnd kleiner
und großer Kanal) entsteht eine Impulsaufladung auf die
Turbinenschaufeln.

Vorteil: keine gegenläufigen Strömungen

\begin{enumerate}
\item
  \textbf{kleiner Kanal} leitet den Abgasstrom auf die Innenflächen der
  Turbinenschaufeln.

  \begin{itemize}
  \item
    schnelleres und sensibleres Ansprechverhalten des Laders
  \end{itemize}
\item
  \textbf{großer Kanal} leitet den Abgasstrom auf den Rand der
  Turbinenschaufeln.

  \begin{itemize}
  \item
    sorgt für höhere Drehzahl und Leistung des Turboladers
  \end{itemize}
\end{enumerate}

\subsection{Mechanische Lader}\label{mechanische-lader}

Der Antrieb erfolgt durch KW über Keilriemen.

\subsubsection{Schraubenkompressor (Roots-Lader,
Kompressor)}\label{schraubenkompressor-roots-lader-kompressor}

Beim Schraubenkompressor verdichten zwei ineinander verdrillte
Laderschaufeln/Rotoren die Luft Richtung Einlasskanal.

\textbf{Ladedruckregelung} erfolgt durch Bypassklappe oder
Magnetkupplung (Kompressor kann entkoppelt werden, um unnötigen
Kraftstoffverbrauch zu reduzieren)

\textbf{Lastwunsch} wird gesteuert durch den Fahrer über $\to$
Hauptdrosselklappe

\begin{enumerate}
\item
  \textbf{Saugbetrieb / Teillast}

  \begin{itemize}
  \item
    Bypassklappe offen, Drossel frei
  \item
    Leer fördern lassen ($\to$ d.h. Überschüssige Luft wird auf die
    Saugseite des Laders gefördert)
  \item
    hier herrscht Unterdruck
  \end{itemize}
\item
  \textbf{Ladebetrieb / Volllast}

  \begin{itemize}
  \item
    Bypassklappe geschlossen
  \item
    voller Ladedruck
  \end{itemize}
\end{enumerate}

\subsubsection{Comprex-Lader (keine Serienreife, keine
Prüfung)}\label{comprex-lader-keine-serienreife-keine-pruefung}

Besteht aus einem rotierenden Röhrenkörper, der von der Kurbelwelle
angetrieben wird. Beim Comprex-Lader schiebt Abgas die Frischluft in den
Ansaugtrakt. Das erfordert eine präzise Abstimmung auf die
Motorsteuerung.

\textbf{Hyprex-Lader}

Der Hyprex-Lader ist eine Weiterentwicklung des Comprex-Laders. Der
Röhrenkörper wird durch einen elektronisch geregelten Elektromotor
angetrieben.

\subsubsection{Kombi von Turbolader und Kompressor (VW bei den
Twincharger-TSI-Motoren)}\label{kombi-von-turbolader-und-kompressor-vw-bei-den-twincharger-tsi-motoren}

Hauptvorteile verknüpfen

\begin{itemize}
\item
  \textbf{Kompressor} (im unteren Drehzahlbereich $\to$ direktes
  Ansprechverhalten)
\item
  \textbf{Turbolader} (im oberen Drehzahlbereich $\to$ nahezu keine
  Leistungsentnahme vom Verbrennungsmotor)
\end{itemize}

\subsection{Elektrische Lader (eLader)}\label{elektrische-lader-elader}

\begin{itemize}
\item
  Antrieb des Verdichterrads: 48 V Elektromotor
\item
  unabhängig vom Abgasstrom und damit kein Turboloch

  \begin{itemize}
  \item
    \textbf{unteren Drehzahlbereich} $\to$ elektrische Lader
  \item
    \textbf{oberen Drehzahlbereich} $\to$ Abgasturbolader
  \end{itemize}
\end{itemize}

\subsection{Warum muss ich die Ladeluft
kühlen?}\label{warum-muss-ich-die-ladeluft-kuehlen}

\textbf{Was begrenzt den maximalen Ladedruck?}

Klopfgrenze, \textbf{wodurch tritt eine klopfende Verbrennung ein?}
Ungewollte Glühzündung, \textbf{wodurch entsteht eine Glühzündung?}
Durch zu viel Druck und Hitze.

Wenn ich dem System Hitze entziehe, kann ich mit dem Ladedruck höher
gehen. Meine angesaugte Luftmasse hat eine höhere Dichte, ich kann
gleichzeitig mehr davon reinpacken. Dadurch ist meine Leistungsfähigkeit
noch mal gestiegen.
