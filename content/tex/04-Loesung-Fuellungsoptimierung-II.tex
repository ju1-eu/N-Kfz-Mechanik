%ju 28-Mai-22 04-Loesung-Fuellungsoptimierung-II.tex
\textbf{1) Nennen Sie Möglichkeiten zur Leistungssteigerung eines
Verbrennungsmotors.}

In der mir verfügbaren Zeit möglichst viel Kraftstoff und Luft in den
Zylinder zu bekommen. Dieses Kraftstoff-Luft-Gemisch wird zur
Verbrennung gebracht und soll meinen Kolben effektiv nach unten treiben.

Mögliche Systeme

\begin{enumerate}
\item
  Einventiltechnik $\to$ Mehrventiltechnik
\item
  Saugmotor $\to$ Fremdaufladung
\item
  Steuerzeiten $\to$ variable Steuerzeiten (Nachladeeffekt nutzen)
\item
  Ventiltrieb $\to$ variable Ventiltrieb (unterschiedliche
  Nockenprofile und Ventilöffnungszeiten)
\item
  Dynamische Aufladung (Strömungsenergie der bereits bewegten Luftmasse
  nutzen innerhalb meines Ansaugsystems)
\item
  Motordrehzahl anheben $\to$ z. B. Honda (kleinen Hubraum und hohe
  Drehzahl)
\item
  Hubraum vergrößern
\item
  Zündung optimieren
\end{enumerate}

\textbf{2) Definieren Sie Dynamische Aufladung und Fremdaufladung}

\textbf{a) Dynamische Aufladung}

Die dynamische Aufladung erfolgt ausschließlich durch Nutzung der
kinetischen Energie der Gassäule im Ansaugtrakt. Wird das EV
geschlossen, kommt es zur Reflexion und an der bereits im Ansaugrohr
stehenden Luftmasse (Außenluft) erneut reflektiert und bewegt sich
wieder auf das EV zu. Im Idealfall soll die Gassäule wieder vor dem EV
stehen, wenn diese gerade öffnet.

Erreichbar ist diese durch

\begin{enumerate}
\item
  dynamische Ansaugwege

  \begin{itemize}
  \item
    lange Wege für niedrige Drehzahlen
  \item
    kurze Wege für hohe Drehzahlen
  \end{itemize}
\item
  Resonanzsaugrohr - durch Änderung der Luftgeschwindigkeit durch
  zuschaltbare Luftmassen

  \begin{itemize}
  \item
    Resonanzklappe offen, zusätzliche Luftmasse aktiviert, das erhöht
    die Gesamtmasse im Saugrohr, wodurch die Geschwindigkeit der
    Luftsäule abnimmt (Massenträgheit) $\to$ für niedrige Drehzahlen
  \item
    Resonanzklappe geschlossen, die bewegte Luftmasse ist gering, sehr
    agil und mit hoher Frequenz vom EV zu stehenden Außenluft und zurück
    reflektiert $\to$ für hohe Drehzahlen
  \end{itemize}
\end{enumerate}

\textbf{b) Fremdaufladung}

Die Frischluft wird von einem Gebläse angesaugt und vor verdichtet und
mit einem Überdruck an den Motor geliefert. Füllungsgrad auf bis zu
160\% erreicht werden können.

Systeme: Abgasturbolader, eLader, Kompressor

\textbf{3) Welche Möglichkeiten bieten Schaltsaugrohre?}

Sie ermöglichen eine bedarfsgerechte Änderung der Ansaugwege. Diese
Maßnahme bewirkt eine Erhöhung der Zylinderfüllung und somit eine
Steigerung des Drehmoments bzw. Motorleistung. Die Laufdauer der
Luftsäule ändert sich mit der Frequenz.

\textbf{Schaltsaugrohr} einfaches umschalten zwischen

\begin{itemize}
\item
  \textbf{lange Saugrohrlänge} und großes Sammlervolumen, große Massen
  (sind träge)

  \begin{itemize}
  \item
    \textbf{unteren Drehzahlbereich}
  \item
    Klappe geschlossen
  \end{itemize}
\item
  \textbf{kurze Saugrohrlänge}, kleine Massen (sind agiler)

  \begin{itemize}
  \item
    \textbf{oberen Drehzahlbereich}
  \item
    Klappe offen
  \item
    Gassäule kann direkt aus dem Luftsammler in Richtung EV strömen
  \end{itemize}
\end{itemize}

\textbf{4) Wie ist grundsätzlich die Wirkungsweise eines
Abgas-Turboladers?}

Das \textbf{Turbinenrad} wird durch den Abgasstrom beschleunigt. Dieses
Turbinenrad ist über eine \textbf{Welle} mit dem \textbf{Verdichterrad}
verbunden, das die Frischluft ansaugt und auf bis zu $2,2~bar$
verdichtet und an den Motor liefert.

\textbf{5) Was bedeutet das Kürzel VTG in Verbindung mit
Fremdaufladung?}

Variable Turbinengeometrie

\textbf{Beschreiben Sie das Verhalten dieses Laders in Abhängigkeit zur
Drehzahl.}

$\boxed{\uparrow M = \sim F \cdot \uparrow r}$

\begin{enumerate}
\item
  Bei \textbf{niedriger Drehzahl} mit geringer Abgasmenge wird durch die
  Leitschaufelstellung ein kleiner Eintrittsquerschnitt bemessen und der
  Abgasstrom auf den äußeren Rand des Turbinenrades geleitet. Hierdurch
  wird das Abgas beschleunigt und trifft zudem auf einen langen
  Hebelarm. Am Turbinenrad entsteht ein großes Moment, die
  Turbinendrehzahl und der Ladedruck steigen.
\item
  Bei \textbf{hohen Drehzahlen} mit entsprechend größere Abgasmenge wird
  durch die Leitschaufel ein großer Einlassquerschnitt eingestellt und
  der Abgasstrom relativ nah an das Zentrum des Turbinenrades geleitet.
  Die höhere Abgasgeschwindigkeit in Verbindung mit dem größeren
  Abgasvolumen kompensiert den kleinen Hebelarm am Turbinenrad. Wodurch
  der Ladedruck konstant bleibt.
\end{enumerate}

\textbf{6) Warum werden VTG-Lader nur bei Dieselmotoren verwendet?}

Die Abgastemperatur bei Ottomotoren ist im Volllastbereich bis zu
1000~°C (im Vergleich Dieselmotor bis ca. 800 °C) zu hoch. Die
Temperatur am Verstellmechanismus darf 850 °C nicht übersteigen, da
dieser sonst ausfallen könnte.

\emph{Ergänzung,} dies gilt nicht für moderne VTG-Lader mit Molybdän
beschichteten Verstellmechanismus. Diese sind für den Einsatz im
Ottomotor geeignet.

\textbf{7) Beschreiben Sie Aufbau und Wirkungsweise der Doppel- und
Registeraufladung}

\textbf{a) Doppelaufladung}

\begin{itemize}
\item
  zwei gleich große/kleine Turbolader sind parallel im Verbund
\item
  Ladedruckbegrenzung $\to$ \textbf{Wastegate} geöffnet
\end{itemize}

\begin{enumerate}
\item
  \textbf{unteren Drehzahlbereich}: Turbo 1 aktiv
\item
  \textbf{mittleren Drehzahlbereich}: Turbo 2 läuft an durch Öffnen
  eines Ventils, die vor verdichtete Luft wird zum Turbo 1 gefördert
\item
  \textbf{oberen Drehzahlbereich}: beide Turbo's aktiv
\end{enumerate}

\textbf{b) Registeraufladung}

\begin{itemize}
\item
  kleiner und großer Turbolader sind in Reihe
\item
  Regelklappen für Abgasstromseite und Frischluftseite
\item
  Ladedruckbegrenzung $\to$ \textbf{Wastegate} (Bypassventil)
  stufenlose Ansteuerung über SG
\end{itemize}

\begin{enumerate}
\item
  \textbf{unteren Drehzahlbereich}:

  \begin{itemize}
  \item
    Regelklappen geschlossen
  \item
    \textbf{kleiner Turbo,} geringe Massenträgheit

    \begin{itemize}
    \item
      bei einem kleinen Abgasstrom
    \item
      kommt schneller auf Drehzahl, agiler
    \item
      Warum? Durch geringere Massenträgheit
    \item
      bestimmt Ladedruck
    \end{itemize}
  \item
    \textbf{großer Turbo}

    \begin{itemize}
    \item
      dreht schon mal mit und arbeitet als Vorverdichter für den kleinen
      Lader
    \end{itemize}
  \end{itemize}
\item
  \textbf{mittleren Drehzahlbereich}:

  \begin{itemize}
  \item
    Regelklappen öffnen synchron
  \item
    verhindert Drossel Wirkung
  \end{itemize}
\item
  \textbf{oberen Drehzahlbereich}:

  \begin{itemize}
  \item
    Regelklappen voll offen
  \item
    \textbf{kleiner Turbo} läuft ohne Wirkung
  \item
    \textbf{großer Turbo} bei einem großen Abgasstrom, max. Fördern
  \end{itemize}
\end{enumerate}

\textbf{8) Welchen Vorteil erreicht man durch die Ladeluftkühlung?}

Eine bessere Zylinderfüllung durch höhere Luftdichte. Durch Senkung der
Ladelufttemperatur z. B. 120 °C auf 70 °C (ca. 50 °C abkühlen) niedrige
Verbrennungstemperatur, Selbstentzündungstemperatur wird später
erreicht, geringere Klopfneigung und dadurch höhere Ladedrücke.

\textbf{Innere Kühlung}

\begin{enumerate}
\item
  durch die Temperatur der angesaugten Luftmasse wird das innere des
  Zylinders gekühlt
\item
  Kraftstoff wird flüssig in den Zylinder eingespritzt und fängt an, an
  der umgebenen Wärme gasförmig zu werden. Durch den Aggregatzustands
  wechsel von flüssig in gasförmig entsteht ein Druckverlust und dadurch
  entziehen wir der Umgebungsluft Wärme.
\end{enumerate}

\textbf{9) Was versteht man unter >>Downsizing<<?}

Verkleinerung der Verbrennungsmotoren (Hubraum und Zylinderzahl) bei
gleicher Leistung.

\textbf{Warum macht man das?} Durch Verringern des Hubraums oder
wegfallen einzelne Zylinder verringern wir die Reibungsverluste und
damit einen geringeren Verlust der erzeugten Leistung. Um Kraftstoff zu
sparen.

\textbf{10) Wodurch ist die Leistungssteigerung durch Aufladung eines
Otto-Motors begrenzt?}

Ladedruck kann nicht unbegrenzt erhöht werden. \textbf{Warum?} Durch die
Klopfgrenze des Kraftstoff-Luft-Gemisches. Lädt man einen Ottomotor zu
stark auf, kommt es zu einer ungewollten Kompressionszündung, der
sogenannten klopfenden Verbrennung.

Klopfgrenze, \textbf{wodurch tritt eine klopfende Verbrennung ein?}
Ungewollte Glühzündung, \textbf{wodurch entsteht eine Glühzündung?}
Durch zu viel Druck und Hitze.

\textbf{11) Wassereinspritzung}

\textbf{Aufbau:} Wassertank, Einspritzdüsen

\textbf{Sinn?}

\begin{itemize}
\item
  dem Brennraum die Temperatur entziehen
\item
  dadurch Klopfneigung reduzieren
\item
  Ladedruck erhöhen
\item
  Leistung ausschöpfen
\end{itemize}

\textbf{Kompensieren der Außentemperatur} z. B. 40 °C

\begin{itemize}
\item
  durch mehr Wasser Einspritzen
\item
  wird vom Motorsteuergerät überwacht
\item
  Last/Drehzahl abhängig
\item
  Ansaugluft 25 °C zusätzlich runterkühlen
\item
  8 \% höhere Leistung und gleichzeitig 8 \% Kraftstoffeinsparung
\end{itemize}

\textbf{Vorteil}

Die Temperaturen von $\to$ Kolbenboden, Ventile, Katalysator, Lader
entlasten.

\textbf{Einspritzung - Zerstäubung} unter einem hohen Druck möglichst
fein zerstäuben (Mehrlochdüse) Tröpfchenbildung (Kugeloberfläche). Je
feiner ich zerstäube, umso höher ist die Wahrscheinlichkeit, dass ich
einen vollständigen Verdunstungsprozess habe, der dazu führt, dass ich
im Idealfall keinerlei Rußbildung erzeuge. Bei bestimmten Lastzuständen,
hohen Einspritzdruck und kurzer Einspritzzeit habe ich das Problem, dass
die Tröpfchengröße ansteigt und so zu einer Entstehung von Ruß kommt.
