%ju 17-Jul-22 Pruefungstrainer-Kfz-Technik.tex
\section{Kraftfahrzeug, Wartung,
Instandhaltung}\label{kraftfahrzeug-wartung-instandhaltung}

\subsection{Technisches System
Kraftfahrzeug}\label{technisches-system-kraftfahrzeug}

\begin{enumerate}
\item
  Energiefluss\\
\item
  Technik: Energie\\
\item
  Öffnungskraft\\
\item
  Resultierende Kraft\\
\item
  Produkt aus Kraft und Hebelarm\\
\item
  Anzugsmoment\\
\item
  Kettenraddurchmesser\\
\item
  Hubeinrichtung\\
\item
  Kolben im Arbeitstakt\\
\item
  Mechanische Arbeit\\
\item
  Wirkungsgrad\\
\item
  Energie\\
\item
  Hauptfunktion des Kraftfahrzeugs\\
\item
  Systemgrenze\\
\item
  Straßenfahrzeuge\\
\item
  Kraftfahrzeuge\\
\item
  Unterscheidung Straßenfahrzeuge\\
\item
  EVA-Prinzip\\
\item
  Funktionseinheiten des Kraftfahrzeugs
\end{enumerate}

\subsection{Wartung und
Instandhaltung}\label{wartung-und-instandhaltung}

\begin{enumerate}
\item
  Wartungsabstände\\
\item
  Flexible Serviceintervalle\\
\item
  Wartungsplan\\
\item
  Luftfilter\\
\item
  Stark verschmutzter Luftfilter\\
\item
  Kraftstofffilter\\
\item
  Kraftstofffilterbauart\\
\item
  Nassluftfilter\\
\item
  Schleuderluftfilter\\
\item
  Wechselfilter\\
\item
  Kfz-Starterbatterie\\
\item
  Kühlwasser\\
\item
  Schäden an Autoscheiben\\
\item
  Bremsenbauart\\
\item
  Bremsflüssigkeit\\
\item
  Radbremszylinder\\
\item
  Bremse\\
\item
  Scheibenbremsbeläge\\
\item
  Zweikreisbremse\\
\item
  Wartung und Instandhaltung\\
\item
  Wartungsplan\\
\item
  Ölwechselintervall\\
\item
  Verschleißzustand von Bremsbelägen\\
\item
  Servicezeitpunkt\\
\item
  Filterbauarten\\
\item
  Verschmutzter Luftfilter\\
\item
  Boxfilter\\
\item
  Fahrzeuglackierung\\
\item
  Lackzustand\\
\item
  Reinigung mit Hochdruckwascher
\end{enumerate}

\subsection{Betriebsstoffe,
Hilfsstoffe}\label{betriebsstoffe-hilfsstoffe}

\begin{enumerate}
\item
  Motoren-Kraftstoffe\\
\item
  Kettenförmig aufgebaute Kraftstoffmoleküle\\
\item
  Cracken\\
\item
  Ottokraftstoffe\\
\item
  Kohlenwasserstoff-Verbindungen\\
\item
  Superbenzin - Super E10\\
\item
  Oktanzahl\\
\item
  Siedekurven von Ottokraftstoff\\
\item
  Cetanzahl\\
\item
  Dieselkraftstoffe\\
\item
  Eigenschaft des Dieselkraftstoffes\\
\item
  Diesel- und Ottokraftstoff\\
\item
  ROZ 98\\
\item
  Cetanzahlen des Dieselkraftstoffs\\
\item
  Viskosität\\
\item
  Mehrbereichsöl\\
\item
  Additive\\
\item
  SAE-Klassen\\
\item
  API-Klassifikation\\
\item
  Anforderungen an Motorenöle\\
\item
  ACEA-Spezifikation\\
\item
  Öle für Automatikgetriebe\\
\item
  Ölverdickung\\
\item
  Ölnormung\\
\item
  Ölnormung\\
\item
  SAE-Klassifikation\\
\item
  Kühlflüssigkeit\\
\item
  Bremsflüssigkeit\\
\item
  Falscher Kraftstoff\\
\item
  Viskosität\\
\item
  Glykol\\
\item
  Betriebs- und Hilfsstoffe\\
\item
  Kohlenwasserstoff-Moleküle\\
\item
  Vergleich Otto- und Dieselkraftstoffe\\
\item
  Arten von Schmierstoffen\\
\item
  Aufgaben von Schmierölen\\
\item
  Viskosität der Schmieröle\\
\item
  API-Klassifikation von Motorölen\\
\item
  Eigenschaftsänderungen von Motorölen\\
\item
  Getriebeöle\\
\item
  Eigenschaften von Kühlflüssigkeiten\\
\item
  Eigenschaften von Bremsflüssigkeiten\\
\item
  Wechsel von Bremsflüssigkeiten\\
\item
  Altöl
\end{enumerate}

\subsection{Umweltschutz}\label{umweltschutz}

\begin{enumerate}
\item
  Luftverschmutzung\\
\item
  Versickern von Mineralölprodukten\\
\item
  Kreislaufwirtschafts- und Abfallgesetz\\
\item
  Recycling\\
\item
  Recyclate\\
\item
  Weiterverwertbare Abfälle\\
\item
  Wiederverwertbare Abfallstoffe des Kfz-Bereiches\\
\item
  Gefährliche Stoffe\\
\item
  Gefahrenklasse A1\\
\item
  Sammeln und Lagern von Altöl\\
\item
  Altöle bekannter Herkunft\\
\item
  Altöle unbekannter Herkunft\\
\item
  Überwachungsbedürftige Abfälle\\
\item
  Überwachungsbedürftige Abfälle\\
\item
  Nicht besonders überwachungsbedürftige verwertbare Abfälle\\
\item
  Beseitigung besonders überwachungsbedürftiger Abfälle\\
\item
  Abfallgruppe\\
\item
  Verschmutzter Kaltreiniger\\
\item
  Entsorgungsnachweis\\
\item
  Schadstoffe\\
\item
  Typische Abfälle zur Verwertung\\
\item
  Abfälle zur Beseitigung\\
\item
  Wiederverwertbare, nachweispflichtige Abfälle\\
\item
  Gewässerverschmutzung\\
\item
  Recycling von Kunststoffteilen
\end{enumerate}

\subsection{Arbeitsschutz}\label{arbeitsschutz}

\begin{enumerate}
\item
  Umgang mit Benzin\\
\item
  Organisationen\\
\item
  Berufsgenossenschaften\\
\item
  Verbindliche Unfallverhütungsvorschriften\\
\item
  Sicherheitszeichen unterscheiden\\
\item
  Gebotszeichen\\
\item
  Sicherheitszeichen\\
\item
  Gelbgrundige Sicherheitszeichen\\
\item
  Rettungszeichen\\
\item
  Bedeutung von Sicherheitszeichen\\
\item
  Rettungswege\\
\item
  Menschliches Versagen\\
\item
  Hebebühne bedienen\\
\item
  Arbeits- oder Wegeunfälle\\
\item
  Hochvoltkomponenten\\
\item
  Arbeits- oder Wegeunfall - Arzt\\
\item
  Unfallmeldung\\
\item
  Betriebsstoffe und Warnzeichen\\
\item
  Gefahrenklasse Al\\
\item
  Gebotszeichen - Tätigkeiten\\
\item
  Arten von Sicherheitszeichen\\
\item
  Symbol - Bedeutung\\
\item
  Fahrzeugteilen mit Reibbelägen\\
\item
  Verpflichtung zu helfen
\end{enumerate}

\section{Steuern und Regeln, Prüf- und
Fertigungstechnik}\label{steuern-und-regeln-pruef--und-fertigungstechnik}

\subsection{Steuern und Regeln}\label{steuern-und-regeln}

\begin{enumerate}
\item
  Steuern und Regeln\\
\item
  Merkmal einer Steuerung\\
\item
  Aktor\\
\item
  Steuerungs- und Regelungsvorgänge\\
\item
  Stellglieder (Aktoren)\\
\item
  Steuerkette\\
\item
  Regelungsvorgang\\
\item
  Energieträger bei pneumatischen Steuerungen\\
\item
  Regelgröße\\
\item
  Regelkreis\\
\item
  Signalarten\\
\item
  Symbole\\
\item
  Wegeventile\\
\item
  Wegeventil: Anschlüsse und Schaltstellungen\\
\item
  3/2 Wegeventil\\
\item
  Ventil\\
\item
  Einfach wirkendes Rückschlagventil\\
\item
  Bauelement\\
\item
  Schaltung\\
\item
  Steuerglieder (Steuergeräte)\\
\item
  Signalformen\\
\item
  Wegeventile\\
\item
  Rückschlagventil
\end{enumerate}

\subsection{Prüftechnik}\label{prueftechnik}

\begin{enumerate}
\item
  Prüftechnik: Begriffe\\
\item
  Physikalische Größen und Einheiten\\
\item
  1/20 Nonius\\
\item
  Messgerät\\
\item
  Messschraube\\
\item
  Überprüfen einer Ventilführung\\
\item
  Erwärmte Messschraube\\
\item
  Messuhren\\
\item
  Universalwinkelmesser\\
\item
  Fühlerlehre\\
\item
  Planlaufabweichung\\
\item
  Messgerät und Messvorgang\\
\item
  Druckeinheiten\\
\item
  Sl-Basiseinheiten\\
\item
  Längenprüftechnik: Messen\\
\item
  Basiseinheit der Länge\\
\item
  Handhabung des Messgerätes\\
\item
  Messschieber\\
\item
  Messschieber: Messungen
\end{enumerate}

\subsection{Fertigungstechnik}\label{fertigungstechnik}

\begin{enumerate}
\item
  Hauptgruppen der Fertigungsverfahren\\
\item
  Sintern\\
\item
  Zu groß gebohrtes Kernloch\\
\item
  Winkel\\
\item
  Freiwinkel am Schneidkeil\\
\item
  Keilwinkel eines Meißels\\
\item
  Zahnteilung von Sägeblättern\\
\item
  Freischneiden des Sägeblattes\\
\item
  Begriffe zuordnen\\
\item
  Feilenzähne\\
\item
  Schaben\\
\item
  Schneidkeil\\
\item
  Schaber\\
\item
  Reiben\\
\item
  Reibahlen\\
\item
  Handreibahle\\
\item
  Satzgewindebohrer\\
\item
  Gewindeschneiden\\
\item
  ISO-Gewinde\\
\item
  Abgebrochene Gewindebohrer\\
\item
  Spiralbohrer mit ungleichen Hauptschneiden\\
\item
  Spiralbohrer mit ungleichen Schneidenwinkel\\
\item
  Zerteilen\\
\item
  Trennen durch Zerteilen\\
\item
  Fertigungsverfahren\\
\item
  Fügen\\
\item
  Stoffschlüssige Fügeverbindungen\\
\item
  Lösbare Verbindungen\\
\item
  Schraubensicherungen\\
\item
  Schraube\\
\item
  Löten\\
\item
  Flussmittel beim Löten\\
\item
  Lötspalt\\
\item
  MIG/MAG-Schweißen\\
\item
  Stahlkarosserie-Instandsetzung\\
\item
  Schutzgas\\
\item
  Schweißen: Acetyle und Sauerstoff\\
\item
  Kleben\\
\item
  Zweikomponentenkleber\\
\item
  Urformen\\
\item
  Druckgießen\\
\item
  Sintern\\
\item
  Biegen\\
\item
  Richten\\
\item
  Meißel\\
\item
  Handbügelsäge\\
\item
  Reiben\\
\item
  Stiftverbindungen\\
\item
  Werkzeuge für Innen- und Außengewinde\\
\item
  Bohrwerkzeug A\\
\item
  Bohrwerkzeug B\\
\item
  Lösbare Verbindungen\\
\item
  Unlösbare Verbindungen\\
\item
  Gewindebezeichnungen\\
\item
  Festigkeitswerte\\
\item
  Flussmittel beim Löten\\
\item
  Löten\\
\item
  Schweißen\\
\item
  Schutzgasschweißen\\
\item
  Schweißnähte\\
\item
  Klebeverbindungen\\
\item
  Beschichten
\end{enumerate}

\subsection{Werkstofftechnik}\label{werkstofftechnik}

\begin{enumerate}
\item
  Physikalische Werkstoffeigenschaften\\
\item
  Technologische Werkstoffeigenschaften\\
\item
  Chemische Werkstoffeigenschaften\\
\item
  Härte\\
\item
  Dichte\\
\item
  Festigkeit\\
\item
  Hilfsstoffe\\
\item
  Positionsnummern\\
\item
  Zugfestigkeit, Streckgrenze, Bruchdehnung\\
\item
  Stoffeigenschaftänderung\\
\item
  Stahlguss\\
\item
  Nichtmetalle\\
\item
  Metalle\\
\item
  Vergüten und Härten\\
\item
  Vergüten\\
\item
  Vergütungsstahl\\
\item
  Achsteile\\
\item
  Nichteisenmetalle\\
\item
  Leichtmetalle\\
\item
  Legieren\\
\item
  Aluminium\\
\item
  G-AlSi 12\\
\item
  Kolben\\
\item
  Gusseisen mit Lamellengrafit\\
\item
  Kunststoffe\\
\item
  Duroplaste\\
\item
  Elastomere\\
\item
  Thermoplaste\\
\item
  Anwendungsbeispiele für Elastomere\\
\item
  Verbundwerkstoffe\\
\item
  Festigkeit\\
\item
  Elastizität\\
\item
  Härte\\
\item
  Korrosion\\
\item
  Elektrochemische Korrosion\\
\item
  Werkstoffe - Hauptgruppen\\
\item
  Grauguss\\
\item
  Nichteisenmetalle\\
\item
  Kupfer\\
\item
  Kunststoffe - Gruppen\\
\item
  Einscheibensicherheitsglas\\
\item
  Verbundsicherheitsglas\\
\item
  Verbundwerkstoffe
\end{enumerate}

\subsection{Reibung, Schmierung, Lager und
Dichtungen}\label{reibung-schmierung-lager-und-dichtungen}

\begin{enumerate}
\item
  \textbf{Reibungsart >>Ölfilm zwischen Lager und Welle<<:}

  \begin{itemize}
  \item
    Flüssigkeitsreibung
  \end{itemize}
\item
  \textbf{Gleitreibungsarten}

  \begin{itemize}
  \item
    Festkörper-, Misch-, Flüssigkeitsreibung
  \end{itemize}
\item
  \textbf{Lager (Aufgabe)}

  \begin{itemize}
  \item
    Wellen führen und abstützen
  \end{itemize}
\item
  \textbf{Wälzlager}

  \begin{itemize}
  \item
    Außenring, Innenring, Käfig, Wälzkörper
  \end{itemize}
\item
  \textbf{Lagerbenennungen (Bild)}

  \begin{itemize}
  \item
    Schrägkugellager, Nadellager, Kegelrollenlager
  \end{itemize}
\item
  \textbf{Lager >>Belastung eines Schrägkugellagers<<}

  \begin{itemize}
  \item
    Kleine Axialkraft, große Radialkraft
  \end{itemize}
\item
  \textbf{Rollenlager vs.~Kugellager}

  \begin{itemize}
  \item
    Übertragen Radialkräfte auf einer Linie
  \end{itemize}
\item
  \textbf{Lagerung >>Kegelrollenlager<<:}

  \begin{itemize}
  \item
    Radiales Spiel kann nicht beeinflusst werden
  \end{itemize}
\item
  \textbf{Wälzlager vs.~Gleitlager (Vorteile)}

  \begin{itemize}
  \item
    Geringere Reibung
  \end{itemize}
\item
  \textbf{Weichstoffdichtung >>Dichtwirkung erfolgt<<}

  \begin{itemize}
  \item
    Flächenpressung und Verformung des Dichtwerkstoffes
  \end{itemize}
\item
  \textbf{Profildichtungen (Anwendung):}

  \begin{itemize}
  \item
    Türgummi, o-Ring
  \end{itemize}
\item
  \textbf{Dynamische Dichtungen}

  \begin{itemize}
  \item
    Dichtwerkstoff muss Undichtheit gering halten
  \item
    Dichtlippe, Schutzlippe, Feder, Versteifungsring
  \end{itemize}
\item
  \textbf{Reibungsarten}

  \begin{itemize}
  \item
    Gleitreibung, Haftreibung, Rollreibung
  \end{itemize}
\item
  \textbf{Reibungsarten: Schmierzustand}

  \begin{itemize}
  \item
    Flüssigkeitsreibung, Trockenereibung, Mischreibung
  \end{itemize}
\item
  \textbf{Aufgaben: Lager}

  \begin{itemize}
  \item
    Führen und Abstützen von Wellen, Verschleiß verringern
  \end{itemize}
\item
  \textbf{Radlagerung bei Vorderradantrieb}

  \begin{itemize}
  \item
    2x Kegelrollenlager (entgegengesetzt), 2x Schrägkugellager
  \end{itemize}
\item
  \textbf{Fettschmierung von Wälzlagern}

  \begin{itemize}
  \item
    Gehäusehohlräume nur zu Hälfte mit Fett füllen
  \item
    Wälzlagerfett oder Mehrzweckfett
  \end{itemize}
\end{enumerate}

\section{Viertaktmotor}\label{viertaktmotor}

\subsection{Grundlagen
Otto-Dieselmotor}\label{grundlagen-otto-dieselmotor}

\begin{enumerate}
\item
  Otto-Viertaktmotor\\
\item
  Dieselmotor\\
\item
  Verdichtungsverhältnis bei Dieselmotoren\\
\item
  Dieselmotoren\\
\item
  Dieselmotor\\
\item
  Innere Gemischbildung beim Dieselmotor\\
\item
  Zündverzug beim Dieselmotor\\
\item
  Nageln beim Dieselmotor\\
\item
  Verbrennungsmotoren\\
\item
  Gemischbildung Dieselmotor\\
\item
  Arbeitsspiel des Viertaktmotors\\
\item
  Dieselmotor
\end{enumerate}

\subsection{Physikale und chemische
Grundlagen}\label{physikale-und-chemische-grundlagen}

\begin{enumerate}
\item
  Verbrennungsmotoren\\
\item
  Viertaktmotor\\
\item
  Einlassventil\\
\item
  Vergrößerung des Hubes\\
\item
  Druck und Temperatur\\
\item
  Verdichtungsverhältnis\\
\item
  Ansaugtakt\\
\item
  Arbeitsweise des Dieselmotors\\
\item
  Ottomotor/Dieselmotor
\end{enumerate}

\subsection{Motor-Diagramme,-
Kennlinien}\label{motor-diagramme--kennlinien}

\begin{enumerate}
\item
  Zündzeitpunkt im Arbeitsdiagramm\\
\item
  Angaben im Arbeitsdiagramm\\
\item
  Angaben im Steuerdiagramm\\
\item
  Steuerdiagramm\\
\item
  Drehrichtung der Kurbelwelle\\
\item
  Zündreihenfolge\\
\item
  Zylindernummerierung\\
\item
  Takte bestimmen\\
\item
  Kurbelwellenbauformen\\
\item
  Zylinderhubraum\\
\item
  Volllastkennlinien Otto-Viertaktmotor\\
\item
  Drehzahl-Drehmoment-Kennlinien Otto-Viertaktmotor\\
\item
  Drehmoment- Drehzahlverlauf Otto-Viertaktmotor\\
\item
  Kurzhub- und Langhubmotoren\\
\item
  Hubraumleistung\\
\item
  Leistungsgewicht\\
\item
  Ventilüberschneidung\\
\item
  Normgerechte Zylindernummerierung\\
\item
  Zündabstand\\
\item
  Motordiagramm
\end{enumerate}

\subsection{Grundlagen Motor}\label{grundlagen-motor}

\subsubsection{Motormechanik}\label{motormechanik}

\begin{enumerate}
\item
  Motorgehäuse\\
\item
  Verdichtungsverhältnis\\
\item
  Motorbauteil\\
\item
  Zylinderlaufbuchsen\\
\item
  Zylinderkopfdichtung\\
\item
  Kurbelgehäuseentlüftung\\
\item
  Kompressionsdruckverlustprüfung\\
\item
  Kompressionsdruckprüfung\\
\item
  Druckverlustprüfung\\
\item
  Kurbelradius\\
\item
  Kolbenringe\\
\item
  Pleuelstange\\
\item
  Pleuelstange - Bauteile\\
\item
  Pleuelstange - Beanspruchungsarten\\
\item
  Pleuelstange\\
\item
  Kurbelwelle\\
\item
  Beanspruchungen der Kurbelwelle\\
\item
  Bestandteile der Kurbelwelle\\
\item
  Pleuelstangen mit schräg geteiltem Pleuelfuß\\
\item
  Zylinder und Zylinderkopf\\
\item
  Zylinderlaufbuchsen\\
\item
  Zylinderkopfdichtung\\
\item
  Zylinderkopfdichtung auswechseln\\
\item
  Motorlagern\\
\item
  Zylinderkopfschrauben\\
\item
  Prüfung des Kompressionsdruckes (OM)\\
\item
  Kompressionsdruckverlust-Prüfung
\end{enumerate}

\subsubsection{Motorkühlsystem}\label{motorkuehlsystem}

\begin{enumerate}
\item
  \textbf{Aufgabe der Motorkühlung}

  \begin{itemize}
  \item
    Überschüssige Verbrennungswärme aus dem Motor an die Umgebungsluft
    abführen
  \end{itemize}
\item
  \textbf{Verbrennungswärme} Welchen Anteil muss die Kühlung abführen?

  \begin{itemize}
  \item
    25 - 33 \%
  \end{itemize}
\item
  \textbf{Auswirkung der Motorkühlung}

  \begin{itemize}
  \item
    Die Füllung wird verbessert
  \end{itemize}
\item
  \textbf{Flüssigkeitskühlung} (Vorteil)

  \begin{itemize}
  \item
    Gleichmäßige Kühlwirkung
  \end{itemize}
\item
  \textbf{Woraus besteht Kühlflüssigkeit}

  \begin{itemize}
  \item
    Gemisch aus Wasser, Gefrierschutzmittel und Korrosionsschutz
  \end{itemize}
\item
  \textbf{Kühler (Aufgabe)}

  \begin{itemize}
  \item
    Er überträgt die Kühlflüssigkeitswärme auf die Umgebungsluft
  \end{itemize}
\item
  \textbf{Überdruck im Kühlsystem (Welche Auswirkung)}

  \begin{itemize}
  \item
    Die Kühlflüssigkeitstemperatur kann auf 100°C bis 120°C ansteigen
  \end{itemize}
\item
  \textbf{Einfüllverschluss - Warum ist ein Unterdruckventil eingebaut?}

  \begin{itemize}
  \item
    Damit sich der Kühler bei Abkühlung nicht einbeult
  \end{itemize}
\item
  \textbf{Thermostat im Kühlsystem (Aufgabe)}

  \begin{itemize}
  \item
    Er steuert die Kühlflüssigkeitsmenge, die den Kühler durchströmt.
  \end{itemize}
\item
  \textbf{Kühlflüssigkeit immer wieder nachfüllen (mögliche Ursache)}

  \begin{itemize}
  \item
    Das Überdruckventil des Verschlussdeckel ist defekt
  \end{itemize}
\item
  \textbf{Kühlwasserthermostat schließt nicht mehr (Auswirkung)}

  \begin{itemize}
  \item
    Der Motor hat besonders im Winter einen höheren Kraftstoffverbrauch,
    da der Motor seine Betriebstemperatur nicht erreicht.
  \end{itemize}
\item
  \textbf{Siedende Kühlflüssigkeit Welche Fehler kann vorliegen?}

  \begin{itemize}
  \item
    Das Überdruckventil des Kühlerverschlusses ist undicht
  \end{itemize}
\item
  \textbf{Aufgabe der Motorkühlung}

  \begin{itemize}
  \item
    Überschüssige Verbrennungswärme, die auf Modobauteile und auf das
    Motoröl übergegangen ist, an die Umgebungsluft abzuführen.
  \end{itemize}
\item
  \textbf{Kühlung Was muss gekühlt werden?}

  \begin{itemize}
  \item
    Die Hitzebeständigkeit der Werkstoffe ist begrenzt
  \item
    die Schmierfähigkeit des Motoröls ist bei zu hoher Temperatur nicht
    gewährleistet
  \item
    bei Otto-Motoren könnte klopfende Verbrennung auftreten
  \end{itemize}
\item
  \textbf{Ventilator Aufgabe}

  \begin{itemize}
  \item
    Er soll den Kühler und den Motorraum mit ausreichender Kühlluftmenge
    durchströmen, wenn das Fahrzeug langsam fährt oder der Motor bei
    stehenden Fahrzeug läuft.
  \end{itemize}
\item
  \textbf{Aufgabe der Kühler}

  \begin{itemize}
  \item
    Die von der Kühlflüssigkeit im Motor aufgenommene Wärme an die
    Umgebungsluft abführen.
  \end{itemize}
\item
  \textbf{Temperaturregler (Thermostat) Aufgabe}

  \begin{itemize}
  \item
    Er sorgt dafür, dass der Motor schnell seine Betriebstemperatur
    erreicht und während des Betriebs möglichst gleichmäßig hält.
  \end{itemize}
\item
  \textbf{Zu hohe Kühlflüssigkeitstemperatur (Fehlermöglichkeiten)}

  \begin{itemize}
  \item
    Kühlflüssigkeitsverlust
  \item
    Defekter oder nicht ausreichend gespannter Keilriemen
  \item
    Defekter Thermostat
  \item
    Stark verschmutzter Kühler
  \item
    Defekter Lüfter
  \end{itemize}
\item
  \textbf{Prüfschritte, wenn die Kühlflüssigkeit im Fahrzeugbetrieb zu
  heiß wird.}

  \begin{itemize}
  \item
    Kühlmittelstand und Keilriemen prüfen
  \item
    Dichtigkeit prüfen
  \item
    Lüfterfunktion prüfen
  \item
    Thermostat auf Funktion prüfen
  \item
    Durchflussmenge prüfen
  \item
    Fehlerspeicher auslesen
  \end{itemize}
\item
  \textbf{Nachfüllen von Kühlflüssigkeit Was ist zu beachten?}

  \begin{itemize}
  \item
    Kalte Flüssigkeit darf nur in den Ausgleichsbehälter bzw. Kühler
    geschüttet werden, wenn der Motor läuft. Die kalte Flüssigkeit ist
    langsam einzugießen, damit gefährliche Spannungen im Motorblock und
    Zylinderkopf vermieden werden.
  \end{itemize}
\item
  \textbf{Wie kann ein Thermostat auf Funktion überprüft werden?}

  \begin{itemize}
  \item
    Er wird im ausgebauten Zustand im Wasserbad auf Funktion geprüft.
    Das Wasser wird langsam erhitzt. Mit dem Thermometer wird der
    Öffnungsbeginn des Thermostaten überprüft.
  \end{itemize}
\end{enumerate}

\subsubsection{Motorschmierung}\label{motorschmierung}

\begin{enumerate}
\item
  Ölverdünnung\\
\item
  Ölverdickung\\
\item
  Bauart der Motorschmierung\\
\item
  Motorschmierung\\
\item
  Überströmventil des Ölfilters\\
\item
  Druckbegrenzungsventil\\
\item
  Öldruckschalter\\
\item
  Ölpumpe\\
\item
  Bezeichnungen zuordnen\\
\item
  Ölfilteranordnung\\
\item
  Zweck der Schmierung\\
\item
  Aufgabe des Schmieröls\\
\item
  Alterung des Schmieröls\\
\item
  Ölverdünnung\\
\item
  Ölwechsel\\
\item
  Ölkreislauf\\
\item
  Ölkühlung\\
\item
  Ölsensor
\end{enumerate}

\subsubsection{Motorsteuerung}\label{motorsteuerung}

\begin{enumerate}
\item
  Aufgabe der Motorsteuerung\\
\item
  Drehrichtung des Motors\\
\item
  Bauteile der Motorsteuerung\\
\item
  Merkmale eines dohc-Motors\\
\item
  Motorsteuerung\\
\item
  Zylinderkopf eines ohc-Motors\\
\item
  Bimetall-Ventile\\
\item
  Beanspruchung\\
\item
  Zu kleines Ventilspiel\\
\item
  Zu großes Ventilspiel\\
\item
  Ventilfedern\\
\item
  Spielfreie, selbstnachstellende Ventileinstellung\\
\item
  Antriebsarten für Nockenwellen\\
\item
  Gliederketten vs.~Zahnriemen\\
\item
  Undichte Ventilschaftdichtung\\
\item
  Hydraulischer Ventilspielausgleich\\
\item
  Aufgaben der Motorsteuerung\\
\item
  Übersetzung von der Kurbelwelle zur Nockenwelle\\
\item
  Spielfreie, selbstnachstellende Ventilspieleinstellung\\
\item
  Nockenwellenantrieben
\end{enumerate}

\subsubsection{Füllungsoptimierung}\label{fuellungsoptimierung}

\begin{enumerate}
\item
  Aufgeladener Motor\\
\item
  Aufladesystem\\
\item
  Leistung eines Verbrennungsmotors\\
\item
  Vorteile des aufgeladenen Motors\\
\item
  Aufladung von Motoren\\
\item
  Abgasturbolader\\
\item
  Verdichter\\
\item
  Ladeluftkühler\\
\item
  Mehrventiltechnik\\
\item
  Ladermotor\\
\item
  Aufgeladene Motoren\\
\item
  Abgasturbolader
\end{enumerate}

\subsection{Motorbauarten}\label{motorbauarten}

\subsubsection{Otto-Zweitaktmotor}\label{otto-zweitaktmotor}

\begin{enumerate}
\item
  Hauptunterschiede: Otto-Zweitaktmotor vs.~Otto-Viertaktmotor\\
\item
  \begin{enumerate}
  \def\labelenumii{\arabic{enumii}.}
  \setcounter{enumii}{1}
  \item
    Takt\\
  \end{enumerate}
\item
  Ansaug- und Auspuffanlagen von Zweitaktmotoren\\
\item
  Otto-Zweitaktmotor vs.~Otto-Viertaktmotor\\
\item
  Zweitaktmotor\\
\item
  Offener Gaswechsel\\
\item
  Begriffe zuordnen\\
\item
  Kurbelwellenumdrehungen, Kolbenhübe\\
\item
  Gaswechsel\\
\item
  Stark verrußte Auspuffanlage\\
\item
  Motorschmierung\\
\item
  Kurbelkammer des Zweitaktmotors
\end{enumerate}

\subsubsection{Kreiskolbenmotor}\label{kreiskolbenmotor}

\begin{enumerate}
\item
  Hauptteile des Kreiskolbenmotors\\
\item
  Kolbenumdrehung\\
\item
  Kreiskolbenmotor\\
\item
  Bauteile benennen\\
\item
  Vorgänge benennen\\
\item
  Gaswechsel Kreiskolbenmotor\\
\item
  Läuferumdrehung eines Einscheibenkreiskolbenmotors\\
\item
  Umdrehung des Kreiskolbens
\end{enumerate}

\subsubsection{Alternative
Antriebskonzepte}\label{alternative-antriebskonzepte}

\begin{enumerate}
\item
  Alternative Antriebe von Kraftfahrzeugen\\
\item
  Hybridantrieb\\
\item
  Elektromotor\\
\item
  Bivalentes Antriebssystem\\
\item
  Erneuerbare Energien\\
\item
  Elektrische Energie\\
\item
  Wasserstoff\\
\item
  CNG\\
\item
  LPG\\
\item
  Hybridantrieb\\
\item
  Regeneratives Bremsen\\
\item
  Voll-Hybrid-Fahrzeug\\
\item
  Verdampfer
\end{enumerate}

\section{Räder, Reifen}\label{raeder-reifen}

\begin{enumerate}
\item
  \textbf{Räder} (Anforderungen)

  \begin{itemize}
  \item
    geringe Masse, gute Wärmeableitung, formfest
  \end{itemize}
\item
  \textbf{Aufgaben der Bereifung}

  \begin{itemize}
  \item
    Gewichtskraft, übertragen von Antriebs-, Brems- und
    Seitenführungskraft
  \end{itemize}
\item
  \textbf{TWI}

  \begin{itemize}
  \item
    Abriebindikator, Restprofiltiefe
  \end{itemize}
\item
  \textbf{Reifentragfähigkeit $[kg]$ abhängig:}

  \begin{itemize}
  \item
    Druck, Volumen, Sturz, Geschwindigkeit, Bauart
  \end{itemize}
\item
  \textbf{Verschleißbild Profilbild >>Wellenförmige Auswaschungen<<:}

  \begin{itemize}
  \item
    Unwucht, Spiel in Lenkung oder Lager, Dämpfer
  \end{itemize}
\item
  \textbf{Hump-Felge Vorteil:}

  \begin{itemize}
  \item
    Bei niedrigen Reifendruck wird Reifen auf Felgenschulter gehalten
  \end{itemize}
\item
  \textbf{Felgenbezeichnung >>6 1/2<<}

  \begin{itemize}
  \item
    Maulweite in Zoll
  \end{itemize}
\item
  \textbf{Einpresstiefe verkleinern, dann:}

  \begin{itemize}
  \item
    vergrößert sich die Spurweite
  \end{itemize}
\item
  \textbf{Latsch}

  \begin{itemize}
  \item
    Reifenaufstandsfläche
  \end{itemize}
\item
  \textbf{Reifenkennzeichnung >>255/45 R 20 101V<<}

  \begin{itemize}
  \item
    Reifenbreite, Verhältnis, Radial, Felgendurchmesser,
    Reifentragfähigkeit, Vmax
  \end{itemize}
\item
  \textbf{Herstellungsdatum Reifen >>3620<<}

  \begin{itemize}
  \item
    Woche: 36 und Jahr: 2020
  \end{itemize}
\item
  \textbf{Verschleißbild Reifenprofil >>Abrieb Reifenmitte<<}

  \begin{itemize}
  \item
    hoher Reifendruck oder Geschwindigkeit
  \end{itemize}
\item
  \textbf{Felgenkennzeichnung >>7 1/2J x 17, ET40<<}

  \begin{itemize}
  \item
    Maulweite, Felgenhorn, Tiefbett, Felgendurchmesser, Einpresstiefe
  \end{itemize}
\item
  \textbf{Einpresstiefe}

  \begin{itemize}
  \item
    Felgenmitte bis zur Anlagefläche des Rades an Radnabe
  \end{itemize}
\item
  \textbf{Reifeninnendruck zu gering:}

  \begin{itemize}
  \item
    thermisch und mechanische Überbelastung
  \end{itemize}
\item
  \textbf{Reifenkennzeichnung >>XL oder Reinforced<<}

  \begin{itemize}
  \item
    erhöhte Tragfähigkeit
  \end{itemize}
\item
  \textbf{Run-Flat-Reifen (RFT)}

  \begin{itemize}
  \item
    Reifendruckkontrollsystem notwendig
  \item
    weiterfahrt bei Luftverlust möglich, Vmax = 80 Km/h Wegstrecke = 200
    km
  \item
    hat verstärkte Seitenwand gegenüber Normalreifen
  \end{itemize}
\item
  \textbf{Räderkennzeichnung >>LK 5/120<<:}

  \begin{itemize}
  \item
    Lochkreis 120 mm 5x Bohrungen
  \end{itemize}
\item
  \textbf{Montagefülldrücke bei Pkw-Reifen >>Springdruck, Setzdruck<<:}

  \begin{itemize}
  \item
    Springdruck: max. 3,3 bar, Setzdruck: 4 bar
  \end{itemize}
\item
  \textbf{Unwucht/Auswuchten}

  \begin{itemize}
  \item
    springt, taumelt
  \item
    Laufunruhe, Reifenverschleiß
  \end{itemize}
\item
  \textbf{Reifendruckkontrollsysteme (RDKS, Erstzulassung nach 2014)
  >>direkt, indirekt<<:}

  \begin{itemize}
  \item
    direkt: Sensor misst Luftdruck und Temperatur
  \item
    indirekt: Raddrehzahlsensor
  \end{itemize}
\end{enumerate}

\section{Grundlagen Elektrotechnik
Kfz}\label{grundlagen-elektrotechnik-kfz}

\subsection{Elektrische
Grundgrößen}\label{elektrische-grundgroessen}

\begin{enumerate}
\item
  Kleinstes Teilchen einer chemischen Verbindung\\
\item
  Kleinste, chemisch nicht mehr aufspaltbare Teilchen\\
\item
  Atommodell\\
\item
  Ladungszustand\\
\item
  Verhalten von elektrischen Ladungen\\
\item
  Definition der elektrischen Spannung\\
\item
  Minuspol einer Batterie\\
\item
  Formelzeichen und Einheit\\
\item
  Umrechnung\\
\item
  Elektrischer Strom\\
\item
  Schaltung\\
\item
  Technische Stromrichtung\\
\item
  Formelzeichen und Einheit\\
\item
  Umrechnung\\
\item
  Wechselstrom\\
\item
  Stromleitung in metallischen Leitern\\
\item
  Stromdichte\\
\item
  Querschnitt\\
\item
  Aufbau eines Atoms\\
\item
  Bestandteile eines Atomkerns
\end{enumerate}

\subsection{Spannung, Strom,Widerstand}\label{spannung-stromwiderstand}

\begin{enumerate}
\item
  Elektrischer Widerstand\\
\item
  Widerstandswert\\
\item
  Temperaturverhalten von Widerständen\\
\item
  Widerstandsart\\
\item
  PTC-Widerstände\\
\item
  NTC-Widerstände\\
\item
  Temperaturfühler\\
\item
  Temperatur und Widerstandswert\\
\item
  Umrechnung\\
\item
  Schaltzeichen für Heiß- und Kaltleiter\\
\item
  VDR-Widerstände\\
\item
  Widerstandswert eines Fotowiderstandes\\
\item
  Fotowiderstände\\
\item
  Elektrische Spannung\\
\item
  Elektrischer Strom\\
\item
  Elektrische Stromstärke\\
\item
  Stromkreis\\
\item
  Elektrischer Stromkreis\\
\item
  Elektrische Sicherungen\\
\item
  Elektrischer Widerstand\\
\item
  Kaltleiter\\
\item
  Heißleiter\\
\item
  Direkte Widerstandsmessung mit einem Multimeter
\end{enumerate}

\subsection{Ohmsches Gesetz, Leistung, Arbeit,
Wirkungsgrad}\label{ohmsches-gesetz-leistung-arbeit-wirkungsgrad}

\begin{enumerate}
\item
  Ohmsches Gesetz - Formel\\
\item
  Indirekte Widerstandsermittlung\\
\item
  Spannung und Strom im Stromkreis\\
\item
  Strom und Widerstand im Stromkreis\\
\item
  Elektrische Arbeit\\
\item
  Einheiten der elektrischen Arbeit\\
\item
  Elektrische Arbeit - Messgeräte\\
\item
  Elektrische Leistung\\
\item
  Einheiten der elektrischen Leistung\\
\item
  Wirkungsgrad\\
\item
  Ohmsches Gesetz\\
\item
  Elektrische Arbeit\\
\item
  Elektrische Leistung
\end{enumerate}

\subsection{Schaltung von
Widerständen}\label{schaltung-von-widerstaenden}

\begin{enumerate}
\item
  Reihenschaltung - elektrische Größen\\
\item
  Reihenschaltung - Formel\\
\item
  Gesamtwiderstand einer Reihenschaltung\\
\item
  Parallelschaltung - Gesamtstrom\\
\item
  Bauelement\\
\item
  Potentiometer\\
\item
  Unbelasteter Spannungsteiler - Spannung\\
\item
  Schaltung\\
\item
  Beziehung zwischen Widerständen\\
\item
  Reihenschaltung von Widerständen\\
\item
  Parallelschaltung von Widerständen
\end{enumerate}

\subsection{Messungen im elektrischen
Stromkreis}\label{messungen-im-elektrischen-stromkreis}

\begin{enumerate}
\item
  Spannungs- und Strommesser\\
\item
  Messgeräte mit digitaler Anzeige\\
\item
  Multimeter\\
\item
  Messung und Messbereich\\
\item
  Spannungs- und Strommessung\\
\item
  Elektrische Prüfgeräte\\
\item
  Diodenlampe\\
\item
  Multimeter\\
\item
  Oszilloskop
\end{enumerate}

\subsection{Wirkungen, Gefahren elektrischer
Strom}\label{wirkungen-gefahren-elektrischer-strom}

\begin{enumerate}
\item
  Auswirkungen durch Stromwirkungen\\
\item
  Erwärmung von metallischen Leitern\\
\item
  Lichtwirkung in Glühlampen\\
\item
  Lichtwirkung in einer Leuchtstofflampe\\
\item
  Elektrolyte\\
\item
  Stromleitung in Flüssigkeiten und Gasen\\
\item
  Elektrolyse\\
\item
  Galvanisieren\\
\item
  Feldlinienverlauf von Stabmagneten\\
\item
  Pole eines Stabmagneten\\
\item
  Magnetfeld einer stromdurchflossenen Spule\\
\item
  Remanenz\\
\item
  Stromrichtung und Feldlinienrichtung\\
\item
  Feldlinienrichtung und Kraftwirkung\\
\item
  Drehrichtung der stromdurchflossenen Spule\\
\item
  Wirkungen des elektrischen Stromes\\
\item
  Wärmewirkung des elektrischen Stroms\\
\item
  Magnetismus\\
\item
  Selbstinduktion\\
\item
  Spannungserzeugung
\end{enumerate}

\subsection{Spannungserzeugung,
Elektrochemie}\label{spannungserzeugung-elektrochemie}

\begin{enumerate}
\item
  Bauelemente zur Spannungserzeugung\\
\item
  Spannungserzeugung durch Induktion\\
\item
  Höhe der induzierten Spannung\\
\item
  Periodische Änderung des magnetischen Flusses\\
\item
  Induktion in einer Spule\\
\item
  Transformator\\
\item
  Transformator - Spannungswert\\
\item
  Verlustfreier Transformator\\
\item
  Stromleitung im Elektrolyten\\
\item
  Spannung in einem galvanischen Element\\
\item
  Höhe der Spannung in einem galvanischen Element\\
\item
  Spannungshöhe eines Kohle-Zink-Elements bzw. eines
  Kupfer-Aluminium-Elements\\
\item
  Elektrode\\
\item
  Erzeugung elektrischer Spannung\\
\item
  Galvanisches Element\\
\item
  Spannung in einem Fotoelement\\
\item
  Piezoelektrischer Effekt
\end{enumerate}

\subsection{Elektronische Bauelemente, Anwendungen der
Elektrotechnik}\label{elektronische-bauelemente-anwendungen-der-elektrotechnik}

\begin{enumerate}
\item
  Dioden\\
\item
  Diodenkennlinie\\
\item
  Schleusenspannung und Diodentyp\\
\item
  Z-Dioden\\
\item
  Schaltung mit Dioden\\
\item
  Kennziffern zuordnen\\
\item
  Polaritäten\\
\item
  Schaltplan\\
\item
  Transistor als Schalter\\
\item
  Kapazität eines Kondensators\\
\item
  Elektronische Bauelemente\\
\item
  Wichtigste elektronische Bauelemente\\
\item
  Halbleiter\\
\item
  Eigenschaften einer Diode\\
\item
  Transistorbauarten\\
\item
  Halbleiterzonen eines Transistors\\
\item
  Transistor als Schalter\\
\item
  Thyristor\\
\item
  Arten von elektrischen Schaltplänen\\
\item
  Blink- und Signalanlage
\end{enumerate}

\subsection{Beleuchtungsanlage, Spannungsversorgung im
Kfz}\label{beleuchtungsanlage-spannungsversorgung-im-kfz}

\subsubsection{Beleuchtungsanlage}\label{beleuchtungsanlage}

\begin{enumerate}
\item
  Lichttechnische Einrichtungen\\
\item
  Lampenarten\\
\item
  Scheinwerferlampen\\
\item
  Halogenlampen\\
\item
  Gasentladungslampen\\
\item
  Kurvenlicht\\
\item
  Scheinwerfereinstellung - Prüfbilder\\
\item
  Scheinwerfereinstellung\\
\item
  Scheinwerfersysteme für Fern- und Abblendlicht\\
\item
  Aufgaben von Leuchten\\
\item
  Bezeichnungen für die Reflektorgrundformen\\
\item
  Lampen mit Abblend-Fernlicht-Leuchtkörpern\\
\item
  Paraboloidförmiger Reflektor bei Fernlicht\\
\item
  Paraboloidförmiger Reflektor bei Ablendlicht\\
\item
  Halogenlampen vs.~Glühlampen\\
\item
  Halogenlampen - Schwärzung\\
\item
  Gasentladungslampen\\
\item
  Aufgabe der Leuchtweitenregelung\\
\item
  Klemmenbezeichnungen\\
\item
  Scheinwerfersystem\\
\item
  Scheinwerfersystem für Kurvenlicht
\end{enumerate}

\subsubsection{Batterie}\label{batterie}

\begin{enumerate}
\item
  Aufgaben einer Starterbatterie\\
\item
  Bestandteile einer Starterbatterie\\
\item
  Energieumformungen\\
\item
  Batteriekennzeichnung\\
\item
  Kapazität\\
\item
  Starthilfe\\
\item
  Laden/Entladen einer Starterbatterie\\
\item
  Nennspannung\\
\item
  Zellen einer 12-V-Starterbatterie\\
\item
  Separatoren\\
\item
  Aktive Masse einer geladenen Starterbatterie\\
\item
  Aktive Masse einer entladenen Starterbatterie\\
\item
  Säuredichte\\
\item
  Lagerung einer außer Betrieb gesetzten Starterbatterie\\
\item
  Nennkapazität\\
\item
  Kennzeichnung einer Starterbatterie\\
\item
  Spannungswerte\\
\item
  12-V-Starterbatterie\\
\item
  Kälteprüfstrom\\
\item
  Zu niedriger Säurestand\\
\item
  Gesamtspannung und Gesamtkapazität\\
\item
  Gesamtbetriebsspannung und Gesamtkapazität\\
\item
  Normalladung einer Starterbatterie\\
\item
  Ladung einer 12-V-Starterbatterie\\
\item
  Selbstentladung einer Säurebatterie\\
\item
  Gel-Batterie
\end{enumerate}

\subsubsection{Generator}\label{generator}

\begin{enumerate}
\item
  Prinzip der Spannungserzeugung\\
\item
  Gleichrichtung des Ladestromes\\
\item
  Drehstromgenerator\\
\item
  Regelung von Generatoren\\
\item
  Regelzustände\\
\item
  Multifunktionsregler\\
\item
  Technische Daten\\
\item
  Drehstromgenerator\\
\item
  Induktion in einem Drehstromgenerator\\
\item
  Klauenpolläufer\\
\item
  Verbinden von drei Einzelwicklungen zur Sternschaltung\\
\item
  Gleichrichterschaltung\\
\item
  Stromkreise eines Generators\\
\item
  Generatorkontrolllampe - Klemmen\\
\item
  Defekte Generatorkontrolllampe\\
\item
  Regler in einem Drehstromgenerator\\
\item
  Spannungsregelung bei einem Drehstromgenerator\\
\item
  Generatorspannung oberhalb des zulässigen Höchstwerts\\
\item
  Generatorspannung unterhalb der Soll-Spannung\\
\item
  Transistor\\
\item
  Prüfung\\
\item
  Oszillogramm eines Drehstromgenerators\\
\item
  Multifunktionsregler\\
\item
  Leitstückgenerator mit Flüssigkeitskühlung
\end{enumerate}

\subsubsection{Starter}\label{starter}

\begin{enumerate}
\item
  Baugruppen des Starters\\
\item
  Gleichstrommotoren\\
\item
  Aufbau und Funktion des Schub-Schraubtrieb-Starters\\
\item
  Nebenschlussmotor\\
\item
  Startdrehzahlen von Verbrennungsmotoren\\
\item
  Anschlussklemmen im Einrückrelais\\
\item
  Starterbauart\\
\item
  Bauteile\\
\item
  Freilaufsystem eines Starters\\
\item
  Aufgabe des Einrückrelais
\end{enumerate}

\section{Motormechanik}\label{motormechanik-1}

\subsection{Motorsteuerung}\label{motorsteuerung-1}

\begin{enumerate}
\item
  Aufgabe der Motorsteuerung\\
\item
  Obengesteuerte Motoren\\
\item
  Zu großes Ventilspiel\\
\item
  Obengesteuerte Motoren\\
\item
  dohc-Motor\\
\item
  ohc-Motor\\
\item
  Bauteile der Motosteuerung\\
\item
  Ventilteile\\
\item
  Einlassventile bei Mehrventilmotoren\\
\item
  Beanspruchung von Ventilen\\
\item
  Beeinflussung der Ventilsitzbreite\\
\item
  Winkel am Ventilsitz\\
\item
  Ventilsitzbreite\\
\item
  Ventileinstellung\\
\item
  Drehzahl der Nockenwelle\\
\item
  Drehzahl der Kurbelwelle\\
\item
  Nockenform\\
\item
  Hydraulischer Ventilspielausgleich\\
\item
  Undichte Ventilschaftsdichtung\\
\item
  Zahnriementrieb\\
\item
  Aufgaben der Motorsteuerung\\
\item
  Zeitpunkte der Motorsteuerung\\
\item
  Öffnungsdauer eines Ventils\\
\item
  Übersetzung von Kurbelwelle zu Nockenwelle\\
\item
  Anordnung der Nockenwelle\\
\item
  Motorsteuerungsarten\\
\item
  Motorsteuerung\\
\item
  Ventile\\
\item
  Bimetallventile\\
\item
  Vorteile von Bimetallventilen\\
\item
  Natriumgefüllte Auslassventile\\
\item
  Motor mit hydraulischem Ventilspielausgleich\\
\item
  Schadhafte Ventilschaftsabdichtung\\
\item
  Breite des Ventilsitzes\\
\item
  Aufgabe der Ventilfedern\\
\item
  Tassenstößel\\
\item
  Nockenwellenantriebe\\
\item
  Zahnriementriebe\\
\item
  Aufbau eines Ventils\\
\item
  Defekte Ventilschaftabdichtungen\\
\item
  Mehrventiltechnik\\
\item
  Motoren mit Mehrventiltechnik\\
\item
  Hydraulischer Ventilspielausgleich\\
\item
  Ventilführung\\
\item
  Mehrventiltechnik\\
\item
  Zahnriemenantrieb\\
\item
  Kettenantrieb
\end{enumerate}

\subsection{Füllungsoptimierung}\label{fuellungsoptimierung-1}

\begin{enumerate}
\item
  Aufgeladener Motor\\
\item
  Aufladesystem\\
\item
  Vorteile des aufgeladenen Motors\\
\item
  Aufladung von Motoren\\
\item
  Serien-Dieselmotoren\\
\item
  Abgasturbolader\\
\item
  Laufzeug eines Abgasturboladers\\
\item
  Abgasturbolader\\
\item
  Aufbau eines Abgasturboladers\\
\item
  Bauteile eines Aufladesystems\\
\item
  Verwendung von Ladeluftkühlern\\
\item
  Ladeluftkühlung\\
\item
  Ladedruckregelung\\
\item
  Kennlinien von hubraumgleichem Saugmotor und Ladermotor\\
\item
  Verstellbare Leitschaufeln\\
\item
  Motorkennlinien\\
\item
  Verstellbarer Abgasturbolader\\
\item
  Verdichtung der Ladeluft\\
\item
  VTG-Lader\\
\item
  VTG-Lader Bauteile\\
\item
  VTG-Lader Funktionsweise\\
\item
  Laderbauart\\
\item
  Diagramm\\
\item
  Nockenwellenverstellung\\
\item
  Dynamisches Aufladungssystem\\
\item
  Dynamische Aufladung\\
\item
  Saugrohrlänge bei Schwingsaugrohraufladung\\
\item
  Aufladung\\
\item
  Overboost\\
\item
  Schwingsaugrohrsysteme\\
\item
  Drehmomentzuwachs\\
\item
  Variabler hydraulischer Ventiltrieb mit schaltbaren Tassenstößeln\\
\item
  Nockenwellenverstellung\\
\item
  Nockenwellenverstellung\\
\item
  Nockenwellenverstellung\\
\item
  Flügelzellenversteller\\
\item
  Variabler Ventiltrieb\\
\item
  Mechanische Aufladung\\
\item
  Elektrischer Turbolader\\
\item
  Mehrventiltechnik\\
\item
  Vierventiler\\
\item
  Variable Steuerzeiten\\
\item
  Erzeugung variabler Steuerzeiten\\
\item
  Einlasssteuerzeiten\\
\item
  Verstellung der Einlassnockenwelle\\
\item
  Variabler Ventiltrieb und Nockenwellenverstellung\\
\item
  Liefergrad\\
\item
  Laderbauarten\\
\item
  Ladeluftkühlung\\
\item
  Ladedrücke von Ladermotoren\\
\item
  Umluftventil\\
\item
  Leitschaufelverstellung -- VTG-Lader\\
\item
  Doppelaufladung\\
\item
  Registeraufladung\\
\item
  Abgasturbolader und Kompressor\\
\item
  Resonanzaufladung\\
\item
  Resonanzaufladung\\
\item
  Signalfolge\\
\item
  Vollvariabler elektromechanischer Ventiltrieb\\
\item
  Vollvariabler Ventiltrieb\\
\item
  Wirkungsablauf des vollvariablen elektromechanischen Ventiltriebs\\
\item
  Kombination von Abgasturbolader und Kompressor\\
\item
  Vergleich füllungsoptimierter und nicht füllungsoptimierter Motor\\
\item
  Flügelzellenversteller\\
\item
  Doppelaufladungssystem\\
\item
  Registeraufladung\\
\item
  Doppelvanos-System\\
\item
  Nockenwellenverstellung
\end{enumerate}

\section{Motormanagement Ottomotor}\label{motormanagement-ottomotor}

\subsection{Grundlagen der
Gemischbildungssysteme}\label{grundlagen-der-gemischbildungssysteme}

\begin{enumerate}
\item
  Theoretisches Mischungsverhältnis\\
\item
  Luftverhältnis $\lambda$\\
\item
  Kraftstoff-Luft-Gemisch\\
\item
  Benzin-Luft-Gemisch\\
\item
  Homogenes Gemisch\\
\item
  Heterogenes Gemisch\\
\item
  Fettes Kraftstoff-Luftgemisch\\
\item
  Folgen eines zu fetten Kraftstoff-Luft-Gemisches\\
\item
  Betriebsbedinungen\\
\item
  Mischungsverhältnis des Kraftstoff-Luft-Gemisches\\
\item
  Gemischbildung Ottomotor\\
\item
  Qualitätsregelung\\
\item
  Quantitätsregelung
\end{enumerate}

\subsection{Kraftstoffversorgungsanlagen bei
Ottomotoren}\label{kraftstoffversorgungsanlagen-bei-ottomotoren}

\begin{enumerate}
\item
  Kraftstoffversorgungsanlage\\
\item
  Aktivkohlefilter\\
\item
  Schwerkraftventil\\
\item
  Elektrisch angetriebene Kraftstoffpumpen\\
\item
  Saugstrahlpumpen\\
\item
  Kraftstofffördersystem\\
\item
  Kraftstoffpumpenrelais\\
\item
  Funktion des Kraftstoffpumpenrelais überprüfen\\
\item
  Kraftstoffpumpe\\
\item
  Aufgabe der Kraftstoff-Förderanlage\\
\item
  Kraftstoff-Förderanlage\\
\item
  Belüftung des Kraftstoffbehälters\\
\item
  Catch-Tank\\
\item
  Aktivkohlefilter\\
\item
  Kraftstoffleitungen\\
\item
  Regenerierventil\\
\item
  Zuleitung der Saugstrahlpumpe\\
\item
  Kraftstoffförderanlage\\
\item
  Kraftstoffförderanlage\\
\item
  Kraftstoffpumpen
\end{enumerate}

\subsection{Benzineinspritzung}\label{benzineinspritzung}

\subsubsection{Aufbau und Funktion der elektronischen
Benzineinspritzung}\label{aufbau-und-funktion-der-elektronischen-benzineinspritzung}

\begin{enumerate}
\item
  Aufgaben der Benzineinspritzung\\
\item
  Sequentielle Einspritzung\\
\item
  ME-Motronic\\
\item
  ME-Motronic, Sensoren, Aktoren\\
\item
  ME-Motronic, Steuergrößen\\
\item
  Kaltstartanreicherung\\
\item
  ME-Motronic, Leerlauffüllungsregelung\\
\item
  Sensoren in Benzineinspritzanlagen\\
\item
  Steuergerät - $\lambda$-Sondenspannung\\
\item
  Schubabschaltung\\
\item
  Benzindirekteinspritzung, Schichtladungsbetrieb\\
\item
  Benzineinspritzung\\
\item
  Direkte Erfassung der Luftmasse\\
\item
  Benzineinspritzanlage, Bauteil zur Lasterfassung\\
\item
  Benzineinspritzanlage, Luftmassenmesser\\
\item
  Leerlauffüllungsregelung\\
\item
  Motronic\\
\item
  MED-Benzineinspritzanlage\\
\item
  Drehmomentenverlauf eines FSI-Motors\\
\item
  Elektronisches Gaspedal\\
\item
  Funktion des elektronischen Gaspedals\\
\item
  ME-Motronic-System\\
\item
  Benzinmotoren mit Direkteinspritzung\\
\item
  Vergleich Benzindirekteinspritzung - Saugrohreinspritzung\\
\item
  Schichtladung\\
\item
  Spannungsversorgung am Einspritzventil\\
\item
  Strommessung am Einspritzventil\\
\item
  Spannungsmessung am Luftmassenmesser\\
\item
  Schichtladungsbetrieb bei Ottomotoren mit Direkteinspritzung\\
\item
  Ottomotor mit strahlgeführter Direkteinspritzung\\
\item
  Ottomotor mit Direkteinspritzung - Homogenbetrieb\\
\item
  Ottomotor mit Direkteinspritzung\\
\item
  Kraftstoffversorgungsanlage bei Ottomotoren mit Direkteinspritzung\\
\item
  ME-Motronic, Einspritzventil\\
\item
  ME-Motronic, Leerlaufregelung\\
\item
  ME-Motronic, Bauteilzuordnung\\
\item
  ME-Motronic, E-Gas-System\\
\item
  ME-Motronic, Schaltplan\\
\item
  ME-Motronic, Spannungsverlauf Einspritzventil\\
\item
  Duales Benzineinspritzsystem
\end{enumerate}

\subsubsection{Zündanlagen,
Zündkerzen}\label{zuendanlagen-zuendkerzen}

\begin{enumerate}
\item
  Zündanlage\\
\item
  Zündspannungsbedarf\\
\item
  Normaloszillogramm Sekundärkreis einer Zündanlage\\
\item
  Zündspule\\
\item
  Zündzeitpunkt\\
\item
  Schließwinkel\\
\item
  Zündabstand\\
\item
  Impulsverlauf eines Hallgebersignals\\
\item
  Zündkennfeld\\
\item
  Bestimmung des Zündzeitpunkts\\
\item
  Zündaussetzer\\
\item
  Ruhende Hochspannungsverteilung\\
\item
  Ansteuerung Primärstromkreis\\
\item
  Messwiderstand bei Zündanlage mit Einzelfunkenzündspulen\\
\item
  Aufgabe Zündkerze in Ottomotoren\\
\item
  Zündkerzeneinbau\\
\item
  Spulenzündanlage\\
\item
  Zündabstand\\
\item
  Schließwinkel\\
\item
  Zündzeitpunkt\\
\item
  Zündimpulsgeber\\
\item
  Hallgeber\\
\item
  Halleffekt\\
\item
  Klopfregelung\\
\item
  Klopfsensor\\
\item
  Vollelektronische Zündanlage\\
\item
  Zündanlage\\
\item
  Aufgabe der Diode\\
\item
  Anlage mit Einzelfunkenzündspulen\\
\item
  Zylinderselektive Klopfregelung\\
\item
  Zündspannungsbedarf\\
\item
  Zündkerze\\
\item
  Zündkerzen\\
\item
  Zündkerzen-Gesichter
\end{enumerate}

\section{Schadstoffminderung}\label{schadstoffminderung}

\subsection{Abgasanlage}\label{abgasanlage}

\begin{enumerate}
\item
  Bestandteile der Abgasanlage\\
\item
  Undichtigkeiten bei Abgasanlagen\\
\item
  Abänderungen bei Abgasanlagen\\
\item
  Dezibel\\
\item
  Schalldämpferbauart\\
\item
  Schalldämpferbauart\\
\item
  Abgasanlage eines Kraftfahrzeugs\\
\item
  Abgasanlage\\
\item
  Abgasanlage mit Abgasreinigungssystem
\end{enumerate}

\subsection{Schadstoffminderung
Ottomotor}\label{schadstoffminderung-ottomotor}

\begin{enumerate}
\item
  Ungiftige Bestandteile in Abgasen\\
\item
  Schadstoffe im Abgas\\
\item
  Aufgabe Katalysator\\
\item
  Betriebstemperatur für Dreiwege-Katalysatoren\\
\item
  Katalysatorfenster/Lambda-Fenster\\
\item
  Konvertierungsrate Katalysator\\
\item
  Aufgabe der Lambdasonde\\
\item
  Lambdasonde nach Kaltstart des Motors\\
\item
  MI-Lampe\\
\item
  Vollkommene Verbrennung\\
\item
  Schadstoffwert\\
\item
  Verbrennungsprodukte\\
\item
  Dreiwegekatalysator\\
\item
  Geregeltes Gemischbildungssystem\\
\item
  Umwandlung schädlicher Abgasbestandteile\\
\item
  Abgasbestandteil\\
\item
  Lambda-Sonde\\
\item
  Luftverhältnis und Lambdasondenspannung\\
\item
  Rohemissionen des verbrannten Kraftstoff-Luft-Gemisches\\
\item
  Lambdasonde\\
\item
  System zur Minderung von NOx im Abgas\\
\item
  Abgasrückführung bei Verbrennungsmotoren\\
\item
  MIL-Anzeige\\
\item
  Diagnoseprotokoll und Diagnoseaussagen\\
\item
  Vorkatsonde, Nachkatsonde und Katalysator\\
\item
  Kennlinie Vorkatsonde (Gutbild)\\
\item
  Lambdasignal -- Gemischzuordnung\\
\item
  Signalbild Vorkatsonde\\
\item
  Signalbild gealterte Lambdasonde\\
\item
  Regelkreis Lambdasonde\\
\item
  Signalbild Nachkatsonde\\
\item
  Signalbilder Vorkat- und Nachkatsonde\\
\item
  Schadstoffkonzentration und Luftverhältnis vor dem Katalysator\\
\item
  Luftverhältnis Lambda\\
\item
  Aufgabe Katalysator\\
\item
  Katalysator\\
\item
  Signalspannung Lambdasonde\\
\item
  Lambdasonde\\
\item
  System zur Abgasreduzierung\\
\item
  Sekundärluftsystem\\
\item
  Breitbandlambdasonde\\
\item
  AU bei Fahrzeugen mit G-Kat und OBD\\
\item
  Readinesscode
\end{enumerate}

\section{Motormanagement
Dieselmotor}\label{motormanagement-dieselmotor}

\subsection{Gemischbildung und
Verbrennungsablauf}\label{gemischbildung-und-verbrennungsablauf}

\begin{enumerate}
\item
  Arbeitsweise des Dieselmotors\\
\item
  Innere Gemischbildung\\
\item
  Selbstzündung\\
\item
  Beginn des Verbrennungsablaufs\\
\item
  Qualität der Gemischbildung\\
\item
  Innere Gemischbildung bei Dieselmotoren\\
\item
  Einlasskanalsteuerung Dieselmotor\\
\item
  Innere Gemischbildung bei Dieselmotoren\\
\item
  Zündverzug\\
\item
  Nageln des Dieselmotors\\
\item
  Verbrennung Ottomotor - Dieselmotor
\end{enumerate}

\subsection{Starthilfsanlagen und
Einspritzsysteme}\label{starthilfsanlagen-und-einspritzsysteme}

\begin{enumerate}
\item
  Selbstregelnde Glühstiftkerzen\\
\item
  Glühstiftkerzen - Bauarten\\
\item
  Elektronisch geregelte Glühstiftkerze\\
\item
  Glühphasen\\
\item
  Einspritzdruck Dieselmotor\\
\item
  Kraftstoffanlage Dieselmotor\\
\item
  Einspritzanlage Dieselmotor\\
\item
  Einspritzmenge bei Common-Rail\\
\item
  Einspritzsystem Dieselmotor\\
\item
  Einspritzsystem Dieselmotor\\
\item
  Aufgabe der Vorglühanlage\\
\item
  Glühkerzenbauart\\
\item
  Selbstregelnde Glühkerze\\
\item
  Glühzeitsteuerung\\
\item
  Stromaufnahme bei selbstregelnden Glühstiftkerzen\\
\item
  Starthilfsanlagen
\end{enumerate}

\subsection{Common-Rail-Systeme}\label{common-rail-systeme}

\begin{enumerate}
\item
  Common-Rail-System\\
\item
  Common Rail\\
\item
  Raildrucksensor\\
\item
  Raildruckregelventil\\
\item
  Common-Rail-System\\
\item
  Kraftstoffverteilung bei Common-Rail Anlage\\
\item
  Öffnung Magnetventilinjektor\\
\item
  Magnetventil-Injektor\\
\item
  Common-Rail-System\\
\item
  Raildruckregelventil\\
\item
  Kraftstoffmengenregelung\\
\item
  Magnetventil-Injektor
\end{enumerate}

\subsection{Pumpe-Düse-System}\label{pumpe-duese-system}

\begin{enumerate}
\item
  Pumpe-Düse System\\
\item
  Pumpe-Düse-Element\\
\item
  Öffnungsvorgang Pumpe-Düse-Element\\
\item
  Verteilung des Kraftstoffs\\
\item
  Einspritzvorgang Pumpe-Düse Element\\
\item
  Kraftstoffmenge Pumpe-Düse-System\\
\item
  Spritzbeginn Pumpe-Düse-Element\\
\item
  Einspritzsystem
\end{enumerate}

\subsection{Schadstoffminderung bei
Dieselmotoren}\label{schadstoffminderung-bei-dieselmotoren}

\begin{enumerate}
\item
  CO Verbrennung bei Dieselmotoren\\
\item
  Abgaskomponenten\\
\item
  Partikelbildung Diesel\\
\item
  Regeneration eines Partikelfilters\\
\item
  Abgasaufbereitung SCR-System\\
\item
  Partikelbildung\\
\item
  Dieselpartikel\\
\item
  Schadstoffe im Dieselabgas\\
\item
  Reduktion von Stickoxiden\\
\item
  Partikelbildung beim Dieselmotor\\
\item
  Abgasreinigung bei Dieselmotoren\\
\item
  Dieselabgasanlage\\
\item
  Oxidationskatalysator\\
\item
  Schadstoffminderung\\
\item
  Regeneration Partikelfilter
\end{enumerate}

\section{Alternative
Antriebskonzepte}\label{alternative-antriebskonzepte-1}

\subsection{Alternative Energieträger, Teil- und Vollelektrische
Antriebe}\label{alternative-energietraeger-teil--und-vollelektrische-antriebe}

\begin{enumerate}
\item
  Vorteile von Fahrzeugantrieben mit elektrischer Energie\\
\item
  Aufbau Hybridfahrzeug\\
\item
  Regeneratives Bremsen\\
\item
  Sicherheitsregeln HV-Fahrzeug\\
\item
  Aufbau IT-Netz\\
\item
  Biodiesel\\
\item
  Erneuerbare Energien\\
\item
  Hybridsystem\\
\item
  Leistungsverzweigter Hybridantrieb\\
\item
  Serieller Hybridantrieb\\
\item
  Paralleler Hybridantrieb\\
\item
  Leistungsverzweigter Hybridantrieb\\
\item
  Schaltplan HV-Batterie (Schütze)\\
\item
  Hochvolt Vorschrift\\
\item
  Hochvoltqualifizierung\\
\item
  Fünf Sicherheitsregeln laut VDE0105\\
\item
  Sicherheitslinie\\
\item
  Sicherheitslinie\\
\item
  IT-Netz\\
\item
  Wartungsstecker\\
\item
  HV-Leitungen\\
\item
  Isolationsprüfung HV-System\\
\item
  Isolationsprüfung\\
\item
  Potentialausgleich\\
\item
  Potentialausgleich
\end{enumerate}

\subsection{Antriebe mit
Brennstoffzellen}\label{antriebe-mit-brennstoffzellen}

\begin{enumerate}
\item
  Wirkungsweise von Brennstoffzellen\\
\item
  Aufbau Brennstoffzelle\\
\item
  Aufbau Brennstoffzellenantrieb\\
\item
  Energieflüsse Brennstoffzellenantrieb\\
\item
  Funktionsprinzip Brennstoffzelle\\
\item
  Brennstoffzelle Protonentransport\\
\item
  Aufbau Brennstoffzellenstapel\\
\item
  Aufbau Wasserstoffversorgungssystem\\
\item
  Vorgang anodische Halbzelle Brennstoffzelle\\
\item
  Herausforderung anodische Halbzelle Brennstoffzelle\\
\item
  Vorgänge Anode Brennstoffzelle\\
\item
  Luftversorgungssystem Brennstoffzelle\\
\item
  Luftbefeuchter Brennstoffzelle\\
\item
  Temperaturmanagementsystem Brennstoffzelle\\
\item
  Deionisator Brennstoffzelle\\
\item
  Ursachen für Leistungsreduzierung Brennstoffzelle
\end{enumerate}

\subsection{Energiespeicherung, Ladesteckertypen,
Ladebetriebsarten}\label{energiespeicherung-ladesteckertypen-ladebetriebsarten}

\begin{enumerate}
\item
  Hybridantrieb\\
\item
  Bezeichnung Ladestecker\\
\item
  Ladesteckertypen\\
\item
  Signalleitungen Ladestecker\\
\item
  Ladekabelvarianten\\
\item
  Ladearten\\
\item
  Ladeprüfschritte\\
\item
  Laden im Hausnetz\\
\item
  Schnellladung\\
\item
  Lademodi
\end{enumerate}

\subsection{Elektrische
Antriebsmotoren}\label{elektrische-antriebsmotoren}

\begin{enumerate}
\item
  Aufbau elektrischer Antriebsmotoren\\
\item
  Merkmale elektrischer Antriebsmotoren\\
\item
  Drehrichtung Magnetfeld\\
\item
  Asynchronmaschine\\
\item
  Kippmoment Asynchronmaschine\\
\item
  Aufbau asynchroner Drehfeldmaschinen\\
\item
  Vorteile von Fahrzeugantrieben mit elektrischer Energie\\
\item
  Akkumulatoren für elektrische Antriebe
\end{enumerate}

\subsection{Arbeiten an HV-Fahrzeugen}\label{arbeiten-an-hv-fahrzeugen}

\begin{enumerate}
\item
  Sicherheitsregeln HV-Fahrzeug\\
\item
  Fahrzeug mit Hybrid-Antrieb\\
\item
  Verbotszeichen\\
\item
  Hybrid-Fahrzeuge\\
\item
  Reparatur eines HV-Fahrzeugs\\
\item
  Qualifikation eines Kfz-Mechatronikers\\
\item
  Freischaltung\\
\item
  Hybrid-Fahrzeug\\
\item
  Spannungsfreiheit eines HV-Systems\\
\item
  Sofortmaßnahmen bei Stromschlag\\
\item
  Hybridsystem\\
\item
  Austausch eines HV-Kabelstrangs\\
\item
  Auswirkungen hoher Spannungen auf den Menschen\\
\item
  Hochvoltkomponenten\\
\item
  Sicherheitskennzeichnungen\\
\item
  Vollhybrid\\
\item
  Bauteile HV-Fahrzeug\\
\item
  HV-Freischaltung bei Fahrzeug mit Hybrid-Antrieb\\
\item
  Gefahren von Fahrzeugen mit Hochvoltsystemen\\
\item
  Maximale Berührungsspannungen\\
\item
  Schutzmaßnahmen\\
\item
  Stromführende Leitungen\\
\item
  HV-Netz eines Hybridfahrzeugs\\
\item
  Gefahrenhinweise HV-Fahrzeuge\\
\item
  Körperdurchströmung\\
\item
  Strom-Gefährdungs-Kennlinie\\
\item
  Gefährlichkeit von Hochvoltanlagen\\
\item
  Körperdurchströmung\\
\item
  Rettungsmittel\\
\item
  Hybridantriebe\\
\item
  Warnhinweis auf Fahrzeugbauteilen\\
\item
  Gefährlichkeit von Hochvoltanlagen\\
\item
  Aufgabe des Verbrennungsmotors\\
\item
  Hybridantrieb\\
\item
  Nennspannung\\
\item
  Arbeiten an HV-Fahrzeugen\\
\item
  Vorteile der Hybridfahrzeuge\\
\item
  Arbeiten an unter Spannung stehenden HV-Komponenten\\
\item
  HV-Batterie-Messgerät\\
\item
  Qualifikation zum Messen der HV-Batterie\\
\item
  Ziehen des Wartungs-/Sicherheitssteckers (HV-Disconnect)\\
\item
  Hochvoltleitung\\
\item
  IT-Netz\\
\item
  Überprüfung einer HV-Leitung\\
\item
  Ladestrom und Spannung\\
\item
  Abkürzungen der Verbindungspole\\
\item
  Elektrische Innenraumheizung\\
\item
  HV-Lithium-Batterien\\
\item
  Energie-und Leistungsdichte\\
\item
  Vorteile von Lithium-Ionen-Batterien\\
\item
  Komponenten Elektrofahrzeug
\end{enumerate}

\subsection{Erdgasantrieb, Flüssiggasantriebe, Arbeiten an Fahrzeugen
mit
Gasantrieben}\label{erdgasantrieb-fluessiggasantriebe-arbeiten-an-fahrzeugen-mit-gasantrieben}

\begin{enumerate}
\item
  Erdgas\\
\item
  Systemübersicht einer LPG-Anlage\\
\item
  Aufgabe des Gasdruckreglers\\
\item
  Bauteile einer LPG-Anlage\\
\item
  Autogas/LPG\\
\item
  Flüssiggas\\
\item
  Bauteile eines CNG-Systems\\
\item
  Übersicht CNG-System\\
\item
  Flüssiggasanlage mit direkter Einspritzung\\
\item
  Flüssiggasanlage mit direkter Einspritzung (Systemübersicht)\\
\item
  Dichtheitsprüfung\\
\item
  Füllstandsregelung\\
\item
  Sensordaten der LPG-Anlage\\
\item
  Flüssiggastank\\
\item
  Gefährdungen bei Arbeiten an Fahrzeugen mit Gasantrieb\\
\item
  Schutzmaßnahmen bei Arbeiten an Fahrzeugen mit Gasantrieb\\
\item
  Gasanlagenprüfung\\
\item
  Verwendung von Biodiesel\\
\item
  Erneuerbare Energien
\end{enumerate}

\section{Antriebsstrang}\label{antriebsstrang}

\subsection{Antriebsarten}\label{antriebsarten}

\begin{enumerate}
\item
  Antriebsmöglichkeiten bei Personen- und Nutzkraftfahrzeugen\\
\item
  Vorteile von Hinterradantrieben\\
\item
  Vorteile von Mittelmotorantrieben\\
\item
  Unterflurmotor-Antrieb\\
\item
  Allradantriebe\\
\item
  Zentrale Ausgleichsmöglichkeit bei Allradantrieben\\
\item
  Aufgaben des Wechselgetriebes\\
\item
  Hinterradantriebe mit Frontmotor\\
\item
  Antriebsarten
\end{enumerate}

\subsection{Kupplung}\label{kupplung}

\begin{enumerate}
\item
  Arten von Kupplungen\\
\item
  Unterbrechung des Kraftflusses\\
\item
  Einscheiben-Reibungskupplung\\
\item
  Aufgaben der Kupplungsscheibe\\
\item
  Reibungsart\\
\item
  Reibungskraft\\
\item
  Übertragbares Drehmoment einer Reibungskupplung\\
\item
  Weiches Einkuppeln\\
\item
  Eigenschaften von Kupplungsbelägen\\
\item
  Arten von Kupplungsbelägen\\
\item
  Hydraulische Kupplungsbetätigung\\
\item
  Aufgabe des Geberzylinders\\
\item
  Aufgabe des Nehmerzylinders\\
\item
  Störungen bei Kupplungen\\
\item
  Auswechseln einer Kupplungsscheibe\\
\item
  Verölen von Kupplungsbelägen\\
\item
  SAC-Kupplung\\
\item
  Kupplung\\
\item
  Bauteile Membranfederkupplung\\
\item
  Verbindung von Kupplungsscheibe und Getriebeabtriebswelle\\
\item
  Aufgabe der Membranfeder einer Reibungskupplung\\
\item
  Kupplungsbauart von Personenkraftwagen\\
\item
  Lüftungsspiel einer Reibungskupplung\\
\item
  Auskuppeln\\
\item
  Kraftfluss\\
\item
  Erhöhung des übertragbaren Kupplungsdrehmomentes\\
\item
  Membranfederkupplung\\
\item
  Nasse Kupplung\\
\item
  Geberzylinder einer hydraulischen Kupplungsbetätigung\\
\item
  Kupplungsbetätigung\\
\item
  Bauteile der hydraulischen Kupplungsbetätigung\\
\item
  Kupplung trennt nicht
\end{enumerate}

\subsection{Wechselgetriebe, Handgeschaltete
Wechselgetriebe}\label{wechselgetriebe-handgeschaltete-wechselgetriebe}

\begin{enumerate}
\item
  Aufgaben des Wechselgetriebes\\
\item
  Schalten der Gänge in einem Wechselgetriebe\\
\item
  Arten von Wechselgetrieben\\
\item
  Mögliche Fehler eines Wechselgetriebes\\
\item
  Aufgabe der Synchronisiereinrichtung\\
\item
  Vorteile der Mehrfach-Synchronisation\\
\item
  Kraftfluss im Wechselgetriebe\\
\item
  Kraftfluss im 5-Gang-Getriebe\\
\item
  Motordrehmoment bei Verbrennungsmotoren\\
\item
  Maximale Steigung\\
\item
  Geschwindigkeitsbereich Wechselgetriebe\\
\item
  Übersetzung im Wechselgetriebe\\
\item
  Wechselgetriebe\\
\item
  Kraftfluss im Getriebe\\
\item
  Übersetzung der Vorwärtsgänge\\
\item
  Synchronisiereinrichtung (Innensynchronisation)\\
\item
  Stellung der Synchronisiereinrichtung\\
\item
  Gleichlauf im Schaltgetriebe\\
\item
  Borg-Warner-Synchronisiereinrichtung\\
\item
  Schadhafter Synchronring\\
\item
  Synchronisierung\\
\item
  Wechselgetriebe\\
\item
  Schalten des 5. Ganges
\end{enumerate}

\subsection{Automatische Getriebe}\label{automatische-getriebe}

\begin{enumerate}
\item
  Automatische Getriebe\\
\item
  Automatisierte Getriebe (DSG)\\
\item
  DSG - Elektrohydraulische Steuereinheit\\
\item
  Direktschaltgetriebe\\
\item
  Drehmomentwandler\\
\item
  Drehmomentwandler, Kupplungspunkt\\
\item
  Drehmomentwandler, Wandler-Überbrückungskupplung\\
\item
  Planetenradsatz\\
\item
  Schaltelemente in Planetengetrieben\\
\item
  Ölpumpe in Automatikgetrieben\\
\item
  Adaptive Getriebesteuerung (AGS)\\
\item
  Hydraulische Schaltventile\\
\item
  Automatikgetriebe, Haupsteuergrößen\\
\item
  Überschneidungssteuerung\\
\item
  Parksperre\\
\item
  Einfacher Planetenradsatz - Kraftfluss\\
\item
  Planetengetriebe\\
\item
  Planetenradsatz\\
\item
  Ravigneauxsatz\\
\item
  Planetenradsatz\\
\item
  Simpson-Planetenradsatz\\
\item
  Baugruppe Automatikgetriebe\\
\item
  Drehmomentwandler\\
\item
  Drehmomentwandler\\
\item
  Drehmomentwandler\\
\item
  Drehmomentwandler\\
\item
  Drehmomentwandler, Wandler-Überbrückungskupplung\\
\item
  Gestufte vollautomatische Getriebe\\
\item
  Automatikgetriebe, Hauptsteuergrößen\\
\item
  Wählhebel\\
\item
  Elektronisch gesteuerte automatische Getriebe\\
\item
  Funktion Kick-Down\\
\item
  Magnetventil und Schaltventil im Automatikgetriebe\\
\item
  Direktschaltgetriebe\\
\item
  8-Gang-Automatikgetriebe\\
\item
  8- Gang-Automatikgetriebe, Kraftfluss 1. Gang\\
\item
  Stufenloses-Automatikgetriebe\\
\item
  Stufenloses-Automatikgetriebe\\
\item
  Stufenloses Automatik-Getriebe (CVT) mit Schubgliederband
\end{enumerate}

\subsection{Gelenkwellen, Antriebswellen,
Gelenke}\label{gelenkwellen-antriebswellen-gelenke}

\begin{enumerate}
\item
  Teile der Gelenkwelle\\
\item
  Trockengelenke\\
\item
  Einbau von Gelenkwellen und Gelenken\\
\item
  Gelenke\\
\item
  Aufgabe eines Antriebsgelenks\\
\item
  Homokinetisches Gelenk\\
\item
  Beugungswinkel bei Gleichlaufgelenken\\
\item
  Gleichlauf an Gelenkwellen\\
\item
  Gelenkwellen\\
\item
  Gleichlaufgelenke\\
\item
  Antriebswelle
\end{enumerate}

\subsection{Achsgetriebe}\label{achsgetriebe}

\begin{enumerate}
\item
  Aufgaben des Achsgetriebes\\
\item
  Differenziale beim Allradantrieb\\
\item
  Kegelrad-Achsgetriebe\\
\item
  Vorteile von Achsgetrieben mit versetzten Achsen\\
\item
  Einsatz von Getriebeölen\\
\item
  Umlenkung des Kraftflusses\\
\item
  Aufgaben eines Achsgetriebes\\
\item
  Übersetzungen bei Pkw-Achsgetrieben\\
\item
  Bauteile Achsgetriebe/Differenzial\\
\item
  Notwendigkeit des Ausgleichsgetriebes\\
\item
  Funktion Ausgleichsgetriebe\\
\item
  Ausgleichsgetriebe
\end{enumerate}

\subsection{Ausgleichssperren}\label{ausgleichssperren}

\begin{enumerate}
\item
  Sperren\\
\item
  Ausgleichssperren\\
\item
  Sperrwirkung bei Sperrdifferenzialen\\
\item
  Selbstsperrendes Ausgleichsgetriebe\\
\item
  Elektronisches Sperrdifferenzial ESD\\
\item
  Elektromechanisch betätigtes Sperrdifferenzial\\
\item
  Elektromechanisch betätigtes Sperrdifferenzial\\
\item
  Aktives Sperrdifferenzial\\
\item
  Aktives Sperrdifferenzial\\
\item
  Torsen-Differenzial als Längssperre\\
\item
  Ausgleichssperre mit Lamellenkupplung\\
\item
  Kraftfluss bei Ausgleichssperre
\end{enumerate}

\subsection{Allradantrieb}\label{allradantrieb}

\begin{enumerate}
\item
  Vorteile eines Fahrzeuges mit permanentem Allradantrieb\\
\item
  Allrad-Antriebssysteme\\
\item
  Aufgabe eines Verteilergetriebes\\
\item
  Aufgabe des zentralen Ausgleichsgetriebes\\
\item
  Aggregate bei permanentem Allradantrieb\\
\item
  Drehmomentverteilung Mittendifferenzial\\
\item
  Allradsystem mit Haldex-Kupplung\\
\item
  Allradantrieb\\
\item
  Permanenter Allradantrieb\\
\item
  Torsen-Differenzial\\
\item
  Allradantrieb\\
\item
  Aggregate Allradantrieb\\
\item
  Planetengetriebe als Mittendifferenzial\\
\item
  Mittendifferenzial\\
\item
  Mittendifferenzial\\
\item
  Kronenraddifferenzial\\
\item
  Asymmetrische Drehmomentverteilung beim Kronenraddifferenzial\\
\item
  Baugruppen des Allradantriebes\\
\item
  Haldex-Kupplung\\
\item
  Bauteile Haldex-Kupplung\\
\item
  Funktion Haldex-Kupplung\\
\item
  Bauteilbezeichnungen Allradantrieb\\
\item
  Differenzialsperren bei Allradantrieben\\
\item
  Bauteile Differenzialsperre xDrive\\
\item
  Kraftfluss im xDrive- System\\
\item
  Kraftfverteilung im xDrive- System
\end{enumerate}

\section{Fahrwerk}\label{fahrwerk}

\subsection{Fahrdynamik}\label{fahrdynamik}

\begin{enumerate}
\item
  Raumachsen am Fahrzeug\\
\item
  Symmetrieachse\\
\item
  Geometrische Fahrachse\\
\item
  Fahrverhalten Kurvenfahrt\\
\item
  Schwingungsart\\
\item
  Untersteuern\\
\item
  Negativer Sturz\\
\item
  Positiver und negativer Sturz\\
\item
  Sturz\\
\item
  Vorderradaufhängung\\
\item
  Fahrwinkel\\
\item
  Negativer Sturz\\
\item
  Lenkrollradius\\
\item
  Spur
\end{enumerate}

\subsection{Lenkung}\label{lenkung}

\subsubsection{Grundlagen der Lenkung,
Lenkgetriebe}\label{grundlagen-der-lenkung-lenkgetriebe}

\begin{enumerate}
\item
  Lenkung eines Kraftfahrzeugs\\
\item
  Aufgaben der Lenkung\\
\item
  Auswirkung der Verwendung eines Lenktrapezes\\
\item
  Lenkungsbauart\\
\item
  Lenktrapez\\
\item
  Aufgaben des Lenkgetriebes\\
\item
  Lenkung\\
\item
  Aufgabe des Lenkgetriebes\\
\item
  Lenkgetriebe\\
\item
  Bauteile des Lenkgetriebes\\
\item
  Rückstellkräfte\\
\item
  Übersetzung des Lenkgetriebes\\
\item
  Variable Übersetzung\\
\item
  Unterstützungskraft
\end{enumerate}

\subsubsection{Hilfskraftlenksysteme, elektrohydraulische,- elektrische
Servolenkung}\label{hilfskraftlenksysteme-elektrohydraulische--elektrische-servolenkung}

\begin{enumerate}
\item
  Hilfskraftlenksystem\\
\item
  Bauteile elektrische Servolenkung\\
\item
  Servolenkung\\
\item
  Elektronisch geregelte Servolenkung\\
\item
  Lenkkraftunterstützung\\
\item
  Radstellung Kurvenfahrt Lenkung\\
\item
  Lenksystem\\
\item
  Servounterstützung Servoelectric\\
\item
  EVA Prinzip Servoelectric
\end{enumerate}

\subsubsection{Überlagerungs,-
Hinterachs,-Allradlenkung}\label{ueberlagerungs--hinterachs-allradlenkung}

\begin{enumerate}
\item
  Lenkung\\
\item
  Fahrzustände Aktivlenkung\\
\item
  Überlagerungslenkung Lenkübersetzung
\end{enumerate}

\subsubsection{Radstellungen}\label{radstellungen}

\begin{enumerate}
\item
  Spreizung\\
\item
  Lenkrollradius\\
\item
  Spurdifferenzwinkel\\
\item
  Postiver Nachlaufwinkel\\
\item
  Spurdifferenzwinkel\\
\item
  Schräglaufwinkel\\
\item
  Radstellung des Fahrzeugs bei Kurvenfahrt\\
\item
  Radeinstellgröße\\
\item
  Vergrößerung der Spurweite\\
\item
  Bezugsgrößen am Fahrwerk\\
\item
  Achsvermessungsprotokoll\\
\item
  Begriffe am Fahrwerk
\end{enumerate}

\subsubsection{Fahrwerksvermessung}\label{fahrwerksvermessung}

\begin{enumerate}
\item
  Achsvermessung\\
\item
  Spurdifferenzwinkel an den Vorderrädern\\
\item
  Radeinstellungsgröße\\
\item
  Richtig eingestellte Spurwerte\\
\item
  Werkstatteinrichtungen für Achsvermessungen\\
\item
  Bremsenspanner bei Achsvermessungen\\
\item
  Einstellarbeiten bei der Achsvermessung\\
\item
  3D-Achsvermessung
\end{enumerate}

\subsection{Radaufhängungen, Federung, Schwingungsdämpfer,
Federung}\label{radaufhaengungen-federung-schwingungsdaempfer-federung}

\begin{enumerate}
\item
  Achskonstruktionen\\
\item
  Halbstarrachsen\\
\item
  Verbundlenkerachse\\
\item
  Gefederte, ungefederte Massen\\
\item
  Aufgaben von Federung und Dämpfung\\
\item
  Gefederte und ungefederte Massen\\
\item
  Gedämpfte Schwingung\\
\item
  Federarten\\
\item
  Progressive Kennlinie\\
\item
  Luftfederung\\
\item
  Vorteile einer Luftfederung\\
\item
  Schwingungsdämpferbauarten\\
\item
  Vorteile von Einrohr-Gasdruckdämpfern\\
\item
  Überprüfung von Stoßdämpfern\\
\item
  Aktives Fahrwerkssystem\\
\item
  Radaufhängung\\
\item
  Sturzänderung\\
\item
  U-förmiger Stabilisator\\
\item
  Achsbauarten\\
\item
  Verbundlenkerachse\\
\item
  Radaufhängung an einer McPhersonachse\\
\item
  Frequenz der Feder\\
\item
  Progressive Federkennlinie\\
\item
  Ungefederte Massen\\
\item
  Federkennlinie\\
\item
  Fahrzeugfederung\\
\item
  Progressive Federkennlinie\\
\item
  Hydropneumatische Feder\\
\item
  Federbein\\
\item
  Fahrzeug mit Luftfederdämpfer\\
\item
  Hydractives Fahrwerk\\
\item
  Active Body Control\\
\item
  Fahrwerkssystem\\
\item
  Beschleunigungsvorgang beim ABC Fahrwerk\\
\item
  Defekter Stoßdämpfer\\
\item
  Stoßdämpfer\\
\item
  Zweirohr-Gasdruckdämpfer\\
\item
  Einrohr-Stoßdämpfer\\
\item
  Zweirohr-Schwingungsdämpfer\\
\item
  Vorteil des Einrohr-Stoßdämpfers\\
\item
  Magnetic Ride
\end{enumerate}

\subsection{Bremsanlage}\label{bremsanlage}

\subsubsection{Grundlagen, Hauptzylinder,
Radzylinder}\label{grundlagen-hauptzylinder-radzylinder}

\begin{enumerate}
\item
  Aufgaben der Betriebsbremsanlage\\
\item
  Sicherheitsvorteil der Zweikreisbremsanlage\\
\item
  Aufgaben des Hauptzylinders\\
\item
  Zentralventil\\
\item
  Hauptzylinder\\
\item
  Tandemhauptzylinder\\
\item
  Primärmanschette des Hauptzylinders\\
\item
  Bauteile Hauptzylinder\\
\item
  Füllscheibe des Hauptzylinders\\
\item
  Leck im Bremskreis\\
\item
  Bremsbelag\\
\item
  Bremsflüssigkeit\\
\item
  Bremskraftverstärker\\
\item
  Kamm'scher Reibkreis\\
\item
  ABS-Arbeitsbereich\\
\item
  Aktive Raddrehzahlsensoren\\
\item
  Zweikreisbremsanlage\\
\item
  Bremsanlage mit II (TT)-Aufteilung\\
\item
  Bremsbacken\\
\item
  Bremsfading
\end{enumerate}

\subsubsection{Trommel-und Scheibenbremse,
Feststellbremse}\label{trommel-und-scheibenbremse-feststellbremse}

\begin{enumerate}
\item
  Feststellbremse\\
\item
  Merkmale einer Trommelbremse\\
\item
  Arten von Scheibenbremsanlagen\\
\item
  Merkmale einer Scheibenbremse\\
\item
  Bremskolben der Scheibenbremse\\
\item
  Lüftspiel der Scheibenbremse\\
\item
  Vorteil der Scheibenbremse\\
\item
  Nachgebendes Bremspedal\\
\item
  Merkmale von Trommelbremsen\\
\item
  Trommelbremse\\
\item
  Scheibenbremse\\
\item
  Scheibenbremsbeläge\\
\item
  Bremsscheibendicke\\
\item
  Bezeichnungen Scheibenbremse\\
\item
  Beurteilung Bremsscheibenstärke\\
\item
  Beurteilung Belagstärke Scheibenbremse\\
\item
  Beurteilung Belagstärke Trommelbremse
\end{enumerate}

\subsubsection{Hilfskraftbremse}\label{hilfskraftbremse}

\begin{enumerate}
\item
  Hilfskraftbremsanlage\\
\item
  Bremskraftverstärker\\
\item
  Unterdruck-Bremskraftverstärker in Lösestellung\\
\item
  Bremsstellung des Bremskraftverstärkers BKV\\
\item
  Elektro-mechanischer Bremskraftverstärker eBKV\\
\item
  Verzögerung beim Elektro-mechanischem Bremskraftverstärker eBKV\\
\item
  Elektro-mechanischer Bremskraftverstärker eBKV\\
\item
  Blended Braking
\end{enumerate}

\subsubsection{Elektronische Fahrwerk-Regelsysteme, Grundlagen, ABS,
EBV,
ESP,SBC,BAS}\label{elektronische-fahrwerk-regelsysteme-grundlagen-abs-ebv-espsbcbas}

\begin{enumerate}
\item
  Vorteile des Anti-Blockier-Systems\\
\item
  Regelphasen des ABS\\
\item
  Druckaufbau eines ABS-Systems\\
\item
  Aufgabe des Bremsassistents BAS\\
\item
  Fahrdynamikregelung\\
\item
  Wirkung des ESP/FDR-Systems\\
\item
  Fehler an der hydraulischen Bremse\\
\item
  Nieder-Hochdruckprüfgerät\\
\item
  Bremsassistent\\
\item
  Unterdruck-Bremskraftverstärker\\
\item
  Bremssystem\\
\item
  Aufgabe von Anti-Blockier-Systemen\\
\item
  Vorteile des Anti-Blockier-Systems\\
\item
  Anti-Blockier-System\\
\item
  Regelphasen des Anti-Blockier-Systems\\
\item
  Select-low-Prinzip beim Anti-Blockier-System\\
\item
  ABS Regelkreis\\
\item
  Anti-Blockier-System Regelphasen\\
\item
  Magnetventil eines Anti-Blockier-Systems\\
\item
  Funktion der Magnetventile\\
\item
  2/2-Wegeventile in ABS Anlagen\\
\item
  Hydraulikkreislauf Anti-Blockier-System\\
\item
  Druckregelphasen des ABS-Systems\\
\item
  ABS/ASR-Anlage\\
\item
  Schaltplan ABS-System Druckabbau\\
\item
  Sensor\\
\item
  Drehzahlfühler HL\\
\item
  ABS-System\\
\item
  Antriebsschlupfregelungen\\
\item
  Elektronisches Fahrpedal beim ASR-System\\
\item
  Antriebsschlupfregelung ASR\\
\item
  Fahrsituationen FDR / ESP\\
\item
  Übersteuern des Fahrzeugs\\
\item
  Fahrdynamikregelsystem FDR\\
\item
  Verzögerung von Hybrid- und Elektrofahrzeugen
\end{enumerate}

\section{Fahrzeugaufbau}\label{fahrzeugaufbau}

\subsection{Fahrzeugaufbau /
Karosserie}\label{fahrzeugaufbau-karosserie}

\begin{enumerate}
\item
  Rahmen für Lastkraftwagen\\
\item
  Schadensbeurteilung durch Sichtprüfung\\
\item
  Bauweise von Karosserien\\
\item
  Karosseriebauformen\\
\item
  Selbstragende Fahrzeugaufbauten\\
\item
  Stechmaß/Stechzirkel bei der in Karosserievermessung\\
\item
  Verformte Rahmenteile richten\\
\item
  Schadensbeurteilung - Karosserie\\
\item
  Karosserieseitenteil\\
\item
  Richtbank mit mechanischem Messsystem\\
\item
  Karosseriemesssysteme\\
\item
  Karosserievermessung\\
\item
  Richten einer Karosserie 4
\item
  Arbeiten an der Karosserie\\
\item
  Fügen bei der Karosseriereparatur\\
\item
  Spachteln bei der Karosseriereparatur\\
\item
  Sicherheitskarosserie\\
\item
  Pralldämpfer
\end{enumerate}

\subsection{Korrosionsschutz an
Kraftfahrzeugen}\label{korrosionsschutz-an-kraftfahrzeugen}

\begin{enumerate}
\item
  Hohlraumversiegelung\\
\item
  Passiver Korrosionsschutz\\
\item
  Konservierungsverfahren\\
\item
  Korrosionsarten
\end{enumerate}

\subsection{Fahrzeuglackierung}\label{fahrzeuglackierung}

\begin{enumerate}
\item
  Aufgabe der Grundierung\\
\item
  Aufgabe des Füllers\\
\item
  Aufbau einer Fahrzeuglackierung\\
\item
  Reparaturlackierung eines Fahrzeugs\\
\item
  Lackierungen\\
\item
  Kataphorese\\
\item
  Lackierungen - Anforderungen
\end{enumerate}

\section{Komfort- und
Sicherheitssysteme}\label{komfort--und-sicherheitssysteme}

\subsection{Fahrzeugsicherheit}\label{fahrzeugsicherheit}

\begin{enumerate}
\item
  Passive Sicherheit\\
\item
  Frontalaufprall\\
\item
  Aktive Sicherheit\\
\item
  Verbundglas\\
\item
  Windschutzscheibe\\
\item
  Einklemmschutz bei elektrischen Fensterhebern\\
\item
  Belegungserkennung bei Sitzen\\
\item
  Sicherheitsgurte\\
\item
  Gurtstraffer\\
\item
  Aufgaben von Knautschzonen\\
\item
  Aufgabe von Frontairbags\\
\item
  Frontairbag\\
\item
  Fahrzeugrückhaltesystem\\
\item
  Fahrzeugrückhaltesystem - Gurtstraffersystem\\
\item
  Fahrzeugrückhaltesystem -- Wickelfeder\\
\item
  Fahrerairbag\\
\item
  Arbeiten an Fahrzeugrückhaltesystemen\\
\item
  Airbagkontrollleuchte\\
\item
  Diagnose - Fahrzeugrückhaltesystem\\
\item
  Steckverbindungen -- Airbag/Gurtstraffer\\
\item
  Arbeiten an Fahrzeugrückhaltesystemen\\
\item
  Gurtstraffersysteme\\
\item
  Airbaggenerator\\
\item
  Sicherheitslenksäule\\
\item
  Rückhaltesystem\\
\item
  Arbeiten an Airbag und Gurtstraffer\\
\item
  Aktive Sicherheit\\
\item
  Reversibler Gurtstraffer\\
\item
  Fußgängerairbag\\
\item
  Rohrgasgenerator\\
\item
  Endbeschlaggurtstraffer/Beckengurtstraffer\\
\item
  Interaktionsairbag\\
\item
  Pyrotechnische Batteriesicherheitsklemme\\
\item
  Post-Crash-Maßnahmen\\
\item
  Rettungskarte\\
\item
  Abschaltung Beifahrerairbag
\end{enumerate}

\subsection{Fahrerassistenzsysteme}\label{fahrerassistenzsysteme}

\begin{enumerate}
\item
  Adaptive Geschwindigkeitsregelung\\
\item
  Betriebs- und Fahrdatenanzeige\\
\item
  Erkennung des vorausfahrenden Fahrzeugs (Systeme)\\
\item
  Erkennung des vorausfahrenden Fahrzeugs (Sensoren)\\
\item
  Fahrerassistenzsysteme\\
\item
  Einparkhilfe\\
\item
  Einparkassistent\\
\item
  Vollautomatischer Einparkassistent\\
\item
  Anhängerassistent (Knickwinkel)\\
\item
  ACC mit Stop\&Go-Funktion\\
\item
  Spurhalteassistent\\
\item
  Aktiver Spurhalteassistent\\
\item
  Spurwechselassistent\\
\item
  Nachtsichtsystem\\
\item
  Head-Up-Display\\
\item
  Verkehrszeichenerkennung\\
\item
  Kalibrierung des Radarsensors\\
\item
  Kalibrierung\\
\item
  Statische Kalibrierung einer Frontkamera\\
\item
  Dynamische Kalibrierung einer Frontkamera\\
\item
  Kalibrier-Matten
\end{enumerate}

\subsection{Infotainmentsysteme}\label{infotainmentsysteme}

\begin{enumerate}
\item
  Infotainment\\
\item
  Infotainmentsysteme\\
\item
  Beispiele für Infotainmentsysteme\\
\item
  Automatisches Notrufsystem\\
\item
  eCall (Datentransport)\\
\item
  Automatisches Notrufsystem „eCall>> -- Wirkungsweise\\
\item
  eCall-Daten
\end{enumerate}

\subsection{Komfortsysteme}\label{komfortsysteme}

\begin{enumerate}
\item
  Navigationssysteme\\
\item
  GPS\\
\item
  Systembenennung\\
\item
  Navigationssystem: Systemkomponenten\\
\item
  Positionsgenauigkeit\\
\item
  Navigationssystem (Ortung)\\
\item
  Navigationssystem / GPS\\
\item
  Bauarten von Navigationssystemen\\
\item
  Dynamische Zielführung\\
\item
  Navigationssysteme / POI\\
\item
  Koppelortung
\end{enumerate}

\subsection{Belüftung, Heizung,
Klimatisierung}\label{belueftung-heizung-klimatisierung}

\begin{enumerate}
\item
  Heizen des Fahrzeuginnenraums\\
\item
  Aufgabe des Kondensators im Kältemittelkreislauf\\
\item
  Aufgabe des Expansionsventils\\
\item
  Drossel in Klimaanlage\\
\item
  Wagenheizung\\
\item
  Innenraumheizung\\
\item
  Heizleistung\\
\item
  Aufgaben der Klimaanlage\\
\item
  Bauteile der Klimaanlage\\
\item
  Klimaanlage\\
\item
  Aufgabe des Verdampfers der Klimaanlage\\
\item
  Aufgabe des Kompressors der Klimaanlage\\
\item
  Kältemitteldampf im Kondensator\\
\item
  Aufgabe des Expansionsventils in der Klimaanlage\\
\item
  Druck und Aggregatzustand des Kältemittels\\
\item
  Arbeiten an der Klimaanlage
\end{enumerate}

\subsection{Diebstahlschutzsysteme}\label{diebstahlschutzsysteme}

\begin{enumerate}
\item
  Komponenten Diebstahlschutzsystem\\
\item
  Bedienung der Zentralverriegelungsanlage\\
\item
  Wegfahrsperre\\
\item
  Passiver Zugang\\
\item
  Diebstahlschutzsysteme\\
\item
  Aufbau Wegfahrsperre
\end{enumerate}

\section{Elektrische Systeme}\label{elektrische-systeme}

\subsection{Beleuchtungsanlage, Scheinwerfer,
Lichttechnik}\label{beleuchtungsanlage-scheinwerfer-lichttechnik}

\begin{enumerate}
\item
  Lichttechnische Einrichtungen\\
\item
  Lampenarten 4
\item
  Scheinwerferlampen\\
\item
  Halogenlampen\\
\item
  Gasentladungslampen\\
\item
  Kurvenlicht\\
\item
  Scheinwerfereinstellung - Prüfbilder\\
\item
  Scheinwerfereinstellgerät\\
\item
  Reflektorarten\\
\item
  H4 Halogenlampen\\
\item
  Paraboloidförmiger Reflektor\\
\item
  Halogenlampen vs.~Glühlampen\\
\item
  Gasentladungslampen\\
\item
  Leuchtweitenregelung\\
\item
  Klemmenbezeichnungen Lampen\\
\item
  Scheinwerfersystem\\
\item
  Kurvenlicht\\
\item
  Reflektor\\
\item
  Lichtfunktionen LED\\
\item
  Definition Scheinwerfer und Leuchten\\
\item
  Lampenarten\\
\item
  Halogenlampen\\
\item
  Bi-Xenon Scheinwerfer\\
\item
  LED-Scheinwerfer\\
\item
  Laser-Scheinwerfer\\
\item
  Neigungsmaß - Scheinwerfereinstellung\\
\item
  Scheinwerfereinstellung\\
\item
  Blendwert Abblendlicht\\
\item
  Fehlersuche Kennzeichenleuchte\\
\item
  Fehlersuche Blinkanlage\\
\item
  Arbeitsschritte Scheinwerfereinstellgerät\\
\item
  Scheinwerfereinstellung - Prüfvoraussetzungen\\
\item
  Lichttechnik LKW Vorderseite\\
\item
  Lichttechnik LKW Rückseite\\
\item
  Lichttechnik PKW Vorderseite\\
\item
  Lichttechnik PKW Rückseite
\end{enumerate}

\subsection{Elektrische Motoren,
Starter}\label{elektrische-motoren-starter}

\begin{enumerate}
\item
  Baugruppen des Starters\\
\item
  Gleichstrommotoren\\
\item
  Aufbau und Funktion des Schub-Schraubtrieb-Starters\\
\item
  Bauarten von Elektromotoren\\
\item
  Startdrehzahlen von Verbrennungsmotoren\\
\item
  Einrückrelais Starter\\
\item
  Starterbauart\\
\item
  Starterbauteile\\
\item
  Freilaufsystem Starter\\
\item
  Aufgabe des Einrückrelais\\
\item
  Startvorgang\\
\item
  Klemmenbezeichnungen am Starter\\
\item
  Reihenschlussmotor\\
\item
  Starter für PKW/LKW\\
\item
  Kommutator\\
\item
  Freilauf Bauteile\\
\item
  Bauteile eines Starters\\
\item
  Messungen am Starter\\
\item
  Fehlersuche am Starter (1)\\
\item
  Widerstand Starterhauptleitung\\
\item
  Strom durch Starterhauptleitung\\
\item
  Starter Temperaturprobleme
\end{enumerate}

\subsection{Sensoren}\label{sensoren}

\begin{enumerate}
\item
  Aufgabe von Sensoren\\
\item
  Funktion Reed-Kontakt\\
\item
  Aufbau Drosselklappenpotentiometer\\
\item
  Funktion induktiver Drehbewegungssensor\\
\item
  Aufbau induktiver Drehzahlgeber\\
\item
  Diagnose Hall-Bezugsmarkengeber\\
\item
  Aufbau Hall-Drehzahlgeber\\
\item
  Hall-Effekt\\
\item
  Aufbau Aktiver Raddrehzahlgeber\\
\item
  Funktion Hall-Winkelsensor\\
\item
  Funktion Klopfsensor\\
\item
  Temperatursensor\\
\item
  Drucksensor\\
\item
  Luftmassenmesser\\
\item
  Aufbau einer Lambdasonde\\
\item
  Schaltzeichen von Sensoren\\
\item
  Aufbau induktiver Drehbewegungssensor\\
\item
  Aufbau Klopfsensor\\
\item
  Kennlinien von Sensoren\\
\item
  Signalbild Hall-Bezugsmarkengeber\\
\item
  Aktive und passive Sensoren\\
\item
  Potentiometer\\
\item
  Induktive Drehzahlgeber\\
\item
  Temperaturmessung\\
\item
  Klopfsensor\\
\item
  Austausch eines Klopfsensors\\
\item
  Signalbilder von Sensoren
\end{enumerate}

\subsection{Hochvolttechnik}\label{hochvolttechnik}

\begin{enumerate}
\item
  Spannungshöhe Hochvolt\\
\item
  Bauteile Hochvolt-System\\
\item
  Arten elektrischer Maschinen\\
\item
  Elektrische Maschinen\\
\item
  Elektrische Maschine - Bauteile\\
\item
  Funktionsbeschreibung Drehstromasynchronmotor\\
\item
  Drehzahl Drehstromsynchronmotor\\
\item
  Vorteile und Nachteile von Drehstrommotoren\\
\item
  Arten von HV-Batterien\\
\item
  Merkmale von HV-Batterien\\
\item
  Sicherheitsschaltung in HV-Speichern\\
\item
  Lithium-Ionen-Batterie\\
\item
  Aufgabe Leistungselektronik\\
\item
  Leistungselektronik\\
\item
  DC/DC Wandler\\
\item
  Isolationswiderstand\\
\item
  Potentialausgleich\\
\item
  Isolationswiderstand\\
\item
  Netzsystem Hochvolt\\
\item
  Eigensicheres Fahrzeug\\
\item
  Isolationsfehler\\
\item
  Teil- und Vollelektrische Antriebe\\
\item
  Hybridkonzepte\\
\item
  Extended Range Electric Vehicle (EREV)\\
\item
  Stromfluss in elektrischen Maschinen\\
\item
  Energiefluss Leistungselektronik\\
\item
  Sicherheitslinie\\
\item
  Wartungsstecker\\
\item
  Klemme 30c\\
\item
  Hochvolt-Leitungen\\
\item
  Persönliche Schutzausrüstung (PSA)\\
\item
  Komponenten eines vollelektrischen Antriebs\\
\item
  Sicherheitsregeln\\
\item
  Ladezeit\\
\item
  Drehzahl Synchronmotor\\
\item
  Polpaarzahl Synchronmotor\\
\item
  Schlupf Asynchronmaschine\\
\item
  Pilotlinie\\
\item
  HV-Komponenten\\
\item
  Verschaltung der Batteriezellen e-tron\\
\item
  Verschaltung der Batteriezellen e-Golf
\end{enumerate}

\section{Messen, Testen, Diagnose}\label{messen-testen-diagnose}

\begin{enumerate}
\item
  Sporadische Fehlern\\
\item
  Einstellung am Multimeter\\
\item
  Multimeter\\
\item
  Strommessung mit Multimeter\\
\item
  Messung mit dem Oszilloskop\\
\item
  Prüfungen mit Multimeter\\
\item
  Geführte Fehlersuche\\
\item
  Systematische Fehlersuche\\
\item
  Oszilloskop\\
\item
  Spannungsverlauf\\
\item
  Spannung und Periodendauer\\
\item
  Spannung mit Multimeter bestimmen\\
\item
  Messung zur Fehlerbestimmung\\
\item
  Messungen\\
\item
  Schaltung\\
\item
  Widerstand\\
\item
  Messen mit dem Oszilloskop\\
\item
  Signal\\
\item
  Ein- und Ausschaltdauer\\
\item
  Einstellung des Multimeters\\
\item
  Oszilloskop oder Multimeter\\
\item
  Ein- und Ausschaltdauern\\
\item
  Schaltung\\
\item
  Erfassung von Messwerten\\
\item
  Erfassung von Messwerten\\
\item
  Signalbilder im Oszilloskop
\end{enumerate}

\section{Netzwerktechnik}\label{netzwerktechnik}

\subsection{Grundlagen
Informationstechnik}\label{grundlagen-informationstechnik}

\begin{enumerate}
\item
  Oszillogramm eines fehlerfreien CAN Class B Signals\\
\item
  Oszillogramm CAN Class B Signal\\
\item
  Fehler eines CAN-Bus Class B\\
\item
  Fehler im CAN Class B\\
\item
  CAN\\
\item
  Signalart\\
\item
  Digitale Signale\\
\item
  Pegel LIN-Bussystem
\end{enumerate}

\subsection{Datenübertragung im
Kraftfahrzeug}\label{datenuebertragung-im-kraftfahrzeug}

\begin{enumerate}
\item
  Asynchrone Datenübertragung\\
\item
  Fehlerursachen bei der Datenübertragung mit Lichtwellenleitern\\
\item
  Datenübertragungsgeschwindigkeit\\
\item
  Datenübertragungsart\\
\item
  Datenübertragungsgeschwindigkeit\\
\item
  Aufgabe der transparenten Beschichtung des Kerns im
  Lichtwellenleiter\\
\item
  Lichtwellenleiter 2
\item
  Datenübertragungssystem\\
\item
  Aufgabe des Protokolls in der Datenübertragung
\end{enumerate}

\subsection{Datenbussysteme, elektrische,
optische}\label{datenbussysteme-elektrische-optische}

\begin{enumerate}
\item
  Vorteile der Datenbusübertragung\\
\item
  Datenbussysteme\\
\item
  Informationsübertragung\\
\item
  Bestandteile von Knoten in Datenbussystemen\\
\item
  Übertragungsprinzip von LIN-Bussystemen\\
\item
  Aufgaben des Master-Steuergeräts im LIN-Bussystem\\
\item
  Dominanter Spannungspegel LIN-Bus\\
\item
  CAN-high-Leitung\\
\item
  Sleep-Modus\\
\item
  Merkmale von optischen Datenbussystemen\\
\item
  Nachteil Ringstruktur Datenbussystem\\
\item
  Aufgabe Ringbruchdiagnose MOST-Datenbussystem\\
\item
  Eindrahtfähigkeit\\
\item
  Datenbussystem\\
\item
  Funktionsprinzip der LIN-Datenbussysteme\\
\item
  Unterbrechung des Lichtwellenleiters\\
\item
  Fehlerursache eines Ringbruchs\\
\item
  Vorteil durch Verwendung von CAN-Datenbus-Systemen\\
\item
  CAN-Bus\\
\item
  Vorteile der Systemkopplung\\
\item
  Aufgabe des Statusfelds\\
\item
  Aufgabe des Datenfelds
\end{enumerate}

\subsection{Hochfrequenztechnik}\label{hochfrequenztechnik}

\begin{enumerate}
\item
  Sende-/Empfangsanlage\\
\item
  Trägerfrequenz\\
\item
  Empfangsantennen\\
\item
  Funkschatten\\
\item
  Informationsübertragung mit Funkwellen\\
\item
  Hochfrequenz-Sendeanlage\\
\item
  Modulationsart\\
\item
  Amplitudenmodulation\\
\item
  Funktion einer Empfangsstabantenne\\
\item
  Länge einer Empfangsstabantenne\\
\item
  Wellenwiderstand\\
\item
  Mehrwegeempfang\\
\item
  Messung zur Prüfung einer Rundfunk-Empfangsanlage\\
\item
  Antennenleitung\\
\item
  Antenne\\
\item
  Elektromagnetische Verträglichkeit
\end{enumerate}

\section{Nutzfahrzeugtechnik}\label{nutzfahrzeugtechnik}

\subsection{Nutzfahrzeugtechnik}\label{nutzfahrzeugtechnik-1}

\begin{enumerate}
\item
  Stickoxidemission senken bei Dieselmotoren\\
\item
  SCR-Verfahren\\
\item
  Felgenbezeichnung bei Nutzfahrzeugen\\
\item
  Gruppengetriebe\\
\item
  PLD\\
\item
  Bauteile einer NFZ-Einspritzanlage\\
\item
  Einspritzanlage\\
\item
  Pumpe-Leitungs-Düse-System\\
\item
  SCR-Anlage\\
\item
  Starthilfsanlage\\
\item
  Vorteil der Luftfederung\\
\item
  Luftgefederte Achse\\
\item
  Einteilige Felge\\
\item
  Lkw-Radialreifen\\
\item
  Gelenkte Vorderachsen\\
\item
  Zwei-Wellen-Vorgelege-Getriebe\\
\item
  Blattfederung Nfz\\
\item
  Luftfederung Nfz\\
\item
  Reifenbezeichnung Nfz\\
\item
  Felgenarten Nfz\\
\item
  Antriebsachse mit Hypoidantrieb\\
\item
  Common-Rail-System Nfz\\
\item
  Verteilergetriebe Nfz\\
\item
  Common-Rail-Anlage NkW mit X-Pulse
\end{enumerate}

\subsection{Nutzfahrzeugbremsen}\label{nutzfahrzeugbremsen}

\begin{enumerate}
\item
  Aufgaben des Druckreglers\\
\item
  Aufgaben des Vierkreisschutzventils\\
\item
  Aufgaben des Betriebsbremsventils\\
\item
  Aufgaben des Feststell- und Hilfsbremsventils\\
\item
  Bauarten von Dauerbremsen\\
\item
  ASR-Anlage für Druckluftbremsanlagen\\
\item
  Elektronisches Bremssystem EBS\\
\item
  Dauerbremse\\
\item
  Druckluftversorgungsanlage\\
\item
  Höchstdruck in der Druckluftbremsanlage\\
\item
  Betriebsbremsanlage\\
\item
  Betriebsbremsventil\\
\item
  Feststellbremsventil\\
\item
  Kombibremszylinder\\
\item
  Drücke beim Kombibremszylinder\\
\item
  Dauerbremse\\
\item
  Vorteile von Dauerbremsen\\
\item
  Elektronisches Bremssystem\\
\item
  Bremszylinder Nfz\\
\item
  Dauerbremsanlage\\
\item
  Pneumatisch betätigte Scheibenbremse\\
\item
  Lufttrockner\\
\item
  Bremswertgeber der EBS-Anlage\\
\item
  ABS-Magnetventil\\
\item
  Funktion ABS-Magnetventil\\
\item
  Achsmodulator EBS-Anlage\\
\item
  Merkmale EBS-Anlage
\end{enumerate}
