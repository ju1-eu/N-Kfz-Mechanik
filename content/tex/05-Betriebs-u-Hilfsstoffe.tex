%ju 28-Mai-22 05-Betriebs-u-Hilfsstoffe.tex
\section{Was sind Betriebsstoffe?}\label{was-sind-betriebsstoffe}

Sind Stoffe, die zum Betrieb des Kraftfahrzeuges nötig sind.

\textbf{Beispiele} Kraftstoffe, Schmieröle, Bremsflüssigkeit

\section{Was sind Hilfsstoffe?}\label{was-sind-hilfsstoffe}

Sind alle Stoffe, die zum Reinigen und Pflegen von Fahrzeugen notwendig
sind.

\section{Scheibenwaschwasserzusatz}\label{scheibenwaschwasserzusatz}

\begin{itemize}
\item
  \textbf{Sommer} mit Enzymen -- Insektenreste besser entfernen
\item
  \textbf{Winter} mit Gefrierschutz
\end{itemize}

\section{Woraus bestehen
Kraftstoffe?}\label{woraus-bestehen-kraftstoffe}

Aus einem Gemisch unterschiedlicher Kohlenwasserstoffverbindungen.

Bei ihrer Verbrennung werden Wasserstoff- und Kohlenstoffatome mit
Sauerstoff zu $H_2O \text{ und } CO_2$ oxidiert. Nur ein Teil der frei
werdenden Energie treibt den Motor an.

\textbf{Wirkungsgrad eines Verbrennungsmotors}

\begin{enumerate}
\item
  Dieselmotoren max. ca. 46 \% und
\item
  Ottomotoren max. ca. 35 \% Bewegungsenergie als Antriebsenergie für
  Motor
\item
  Rest in Reibung und Wärme
\end{enumerate}

\begin{itemize}
\item
  \textbf{Oktanzahl} Klopffestigkeit des Kraftstoffes

  \begin{itemize}
  \item
    Je klopffester der Kraftstoff ist, umso höher kann er eine
    thermische Belastung aushalten, ohne sich selbst zu entzünden.
  \end{itemize}
\item
  \textbf{Cetanzahl} Zündwilligkeit eines Dieselkraftstoffes
\item
  \textbf{Zündverzug} (eines intakten Motors ohne Verbrennungsstörung)
  $\frac{1}{1000}~s$
\end{itemize}

\textbf{Aufbau der Kohlenwasserstoffmoleküle}

\begin{enumerate}
\item
  \textbf{einfache Kettenform} zündwillig und verbrennen leicht (nicht
  klopffest)
\item
  \textbf{verzweigte Kettenform} (Isomere) zündunwillig (klopffest)
\end{enumerate}

\begin{itemize}
\item
  \textbf{Paraffine} kettenförmiger Aufbau, wenig klopffest,

  \begin{itemize}
  \item
    gasförmig -- bei niedrigem Druck verflüssigtes Treibgas, z. B.
    Propan
  \item
    flüssig -- bestandteile des Benzins und Dieselkraftstoffes, z. B.
    Oktan, Cetan
  \end{itemize}
\item
  \textbf{Isoparaffine} kettenförmiger Aufbau mit Seitenketten, sehr
  klopffest

  \begin{itemize}
  \item
    Bestandteil des Eichkraftstoffes für Ottokraftstoffe, z. B. Isooktan
  \end{itemize}
\item
  \textbf{Aromaten} ringförmiger Aufbau, sehr klopffest
\end{itemize}

\section{Wo kommen die Kraftstoffe
her?}\label{wo-kommen-die-kraftstoffe-her}

\begin{enumerate}
\item
  \textbf{Erdöl}
\item
  \textbf{E-Fuels} Kraftstoffe aus dem $CO_2$ der Luft, klimaneutral
  \footnote{\url{https://www.youtube.com/watch?v=qq0fjl0LQXo}}

  \begin{itemize}
  \item
    Stromerzeuger: Windrad oder Solarenergie
  \item
    Offshore-Windparks sind Windparks, die im Küstenvorfeld der Meere
    errichtet werden.

    \begin{itemize}
    \item
      haben keine Speicher, Wechselspannung kann nicht gespeichert
      werden
    \end{itemize}
  \end{itemize}
\end{enumerate}

\section{Warum ein Dieselmotor effizienter ist als ein
Ottomotor?}\label{warum-ein-dieselmotor-effizienter-ist-als-ein-ottomotor}

\begin{itemize}
\item
  Energiedichte höher
\item
  Wirkungsgrad höher gegenüber Ottomotor
\item
  Wärmeabführung geringer
\end{itemize}

\section{Herstellung von
Kraftstoffen}\label{herstellung-von-kraftstoffen}

\textbf{Trennverfahren}

\begin{enumerate}
\item
  \textbf{Filtern} Verunreinigungen werden aus dem Rohöl entfernt
\item
  \textbf{Destillieren} Trennen
\item
  \textbf{Raffinieren} nach behandeln, Reinigen
\end{enumerate}

\textbf{Destillieren von Rohöl}

\textbf{fraktionierenden Destillation} Sammeln der Kraftstoffe nach
ihren Siedebereichen - Trennung der im Erdöl enthaltenen Stoffgruppen
nach ihren Siedebereichen durch Erhitzen des Erdöls unter Luftabschluss
bis auf etwa 360~°C. Bei der anschließenden Abkühlung kondensieren die
verschiedenen Bestandteile bei unterschiedlichen Temperaturen.

\begin{enumerate}
\item
  \textbf{atmosphärische Destillation} Druck bei 1013 mbar
\item
  \textbf{Vakuum Destillation} Unterdruck -- Siedepunkt herabsetzen
\end{enumerate}

\textbf{Umwandlungsverfahren}

\begin{enumerate}
\item
  \textbf{Cracken} Umwandeln - Abbau von Großmolekülen der höher
  siedenden Schwerkraftstoffe durch Zerlegen in leichtere und
  klopffester Isoparaffine
\item
  \textbf{Reformieren} Kettenförmige Paraffine aus der Destillation
  werden mit Katalysatoren (Platin) in klopffeste Isoparaffine und
  Aromate umgewandelt
\item
  \textbf{Polymerisieren} die beim Cracken und Reformieren entstandenen
  gasförmigen Kohlenwasserstoffe werden über Katalysatoren zu größeren
  Molekülen zusammengeballt, hauptsächlich zu Isoparaffinen
\end{enumerate}

\textbf{Katalysator} altern vs.~\textbf{Verschleißteil} z. B. Bremsbelag
(Reibung)

\textbf{Schwefel} giftig, Kraftstoff entschwefeln, schmierende Wirkung,
Ersatzstoff gesucht und \textbf{Harz} Einspritzung verharzen, betrifft
Oldtimer

\section{Ottokraftstoffe -- leicht siedende
Kraftstoffe}\label{ottokraftstoffe-leicht-siedende-kraftstoffe}

\textbf{Eigenschaften}

\begin{itemize}
\item
  leicht und vollständig vergasen
\item
  klopffest sein
\item
  rückstandsfrei verbrennen
\item
  keine Verunreinigung enthalten
\item
  hohen Heizwert
\end{itemize}

\begin{enumerate}
\item
  \textbf{Flammpunkt} bedarf eine fremde Zündquelle (Ottokraftstoff
  unter 21~°C) vs.~
\item
  \textbf{Selbstentzündungstemperatur} entzündet sich selbst
  (Dieselkraftstoff)
\item
  \textbf{Siedeverlauf} beim Ottomotor muss der Kraftstoff leicht und
  vollständig vergasen, da nur gasförmiger Kraftstoff verbrannt werden
  kann.
\item
  \textbf{Siedebereich} Verdampfen zwischen 25~°C und 215~°C
\item
  \textbf{Kaltstartverhalten} Kraftstoff mit niedriger Siedekurve
\item
  \textbf{Heisstartverhalten} Dampfblasenbildung im Kraftstoffsystem (zu
  viel Luft) z. B. K-Jetronic
\item
  \textbf{Klopffestigkeit} geringe Neigung eines Kraftstoffes, sich
  unter hohen Temperaturen und Drücken selbst zu entzünden

  \begin{itemize}
  \item
    Maß für die Klopffestigkeit

    \begin{itemize}
    \item
      ROZ (Research-Oktanzahl) Superbenzin 95 ROZ
    \item
      MOZ (Motor-Oktanzahl)
    \end{itemize}
  \item
    Was gibt die Oktanzahl an?

    \begin{itemize}
    \item
      wie viel Vol.-\% Iso-Oktan sich in einem Bezugskraftstoffgemisch
      befinden
    \end{itemize}
  \item
    Oktanzahl bestimmen

    \begin{itemize}
    \item
      durch Vergleich des prüfenden Kraftstoffes mit einem
      Bezugskraftstoff (gleiches Klopfverhalten) -- einer Mischung aus
      Normal-Heptan (OZ=0) klopffreudig und Iso-Oktan (OZ=100) klopffest
      in einem Prüfmotor
    \end{itemize}
  \end{itemize}
\item
  \textbf{Arten von Klopfbremsen} (Maßnahmen, um die Klopffestigkeit zu
  erhöhen)

  \begin{itemize}
  \item
    Zusatz von Klopfbremsen wie MTP
  \item
    Zusatz von metallfreien Klopfbremsen wie Benzol, begrenzt auf 1
    Vol.-\%
  \item
    Zusatz von organischen Sauerstoff-Verbindungen wie Alkohole
  \end{itemize}
\end{enumerate}
