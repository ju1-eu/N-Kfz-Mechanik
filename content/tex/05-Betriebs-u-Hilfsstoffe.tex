%ju 05-Jun-22 05-Betriebs-u-Hilfsstoffe.tex
\section{Was sind Betriebsstoffe?}\label{was-sind-betriebsstoffe}

Sind Stoffe, die zum Betrieb des Kraftfahrzeuges nötig sind.

\textbf{Beispiele:} Kraftstoffe, Motoröl, Bremsflüssigkeit

\section{Was sind Hilfsstoffe?}\label{was-sind-hilfsstoffe}

Sind alle Stoffe, die zum Warten, Reinigen und Pflegen von Fahrzeugen
notwendig sind.

\textbf{Beispiele:} Politur, Bremsenreiniger, Scheibenreiniger

\section{Scheibenwaschwasserzusatz}\label{scheibenwaschwasserzusatz}

\begin{itemize}
\item
  \textbf{Sommer} mit Enzymen -- Insektenreste besser entfernen
\item
  \textbf{Winter} mit Gefrierschutz
\end{itemize}

\section{Woraus bestehen
Kraftstoffe?}\label{woraus-bestehen-kraftstoffe}

Aus einem Gemisch unterschiedlicher Kohlenwasserstoffverbindungen.

Bei ihrer Verbrennung werden Wasserstoff- und Kohlenstoffatome mit
Sauerstoff zu $H_{2}O \text{ und } CO_2$ oxidiert. Nur ein Teil der
frei werdenden Energie treibt den Motor an.

\textbf{Wirkungsgrad eines Verbrennungsmotors}

\begin{enumerate}
\item
  Dieselmotoren ca. $46~\%$ und Ottomotoren ca. $35~\% \to$
  Bewegungsenergie als Antriebsenergie für Motor
\item
  Rest in Reibung und Wärme
\end{enumerate}

\begin{itemize}
\item
  \textbf{Oktanzahl} Klopffestigkeit des Kraftstoffes

  \begin{itemize}
  \item
    >>Je klopffester der Kraftstoff ist, umso höher kann er eine
    thermische Belastung aushalten, ohne sich selbst zu entzünden.<<
  \end{itemize}
\item
  \textbf{Cetanzahl} Zündwilligkeit von Dieselkraftstoff (Wie stark ein
  Kraftstoff zur Selbstzündung neigt)
\item
  \textbf{Zündverzug} $\frac{1}{1000}~s$ (eines intakten Motors ohne
  Verbrennungsstörung)
\end{itemize}

\textbf{Aufbau der Kohlenwasserstoffmoleküle}

\begin{enumerate}
\item
  \textbf{einfache Kettenform,} zündwillig und verbrennen leicht (nicht
  klopffest)
\item
  \textbf{verzweigte Kettenform} (Isomere) zündunwillig (klopffest)
\end{enumerate}

\begin{itemize}
\item
  \textbf{Paraffine} kettenförmiger Aufbau, wenig klopffest,

  \begin{itemize}
  \item
    gasförmig -- bei niedrigem Druck verflüssigtes Treibgas, Beispiel:
    Propan
  \item
    flüssig -- bestandteile des Benzins und Dieselkraftstoffes,
    Beispiel: Oktan, Cetan
  \end{itemize}
\item
  \textbf{Isoparaffine} kettenförmiger Aufbau, mit Seitenketten,
  verzweigt, sehr klopffest

  \begin{itemize}
  \item
    Bestandteil des Eichkraftstoffes für Ottokraftstoffe, Beispiel:
    Isooktan
  \end{itemize}
\item
  \textbf{Aromaten} ringförmiger Aufbau, sehr klopffest, Beispiel:
  Benzol
\end{itemize}

\section{Wo kommen die Kraftstoffe
her?}\label{wo-kommen-die-kraftstoffe-her}

\begin{enumerate}
\item
  \textbf{Erdöl} aus ca. 80 \% Kohlenstoff und 12 \% Wasserstoff, ca.
  1--3 \% Schwefel
\item
  \textbf{E-Fuels} Kraftstoffe aus dem $CO_2$ der Luft, klimaneutral
  \footnote{\url{https://www.youtube.com/watch?v=qq0fjl0LQXo}}

  \begin{itemize}
  \item
    Stromerzeuger: Windrad oder Solarenergie
  \item
    Offshore-Windparks sind Windparks, die im Küstenvorfeld der Meere
    errichtet werden.

    \begin{itemize}
    \item
      haben keine Speicher, Wechselspannung kann nicht gespeichert
      werden
    \end{itemize}
  \end{itemize}
\end{enumerate}

\section{Warum ist ein Dieselmotor effizienter xals ein
Ottomotor?}\label{warum-ist-ein-dieselmotor-effizienter-xals-ein-ottomotor}

\begin{itemize}
\item
  Energiedichte höher
\item
  Wirkungsgrad höher gegenüber Ottomotor
\item
  Wärmeabführung geringer
\end{itemize}

\section{Herstellung von
Kraftstoffen}\label{herstellung-von-kraftstoffen}

\begin{enumerate}
\item
  Erdöl
\item
  Destillation (Erhitzen, Verdampfen und Kondensieren)
\item
  Vakuumdestillation
\item
  Reformieren
\item
  Raffinieren
\item
  Endparaffinierung\\
\item
  Cracken
\item
  Raffinat / Grundöl
\end{enumerate}

\textbf{Trennverfahren}

\begin{enumerate}
\item
  \textbf{Filtern} Verunreinigungen werden aus dem Rohöl entfernt
\item
  \textbf{Destillieren} Trennen
\item
  \textbf{Raffinieren} Nachbehandeln, Reinigen
\end{enumerate}

\textbf{Destillieren von Rohöl}

\textbf{Fraktionierende Destillation} Sammeln der Kraftstoffe nach ihren
Siedebereichen

Aufteilen von Rohöl nach Siedebereichen. Das Rohöl wird in einem
Röhrenofen auf ca. $360^\circ\text{C}$ erhitzt und anschließend in
einem Turm mit mehreren Ebenen geleitet. Die Kraftstoffdämpfe steigen
nach oben und kondensieren dabei nach und nach (Temperaturabnahme).
Zunächst wird Diesel von Petroleum, Schwer- und Leichtbenzin getrennt.
Propan und Butan wird zu LPG (verflüssigtes Petroleum Gas)
weiterverarbeitet.

Die Rohölbestandteile, die den Ofen flüssig verlassen, werden nach
nochmaligen erhitzen in einen weiteren Turm geleitet. Der Druckabfall
senkt den Siedebereich der Flüssigkeiten und es wird nach den gleichen
Verfahren Öle gewonnen, zum Beispiel Motoröl. Der Rest ist Bitumen.

>>Je höher der Druck, desto höher der Siedepunkt.<<

\begin{enumerate}
\item
  \textbf{Atmosphärische Destillation} Druck bei $1013~mbar$
\item
  \textbf{Vakuum Destillation} Unterdruck -- Siedepunkt herabsetzen
\end{enumerate}

\textbf{Umwandlungsverfahren}

\begin{enumerate}
\item
  \textbf{Cracken} Umwandeln

  \begin{itemize}
  \item
    langkettige Kohlenwasserstoffmoleküle (schwer siedend) werden unter
    Wärme und Druck oder mithilfe eines Katalysators in kurzkettige
    Kohlenwasserstoffmoleküle (leicht siedend, Benzin, Gas) zerteilt
  \item
    Verfahren zur Erhöhung der Klopffestigkeit von Ottokraftstoffen
  \end{itemize}
\item
  \textbf{Reformieren} Kettenförmige Paraffine aus der Destillation
  werden mit Katalysatoren (Platin) in klopffeste Isoparaffine und
  Aromate umgewandelt
\item
  \textbf{Polymerisieren}, die beim Cracken und Reformieren entstandenen
  gasförmigen Kohlenwasserstoffe werden über Katalysatoren zu größeren
  Molekülen zusammengeballt, hauptsächlich zu Isoparaffinen
\end{enumerate}

Katalysator \textbf{altern} vs.~Beispiel Bremsbelag
\textbf{verschleißen} (Reibung)

\textbf{Schwefel} giftig, Kraftstoff entschwefeln, schmierende Wirkung,
Ersatzstoff gesucht und \textbf{Harz} Einspritzung verharzen (betrifft
Oldtimer)

\section{Ottokraftstoffe -- leicht siedende
Kraftstoffe}\label{ottokraftstoffe-leicht-siedende-kraftstoffe}

\textbf{Eigenschaften}

\begin{itemize}
\item
  Gefahrenklasse A I, d.h. Flüssigkeiten mit einem Flammpunkt unter
  $21~^\circ\text{C}$
\item
  leicht und vollständig vergasen, leicht siedend
\item
  klopffest sein
\item
  rückstandsfrei verbrennen
\item
  keine Verunreinigung enthalten
\item
  hohen Heizwert
\item
  giftig und umweltgefährlich
\end{itemize}

\begin{enumerate}
\item
  \textbf{Flammpunkt} bedarf eine fremde Zündquelle

  \begin{itemize}
  \item
    vs.~\textbf{Selbstentzündungstemperatur} entzündet sich selbst
    (Dieselkraftstoff)
  \end{itemize}
\item
  \textbf{Siedeverlauf} beim Ottomotor muss der Kraftstoff leicht und
  vollständig vergasen, da nur gasförmiger Kraftstoff verbrannt werden
  kann.
\item
  \textbf{Siedebereich} Verdampfen zwischen
  $25^\circ\text{C} \text{ und } 215^\circ\text{C}$

  \begin{itemize}
  \item
    Siedepunkt: Übergang vom flüssigen in den gasförmigen Zustand
  \end{itemize}
\item
  \textbf{Kaltstartverhalten} Kraftstoff mit niedriger Siedekurve
\item
  \textbf{Heisstartverhalten} Dampfblasenbildung im Kraftstoffsystem (zu
  viel Luft) Beispiel: K-Jetronic
\item
  \textbf{Klopffestigkeit} geringe Neigung eines Kraftstoffes, sich
  unter hohen Temperaturen und Drücken selbst zu entzünden

  \begin{itemize}
  \item
    Maß für die Klopffestigkeit ($\to$ wie stark ein Kraftstoff zur
    Selbstzündung neigt)

    \begin{itemize}
    \item
      ROZ (Research-Oktanzahl)
    \item
      MOZ (Motor-Oktanzahl) $\to$ wird unter anderen Prüfbedingungen
      ermittelt
    \end{itemize}
  \item
    Was gibt die Oktanzahl an?

    \begin{itemize}
    \item
      wie viel Vol.-\% Iso-Oktan sich in einem Bezugskraftstoff befinden
    \end{itemize}
  \item
    Oktanzahl bestimmen

    \begin{itemize}
    \item
      Beispiel: Super (ROZ 95
      $\to 95~\% \text{ Isooktan und Normalheptan } 5~\%$)
    \item
      Wird in einem Prüfmotor mit variablem Verdichtungsverhältnis
      ermittelt, in dem der Kraftstoff mit einem Referenzkraftstoff aus
      Normalheptan (ROZ = 0, klopffreudig) und Isooktan (ROZ = 100,
      klopffest) verglichen wird.
    \end{itemize}
  \end{itemize}
\item
  \textbf{Arten von Klopfbremsen} (Maßnahmen, um die Klopffestigkeit zu
  erhöhen)

  \begin{itemize}
  \item
    Zusatz von Klopfbremsen wie MTP
  \item
    Zusatz von metallfreien Klopfbremsen wie Benzol, begrenzt auf 1
    Vol.-\%
  \item
    Zusatz von organischen Sauerstoff-Verbindungen wie Alkohole
  \end{itemize}
\end{enumerate}

\textbf{Welcher Unterschied besteht zwischen Sommer- und
Winter-Ottokraftstoff?}

\begin{enumerate}
\item
  \textbf{Sommerkraftstoff} neigt, aufgrund des Siedeverlaufs, bei
  höheren Temperaturen weniger zur Dampfblasenbildung und verursacht
  keine Warmstartprobleme.
\item
  \textbf{Winterkraftstoff} muss bei niedrigen Temperaturen eine größere
  Dampfmenge liefern, damit bei Kaltstart ein zündfähiges
  Kraftstoff-Luft-Gemisch zur Verfügung steht.
\end{enumerate}

\section{Bremsflüssigkeit}\label{bremsfluessigkeit}

\textbf{Eigenschaften}

\begin{itemize}
\item
  hygroskopisch
\item
  hoher Siedepunkt bis etwa $300~^\circ\text{C}$
\item
  giftig
\item
  Schmierung der beweglichen Teile (Beispiel Bremszylinder)
\end{itemize}

\section{AdBlue}\label{adblue}

\begin{itemize}
\item
  AdBlue ist eine wässrige Harnstofflösung mit $32,5~\%$ Harnstoff
\item
  reduziert Stickoxide $\text{NO}_\text{x}$ bei Dieselmotoren mit
  einem SCR-Katalysator
\item
  gefriert bei $- 11,5~^\circ\text{C}$
\end{itemize}

\textbf{chemische Prozess} Dosierventil $\to$ Hydrolysestrecke (AdBlue
(Harnstoff) $\to$ Ammoniak (Reduktionsmittel, Gefahrstoff, giftig))
$\to$ SCR-Katalysator $\to$ $NO_\text{x}$ + Ammoniak
($NH_\text{3}$) $\to$ umgewandelt in Stickstoff + Wasser

\begin{enumerate}
\item
  \textbf{Oxidationskatalysator} (Sauerstoff wird frei), Arbeitsbereich
  $400 - 800~^\circ\text{C}$, ca. $350~^\circ\text{C}$
  >>light-off-Point<< ($50~\%$ Umwandlungsrate)

  \begin{itemize}
  \item
    Benzin: $CO + O_2 \to CO_2$, $HC + O_2 \to CO_2 + H_{2}O$,
    $NO_\text{x} \to N_2 + O_2$
  \item
    Diesel: $CO \to CO_2$, $HC \to CO_2 + H_{2}O$
  \item
    (Kohlenmonoxid in Kohlenstoffdioxid, unverbrannte Kohlenwasserstoffe
    in Kohlenstoffdioxid und Wasser, Stickoxide in Stickstoff und
    Sauerstoff)
  \end{itemize}
\item
  \textbf{Dieselpartikelfilter} (DPF-Regeneration) ab ca.
  $600~^\circ\text{C}$

  \begin{itemize}
  \item
    $\text{PM} + O_2 \to CO_{2}$ (Partikel und Sauerstoff in
    Kohlenstoffdioxid)
  \item
    Regeneration (angesammelten Partikel im Partikelfilter verbrennen,
    Staudruck, Differenzdrucksensor)

    \begin{itemize}
    \item
      Nacheinspritzung in Verbrennungsraum: Dabei wird der Kraftstoff
      erst sehr spät in den Brennraum eingespritzt, wodurch die Flamme
      bis in den Ausstoßtakt brennt und die Abgastemperatur im
      Partikelfilter steigt.
    \item
      Auspufföffnungs-Einspritzung über ein EPI-Ventils (Exhaust Port
      Injection) direkt in den Abgaskrümmer
    \end{itemize}
  \end{itemize}
\item
  \textbf{SCR-Katalysator} (selektive katalytische Reduktion) ab ca.
  $170 - 250~^\circ\text{C}$

  \begin{itemize}
  \item
    Stickoxidreduktion: $NO_\text{x} + NH_\text{3} \to N_2 + H_{2}O$
    (Stickoxide und Ammoniak in Stickstoff und Wasser)
  \end{itemize}
\item
  \textbf{NOx-Speicherkatalysator} wird von Schwefelbestandteilen
  zugesetzt und muss regeneriert werden.
\end{enumerate}

Wenn der Schadstoffausstoß steigt, führt das zum Erlöschen der
Betriebserlaubnis (Abgasemissionsklasse nicht mehr gültig, bedeutet
\textbf{Steuerhinterziehung})

\section{Ölverdünnung oder
Ölvermehrung}\label{oelverduennung-oder-oelvermehrung}

\emph{ACHTUNG:} Ein zu hoher Ölstand kann ein Indiz für eine
Ölverdünnung sein durch häufige Kaltstarts.

\begin{itemize}
\item
  schädlich für Motoröl und Katalysator
\item
  beim Diesel: lange Stillstandszeiten (vgl. Biodiesel -- Wasser
  anziehend)
\item
  wenn Dieselkraftstoff in das Motoröl gelangt

  \begin{itemize}
  \item
    Beispiel: DPF-Regeneration $\to$ Nacheinspritzung und den an den
    Kolbenringen abfließenden Kraftstoff kommt es zu Motorproblemen
    durch Ölverdünnung (zu hoher Ölstand, schlechte Ölqualität)
  \end{itemize}
\end{itemize}

\section{Kältemittel}\label{kaeltemittel}

\textbf{Anforderungen}

\begin{enumerate}
\item
  geringes Treibhauspotenzial
\item
  nicht Ozon schädigend
\item
  gering bzw. nicht toxisch (giftig)
\item
  nicht brennbar
\item
  gute thermodynamischen Eigenschaften
\end{enumerate}

\textbf{GWP} (Global Warming Potential, Treibhauspotenzial) gibt den
Treibhauseffekt eines Stoffes im Vergleich zu Kohlendioxid an.

\textbf{R134a} (Tetrafluorethan) hat seinen Siedepunkt bei ca.
$-26~^\circ\text{C}$ bei atmosphärischem Druck. Bei 15 bar Überdruck
liegt der Siedepunkt von R134a bereits bei ca. $55~^\circ\text{C}$.

\begin{itemize}
\item
  GWP-Faktor 1430
\end{itemize}

\textbf{R1234yf} (Tetrafluorpropen) verhält sich ähnlich
(Siedetemperatur bei Atmosphärendruck $-29~^\circ\text{C}$).

\begin{itemize}
\item
  GWP-Faktor 4
\end{itemize}

\textbf{R744} ($CO_2$) sind höhere Drücke in der Klimaanlage
erforderlich.

\begin{itemize}
\item
  GWP-Faktor 1
\end{itemize}

\textbf{R12} enthält Fluorchlorkohlenwasserstoffe (\textbf{FCKW}), die
in der Atmosphäre die Ozonschicht zerstören. Seit 1991 wurde daher R134a
(ohne Chlorverbindungen) verwendet und 2017 meist durch R1234yf
abgelöst.

\textbf{Funktionsweise des Kältemittels} bei der Änderung seines
Aggregatzustands (fest, flüssig, gasförmig) kommt es zur Energieaufnahme
bzw. -abgabe.

Beim Übergang vom flüssigen in den gasförmigen Zustand benötigt das
Kältemittel Energie, die es der Umgebung in Form von Wärme entzieht.
Umgekehrt gibt das Kältemittel beim Übergang vom gasförmigen in den
flüssigen Zustand Wärme ab.

Als Kältemittel muss ein Stoff verwendet werden, der einen möglichst
niedrigen Siedepunkt (flüssig $\to$ gasförmig) hat. Der Siedepunkt
kann durch Einwirken von Druck verschoben werden (vgl. Dampfdruckkurve);
gleichzeitig findet dadurch auch eine Eigenerwärmung statt.

\textbf{Diffusion} beschreibt einen physikalischen Prozess, bei dem sich
2 Stoffe nach und nach durchmischen, bzw. ein Stoff einen anderen
durchdringt. (Beispiel: Bremsflüssigkeit, Kältemittel)

\section{Biodiesel -- Fat-Acid-Methyl-Esther (kurz:
FAME)}\label{biodiesel-fat-acid-methyl-esther-kurz-fame}

entsteht, indem ölhaltige Erzeugnisse, wie Raps mithilfe von Ethanol
oder Methanol nachbehandelt werden. Dieser Vorgang wird als >>umestern<<
bezeichnet. Biodiesel kommt entweder als Reinkraftstoff oder als bis zu
7\%ige Beimischung zum Dieselkraftstoff (sog. Petroldiesel) zum Einsatz.

\textbf{Eigenschaften von Biodiesel und Folgen für den Einsatz im
Verbrennungsmotor:}

\begin{enumerate}
\item
  \textbf{Umweltfreundlich} (hängt stark von der Umsetzung ab, Einsatz
  fossiler Energieträger -- Erdöl -- reduziert und den Ausstoß von
  Treibhausgasen mindert.)
\item
  \textbf{Korrosiv} (kann zur Zersetzung Beispiel: Dichtungen und
  Schläuchen führen)
\item
  \textbf{Reinigend} (kann Filtersysteme oder kraftstoffführende
  Bauteile verstopfen)
\item
  \textbf{Hygroskopisch} (zieht aufgrund seines Alkoholanteils Wasser
  an) \textbf{erhöhter Wasseranteil kann zu folgenden Erscheinungen
  führen}

  \begin{itemize}
  \item
    Heraufsetzung des Cold Filter Plugging Point (\textbf{CFPP})

    \begin{itemize}
    \item
      Die Wasserbestandteile stocken wesentlich früher aus, was bei
      Temperaturen unter $0~^\circ\text{C}$ zum Verstopfen des
      Kraftstofffilters führen kann.
    \end{itemize}
  \item
    Übersäuerung des Kraftstoffs

    \begin{itemize}
    \item
      pH-Wert kann sinken, dass Korrosionsschutzschichten angegriffen
      werden.
    \end{itemize}
  \item
    Förderung des Wachstums von Bakterien (\textbf{Dieselpest})

    \begin{itemize}
    \item
      Verstopfung von Filter durch Bakterienkulturen
    \item
      Erosive Schädigung des Einspritzsystems: Die Mikroorganismen
      werden mit hoher Geschwindigkeit durch das Einspritzsystem
      gefördert und tragen dabei oberflächlich Material ab. Das kann zu
      Undichtigkeiten führen (Beispiel: Dichtsitz des Injektors).
    \end{itemize}
  \item
    Kavitation

    \begin{itemize}
    \item
      Durch den Abfall des Siedepunkts (Diesel vs.~Wasser) kann es zu
      Folgeerscheinungen (Beispiel: Druckabfall, Undichtigkeit) kommen.
    \end{itemize}
  \end{itemize}
\item
  \textbf{Hoher Flammpunkt} (Biodiesel vs.~Diesel)

  \begin{itemize}
  \item
    Während Dieselkraftstoff bei betriebswarmen Motor zumindest
    teilweise verdampft und über die Kurbelgehäuseentlüftung abgeführt
    wird, bleibt der Biodiesel nahezu vollständig im Motoröl enthalten.
    Dies führt zu Ölverdünnung und Überfüllung.
  \end{itemize}
\item
  \textbf{Geringer Energiegehalt} (Leistungsrückgang bzw. ein
  Mehrverbrauch)
\item
  \textbf{Biologisch abbaubar}
\item
  \textbf{Nahezu schwefelfrei}

  \begin{itemize}
  \item
    Vorteil, wenn Fahrzeug über NOx-Speicherkatalysator verfügt.
  \end{itemize}
\end{enumerate}

\section{Bioethanol (Ottomotoren)}\label{bioethanol-ottomotoren}

\textbf{Benzin}

\begin{enumerate}
\item
  Super E5 ($5~\%$ Ethanol) Beimischung zum ($95~\%$ Benzin)
\item
  E10 ($10~\%$ Ethanol)
\item
  \textbf{E85} ($85~\%$ Ethanol)
\end{enumerate}

\textbf{Ethanol}

\begin{itemize}
\item
  geringer Heizwert (hoher Verbrauch, geringe Reichweite)
\item
  Ethanol wirkt als Lösungsmittel und kann Dichtungen angreifen
\item
  Klopffest, steigert die Oktanzahl (ROZ104)
\end{itemize}

\section{Gasförmige Kraftstoffe (Motoren mit
Fremdzündung)}\label{gasfoermige-kraftstoffe-motoren-mit-fremdzuendung}

\begin{enumerate}
\item
  \textbf{Autogas} / Flüssiggas / \textbf{LPG} (Liquefied Petroleum Gas)

  \begin{itemize}
  \item
    Gemisch aus Propan und Butan
  \item
    Speicherung: \emph{flüssig} bei niedrigem Druck ca. 2 -- 10 bar
  \item
    Sommermischung $60~\%$ Butan und $40~\%$ Propan
  \item
    Wintermischung $40~\%$ Butan und $60~\%$ Propan
  \end{itemize}
\item
  \textbf{Erdgas}

  \begin{itemize}
  \item
    Gasgemisch, Hauptbestandteil ist Methan
  \item
    Speicherung: \textbf{CNG} (Compressed Natural Gas, komprimiertes
    Gas) \emph{gasförmig} bei Umgebungstemperatur und 200 bar
  \item
    Speicherung: \textbf{LNG} (Liquefied Natural Gas) \emph{flüssig} bei
    $- 160~^\circ\text{C}$ und 2 bar
  \end{itemize}
\item
  \textbf{Wasserstoff}

  \begin{itemize}
  \item
    ideale Kraftstoff (unbegrenzte Verfügbarkeit, Energiegehalt,
    Verbindungseigenschaften)
  \end{itemize}
\end{enumerate}

\textbf{Wie wird Wasserstoff gewonnen?} Wasserstoff wird durch
Elektrolyse gewonnen. Dabei wird mithilfe der elektrischen Energie
Wasser in Wasserstoff ($H_2$) und Sauerstoff ($O_2$) zerlegt.

\textbf{Brennstoffzellen} (kalte Verbrennungsluft) sind elektrochemische
Zellen, mit denen die chemische Energie eines geeigneten Brennstoffs
(Methanol) mit Sauerstoff ($O_2$) aus der Luft ununterbrochen in
elektrische Energie umgewandelt werden kann.

\section{Dieselkraftstoff -- schwer siedende
Kraftstoffe}\label{dieselkraftstoff-schwer-siedende-kraftstoffe}

\textbf{Eigenschaften}

\begin{itemize}
\item
  Gefahrenklasse A III, d.h. Flüssigkeiten mit einem Flammpunkt über
  $55~^\circ\text{C}$
\item
  zündwillig
\item
  schwer siedende Kraftstoffe
\item
  gesundheitsschädlich und umweltschädlich
\end{itemize}

\textbf{Additivierung / Additive und Auswirkung}

\begin{enumerate}
\item
  \textbf{Fließverbesserer} (kältefest, filtrierbar, filtergängig)
\item
  \textbf{Schmierfähigkeit} (Schwefelersatz)
\item
  \textbf{Biozide}

  \begin{itemize}
  \item
    vermeiden von Bakterienwachstum (sonst wird Material der
    Hochdruckkomponenten abgetragen, ähnlich Sandstrahleneffekt
    (abgestorbene Bakterien werden mit hoher Geschwindigkeit durch das
    Einspritzsystem gefördert))
  \item
    verhindern ein Verstopfen der Filtersysteme (durch hohe Anzahl von
    Bakterien)
  \end{itemize}
\item
  \textbf{Zündbeschleuniger} (Cetanzahl erhöhen)
\end{enumerate}

\textbf{Cetanzahl} (CZ 51 -- 60) ist ein Maß für die Zündwilligkeit von
Dieselkraftstoff und gibt an, wie stark ein Kraftstoff zur Selbstzündung
neigt. >>Je höher die Cetanzahl, desto zündwilliger ist der
Kraftstoff.<<

\textbf{CFPP} (Cold Filter Plugging Point, Filter-Verstopfungs-Punkt bei
Kälte) Dieselkraftstoff enthält Paraffin, das bei geringen Temperaturen
kristallisiert. Die entstandenen Kristalle setzen sich in den
Kraftstofffilter und verstopfen diesen. CFPP gibt die Temperatur an,
dass den Filter für Kraftstoff nicht mehr durchfließen kann.

\textbf{Winterdiesel} ist Dieselkraftstoff, der einen geringeren CFPP
aufweist. Wird erreicht durch Zugabe von Kerosin oder Fließverbesserern.
Im Winter (16.11. - 28.02) muss Dieselkraftstoff bis mindestens
$- 20~^\circ\text{C}$ filtrierbar (Durchfließen eines Filters) sein.

Kraftstoffvorwärmung bei modernen Dieselmotoren.

\textbf{Schwefel} hat Schmierwirkung (Ausgleich durch Additive)

\begin{itemize}
\item
  Verbrennen: Schwefel und Wasser $\to$ Säure / Übersäuren (vgl.
  Eigenschaften von Biodiesel)
\item
  für NOx-Speicherkatalysator sollte Schwefelanteil gering sein
\end{itemize}

\section{Schmieröle}\label{schmieroele}

\textbf{Aufgaben}

\begin{enumerate}
\item
  Schmierung von Bauteilen (Reibung und Verschleiß vermindern)
\item
  Kühlen (Wärme abführen)
\item
  Abdichten (zwischen Kolben und Zylinder)
\item
  Reinigen (Verschleißpartikel zum Filter transportieren)
\item
  Korrosionsschutz von Oberflächen
\item
  Geräusche dämpfen
\end{enumerate}

\textbf{Merkmale von synthetischen im Vergleich zu mineralischen
Grundölen}

\begin{enumerate}
\item
  \textbf{Sehr hoher Viskositätsindex} (stabile Schmierung über einen
  großen Temperaturbereich)
\item
  \textbf{Gute Fließfähigkeit} (Kraftstoffeinsparung und schnelle
  Förderung des Öls an die Schmierstellen bei sehr niedrigen
  Temperaturen)
\item
  \textbf{Hohe Druckfestigkeit} (Schmierfilm wird auch bei starker
  Druckbelastung nicht unterbrochen)
\item
  \textbf{Gutes Schmutztrageverhalten} (Abrieb oder
  Verbrennungsrückstände werden im Öl in Schwebe gehalten)
\item
  \textbf{Sehr alterungsbeständig} (deshalb sind
  Langzeitölwechselintervalle bei Verbrennungsmotoren möglich)
\item
  \textbf{Geringe Verdampfungsverluste} (niedriger Ölverbrauch auch bei
  hohen thermischen Belastungen)
\end{enumerate}

\textbf{Anforderung an Motorenöle}

\begin{enumerate}
\item
  Schmieren (Lager, Gleitstellen Kolben oder Zylinder)
\item
  Kühlen (ableiten der Wärme vom Kolben)
\item
  Abdichten (Zwischen Kolbenringen und Zylinderlaufbuchsen,
  Feinabdichtung an Radialwellendichtringe)
\item
  Reinigen (Aufnehmen von Verbrennungsrückständen, Abrieb, Wasser,
  Säuren)
\item
  Geräusche dämpfen
\item
  Hohe thermische Stabilität (geringe temperaturabhängige
  Viskositätsänderung)
\item
  Geeignet für Katalysatoren, Dieselpartikelfilter und Ladermotoren
\item
  geringe Verdampfungsverluste (geringer Ölverbrauch,
  Ölkohleablagerungen)
\end{enumerate}

\textbf{Einteilung der Motoröle / Klassifizierung}

\begin{enumerate}
\item
  \textbf{SAE-Viskositätsklassen:} (Auswahl nach Temperaturbereich)

  \begin{itemize}
  \item
    Einbereichsölen (Beispiel: SAE 50)
  \item
    Mehrbereichsölen (Beispiel: SAE 0W-40)
  \end{itemize}
\item
  Leistungsklassen:

  \begin{itemize}
  \item
    \textbf{API} höhere Anforderungen

    \begin{itemize}
    \item
      S-Klassen für Ottomotoren
    \item
      C-Klassen für Dieselmotoren
    \end{itemize}
  \item
    \textbf{ACEA} (europäische Klasse) Mindestanforderungen an die
    Qualität

    \begin{itemize}
    \item
      A-Klassen-Öle für Ottomotoren
    \item
      B-Klassen-Öle für Pkw-Dieselmotoren
    \item
      E-Klassen-Öle für Nfz-Dieselmotoren
    \end{itemize}
  \item
    \textbf{ILSAC} entspricht in etwa der API-Norm

    \begin{itemize}
    \item
      International
    \end{itemize}
  \end{itemize}
\end{enumerate}

\textbf{Freigabe-Vorschriften der Hersteller} unbedingt beachten, um
Motorschäden oder Schäden an der Einspritzanlage zu vermeiden.

\textbf{Mehrbereichsöle} sind Schmieröle, die mehr als eine
Viskositätsklasse abdecken. Beispiel: SAE 15W-40 verhält sich bei tiefen
Temperaturen wie ein Öl der Klasse 15W und bei hohen Temperaturen wie
ein Öl der Klasse 40. Kaltstarterleichterung und schnelle Durchölung bei
niedrigen Außentemperaturen und Temperaturfestigkeit bei hohen
Temperaturen.

Kaltstart und Wärmebelastbarkeit (geringe Reibung, gute
Schmierfähigkeit)

\textbf{Pourpoint} (Grenzpumptemperatur) ist die Temperatur, bei der das
Öl gerade noch fließt. \textbf{Stockpunkt} gibt die Temperatur an, bei
der das Öl >>stockt<<. Meist ist der Stockpunkt $5^\circ\text{C}$
tiefer als der Pourpoint. (Beispiel: SAE 5W Stockpunkt
$- 35^\circ\text{C}$) Dadurch ist gewährleistet, dass beim Motorstart
genügend Öl zur Ölpumpe und in den Schmierölkreislauf fließt.

>>Je höher die SAE-Kennzahl, desto zähflüssiger ist das Öl.<<

\textbf{Was bedeutet bei der SAE-Klasse 10W-40 der Buchstabe W?}

Der Buchstabe \emph{W} bedeutet Winter. >>Je kleiner die Zahl vor dem
\emph{W}, desto fließfähiger ist das Öl in der Kälte.<<

\textbf{Was sind Leichtlauföle?}

Als Leichtlauföle (z.B. 0W-30) bezeichnet man Mehrbereichsöle, die ein
sehr gutes Niedrigtemperaturverhalten und bei hohen Temperaturen eine
Viskosität wie ein Einbereichsöl SAE 30 bieten. Geringe Reibung bei
Kaltstart.

\textbf{Warum sind bei Dieselmotoren mit Dieselpartikelfiltern spezielle
Öle erforderlich?}

Um einen niedrigen Schwefel-, Asche- und Phosphorgehalt zu erreichen.
Ascherückstände aus dem Öl lagern sich in den Dieselpartikelfiltern ab
und verringern dessen Speicherkapazität. Da Ascherückstände selbst bei
hohen Temperaturen nicht frei gebrannt werden können, kommt es zu
Fehlfunktionen und letztlich zum vollständigen Ausfall des Filters.

\textbf{Viskosität} ist ein Maß für die Zähflüssigkeit des Öls und
entspricht der inneren Reibung (Schubspannung). Öl hat eine niedrige
Viskosität, wenn es dünnflüssig ist und eine hohe Viskosität, wenn es
zähflüssig ist. >>Je nach Ölsorte ist die Viskosität verschieden groß,
sie nimmt mit steigender Temperatur ab.<<

\textbf{Nenne 3x Additive und Eigenschaften}

\begin{enumerate}
\item
  \textbf{Detergants} Reinigungszusätze (lösen Ablagerungen von
  Metalloberflächen Beispiel: Kolben und Ölleitungen, halten dadurch
  geschmierte Oberflächen sauber)
\item
  \textbf{Dispersants} Schlammtragende Zusätze (halten Stoffe, die beim
  Verbrennungsprozess entstehen, in der Schwebe und verhindern somit
  Ablagerungen)
\item
  \textbf{Verschleißschutzzusätze} (EP - Extreme Pressure) (die unter
  hohen Druck stehenden metallischen Gleitflächen einen übermäßigen
  Verschleiß verhindern, Beispiel: Zahnradflanken oder zwischen Nocken
  und Tassenstößel)
\item
  \textbf{Korrosionsschutzzusätze} (bauen auf nicht mehr Ölbenetzen
  Metalloberflächen wasserabweisende Schutzfilme auf, schützen vor
  aggressiven Verbrennungsrückständen und neutralisieren Säuren)
\item
  \textbf{Reibwertveränderer} (beeinflussen gezielt den Reibwert
  zwischen Materialpaarungen. Beispiel: erforderlich bei
  Synchrongetrieben, Nasskupplungen, Lamellenkupplung in
  Automatikgetrieben)
\item
  \textbf{Alterungsschutzadditive} (verhindern die Oxidation des Öls
  unter Einfluss von Wärme und Sauerstoff)
\item
  \textbf{Stockpunkterniedriger} (verbessert die Fließeigenschaften des
  Öls bei tiefen Temperaturen. Beispiel: verringert Motorverschleiß bei
  Kaltstart)
\item
  \textbf{Antischaum} verhindern die Schaumbildung im Öl.
\item
  \textbf{Viskositätsverbesserer} VI-Verbesserer (sind im kalten Zustand
  zusammengeknäuelt im Öl enthalten. Erwärmt sich das Öl entknäueln sie
  sich und nehmen ein größeres Volumen ein. Dadurch wirken sie der
  zunehmenden Dünnflüssigkeit des Öls bei Erwärmung entgegen und können
  einen belastungsfähigeren Schmierfilm aufbauen.)
\end{enumerate}

\textbf{Wodurch altert Öl?}

\begin{enumerate}
\item
  Druck und Temperatur
\item
  Sauerstoff ($O_2$)
\item
  Laufkilometer und Zeit
\end{enumerate}

\textbf{Schlammablagerungen} wird durch Alterungsprodukte, Ruß,
undverbrannte Kraftstoffreste, Stickoxide und Wasser verursacht.
\textbf{Folge} sind Verstopfen von Ölleitungen und Ölfiltern, erzeugen
von Fressschäden an Kolben und Zylinderlaufbahnen sowie Lagerschäden.

\textbf{Schaumbildung} dadurch wird der Ölfilm unterbrochen, Ölalterung
beschleunigt und die Kompressibilität des Öls erhöht. \textbf{Folge} (1)
Schmiereigenschaften verringert sich, dadurch sind Fressschäden möglich.
(2) Ölwechselintervalle verkürzen sich (3) Störung bei der
Kraftübertragung durch verringerten Druckaufbau in hydraulischen
Schaltelementen

Neues Öl ist basisch (>>Motor sauer fahren<<).

Ölanalyse \footnote{\url{https://de.oelcheck.com/}}

\section{Getriebeöle}\label{getriebeoele}

\textbf{Anforderungen}

\begin{enumerate}
\item
  Verschleißschutz (an Zahnflanken und Lagerlaufflächen. Besonders bei
  Hypoidantrieben kann der Schmierfilm weg gequetscht werden, was zu
  erhöhtem Verschleiß führt.)
\item
  Unterschiedliches Reibverhalten (beim Synchronisieren muss der Ölfilm
  abgetragen werden können)
\item
  Alterungsschutz (über die gesamte Lebensdauer)
\item
  Dichtungsverträglichkeit
\end{enumerate}

\textbf{Welche Viskositätsklassen gelten für Mehrbereichsgetriebeöle?}

\begin{enumerate}
\item
  SAE 80W-90
\item
  SAE 75W-90 (Leichtlauf-Getriebeöl)
\end{enumerate}

\textbf{Nennen Sie die Besonderheit der Getriebeöle für Hypoidachsen.}

Getriebeöle für Hypoidachsen sind, um den Verschleiß gering zu halten,
mit einem sehr hohen Anteil an EP-Zusätzen (Extrem Pressure,
Lasttrageverhalten) versehen, die an den Metalloberflächen
Schutzschichten bilden, damit der Schmierfilm zwischen den
Zahnradflanken nicht weggedrückt wird.

\textbf{Welche Aufgaben/Anforderungen haben Automatikgetriebeöle?}

\begin{enumerate}
\item
  Drehmomentübertragung von Pumpen- zum Turbinenrad
\item
  Schmieren von Lager, Planetenrädern und Freiläufe,
\item
  Betätigen von Lamellenkupplungen
\end{enumerate}

\section{Schmierfette}\label{schmierfette}

\textbf{Schmierfette} sind eingedickte Schmieröle. Sie bestehen aus
einer Basisflüssigkeit (Mineralöl), die durch Gerüstbildner (Eindicker,
bilden Struktur des Fettes) zu einer pastenartigen Masse stabilisieren.

\textbf{Konsistenz} (NLGI-Klassen) ist der Widerstand eines Fettes gegen
Verformung.

\begin{itemize}
\item
  000 -- 1 sehr weich (Beispiel: Fließfette für Zentralschmieranlagen)
\item
  2 -- 3 weich (Beispiel: Abschmierfette)
\item
  4 -- 5 fest (Beispiel: Wasserpumpenfette)
\end{itemize}

\textbf{Eigenschaften von Schmierfetten}

\begin{table}[!ht]% hier: !ht 
\centering 
	\caption{}% \label{tab:}%% anpassen 
\begin{tabular}{@{}llll@{}}
\hline
\textbf{Seifenbasis} & \textbf{Tropfpunkt} in $^\circ\text{C}$ &
\textbf{wasserfest} & \textbf{Verwendung} \\
\hline
Kalziumseifenfett & bis 200 & ja & Abschmierfett \\
Natriumseifenfett & 120 -- 250 & nein & Wälzlagerfett \\
Lithiumseifenfett & 100 -- 200 & ja & Mehrzweckfett \\
\hline
\end{tabular} 
\end{table}

\textbf{EP-Schmierfette} (Extreme-Pressure, hohen Drücken standhalten)

\textbf{Hochtemperaturfette} ($> 130^\circ\text{C}$)

\textbf{Tropfpunkt} ist die Temperatur, bei der unter Prüfbedingungen,
der erste Tropfen des schmelzenden Schmierfettes abtropft.
