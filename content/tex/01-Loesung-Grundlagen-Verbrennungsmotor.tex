%ju 08-Jun-22 01-Loesung-Grundlagen-Verbrennungsmotor.tex
\textbf{Bemerkung}, die in Klammern stehende Kommentare gehören nicht
zur Beantwortung der Frage.

\textbf{1) Ein Verbrennungsmotor benötigt zum Arbeiten ein
Kraftstoff-Luft-Gemisch. Was ist Kraftstoff und was ist Luft?}

Fachbuch (\textcite{brand:2020:fachkundeKfz} S. 29)

\begin{enumerate}
\def\labelenumi{(\arabic{enumi})}
\item
  \textbf{Kraftstoffe} sind hauptsächlich Kohlen - Wasserstoff -
  Verbindungen (geringer Anteil Schwefel $\to$ Schmierung). Die Anzahl
  der Atome und deren Verbindungen bestimmen die Art des Kraftstoffes.
  Zur Verbesserung der Eigenschaften werden Ihnen Additive zugefügt.
\end{enumerate}

(Flammpunkt; Benzin: ringförmiger Molekülaufbau, Oktanzahl $\to$
Zündunwillig, Klopffestigkeit; Diesel: kettenförmiger Molekülaufbau,
Cetanzahl $\to$ Zündwilligkeit)

\begin{enumerate}
\def\labelenumi{(\arabic{enumi})}
\setcounter{enumi}{1}
\item
  \textbf{Luft} ist ein Gasgemisch aus
\end{enumerate}

\begin{itemize}
\item
  $78~\%$ Stickstoff
\item
  $21~\%$ Sauerstoff
\item
  $0,9~\%$ sonstige Gase (Edelgase)
\item
  $0,1~\%$ Schwebeteilchen (Partikel, Feinstaub, Sandstrahl verschleiß
  $\to$ Luftmassenmesser)
\item
  ($0,040~\% ~ CO_2$)
\end{itemize}

\textbf{2) Unterscheiden Sie Boxer-Motor und 180 Grad V-Motor}

\textbf{Boxer-Motor} arbeiten die Kolben der gegenüberliegenden Zylinder
aufeinander zu. Es befinden sich also beide zeitgleich im oberen oder
unteren Totpunkt. Um dies zu ermöglichen, benötigt der Boxer-Motor einen
Kurbelzapfen je Kolben.

(flache Bauweise, tiefer Schwerpunkt, Kurvenverhalten)

$180^\circ~$\textbf{V-Motor} arbeiten die Kolben der
gegenüberliegenden Zylinder in gleicher Richtung. Wenn also der eine im
oberen Totpunkt ist, ist der andere im unteren Totpunkt und umgekehrt.
Beim $180^\circ~$V-Motor können sich daher jeweils zwei Kolben ein
Kurbelzapfen teilen.

\textbf{3) Wie unterscheidet man Hubkolbenmotoren nach dem
Arbeitsverfahren?}

\textbf{Arbeitsverfahren}

\begin{enumerate}
\def\labelenumi{(\arabic{enumi})}
\item
  Vier-Takt-Motor
\item
  Zwei-Takt-Motor
\end{enumerate}

\textbf{4) Was bezeichnet man als Hubraum?}

Raum zwischen unteren und oberen Totpunkt eines Zylinders.

\textbf{5) Erläutern Sie, welche Faktoren den Druck im Brennraum am Ende
des Verdichtungstaktes beeinflussen}

\begin{enumerate}
\def\labelenumi{(\arabic{enumi})}
\item
  \textbf{Druck} zu Beginn des Verdichtungstaktes

  \begin{itemize}
  \item
    bei Saugmotoren um den atmosphärischen Luftdruck
  \item
    bei Ladermotoren (externe Aufladung) Überdruck von bis zu
    $2,2~bar$
  \end{itemize}
\item
  Als nächstes wäre die \textbf{Temperatur} der angesaugten Luft zu
  nennen.
\item
  \textbf{Verdichtungsverhältnis}

  \begin{itemize}
  \item
    Verkleinerung des Raumes und damit verdichten des Gases
  \item
    Ausdehnung des Gases durch Erwärmung

    \begin{itemize}
    \item
      pro Grad der Erwärmung $\frac{1}{273}$
    \end{itemize}
  \end{itemize}
\item
  \textbf{Verluste}

  \begin{itemize}
  \item
    durch Wärmeentzug an der Brennraumoberfläche (Brennraumgestaltung)
  \item
    durch Brennraumundichtigkeiten

    \begin{itemize}
    \item
      Kolbenringe, Ventile, Brennraumabdichtung, \ldots{}
    \end{itemize}
  \end{itemize}
\end{enumerate}

\textbf{6) Worin besteht der Unterschied zwischen Kurz-, Lang- und
Quadrathuber?}

\emph{Prüfung}

\textbf{Hub-Bohrung-Verhältnis}

\begin{enumerate}
\def\labelenumi{(\arabic{enumi})}
\item
  \textbf{Kurzhuber} Hub $<$ Zylinderbohrung

  \begin{itemize}
  \item
    (\emph{Formel 1} $\to$ sehr hohe Drehzahlen)
  \end{itemize}
\item
  \textbf{Langhuber} Hub $>$ Zylinderbohrung

  \begin{itemize}
  \item
    (\emph{Schiff, Traktor, Lanz Bulldog} $\to$ Drehmoment, Leistung,
    Länge des Kurbelzapfens, Hebelarm, höhere mittlere
    Kolbengeschwindigkeit, Massenträgheit, Drehzahlbegrenzung)
  \end{itemize}
\item
  \textbf{Quadrathuber} Hub $=$ Zylinderbohrung
\end{enumerate}

\textbf{7) Worin unterscheiden sich Kipp- und Schlepphebel?}

Fachbuch (\textcite{brand:2020:fachkundeKfz} S. 242)

\begin{enumerate}
\def\labelenumi{(\arabic{enumi})}
\item
  \textbf{Kipphebel} ist in der Mitte gelagert und besitzt dadurch zwei
  Arme. Der eine Arm wird direkt vom Nocken über einen Stößel oder von
  der unten liegenden Nockenwelle über Stößel und Stößelstange betätigt.
  Der andere Kipphebelarm betätigt das Ventil.
\item
  \textbf{Schlepphebel oder Schwinghebel} besitzt nur einen Arm. Dieser
  ist an einem Ende gelagert und stützt sich mit dem anderen Ende auf
  das Ventil. Der Nocken wirkt von oben auf diesen Arm.
\end{enumerate}

\textbf{8) Was wird in einem Steuerdiagramm dargestellt?}

Fachbuch (\textcite{brand:2020:fachkundeKfz} S. 195)

In einem \textbf{Steuerdiagramm} werden die Steuerzeiten eines
Verbrennungsmotors in >>Grad Kurbelwinkel<< dargestellt.

\textbf{9) Was wird als Ventilüberschneidung bezeichnet?}

\textbf{Ventilüberschneidung} bezeichnet man den Drehwinkel, den die
Kurbelwelle zwischen >>EV öffnet vor OT<< und >>AV schließt nach OT<<
durchläuft.

\textbf{10) Erläutern Sie die Begriffe Passlager, Minutenring und
Trockensumpfschmierung}

Fachbuch (\textcite{brand:2020:fachkundeKfz} S. 213)

\begin{enumerate}
\def\labelenumi{(\arabic{enumi})}
\item
  \textbf{Passlager} bezeichnet man das/die Hauptlager, das die
  Kurbelwelle gegen axiales verschieben z. B. beim Auskuppeln sichert.
\end{enumerate}

(Gleitlager, Wälzlager, radial, axial)

\begin{enumerate}
\def\labelenumi{(\arabic{enumi})}
\setcounter{enumi}{1}
\item
  \textbf{Minutenring} ist ein spezieller Kolbenring, der durch seine
  trapezförmige Form im Neuzustand eine sehr schmale Dichtkante zum
  Zylinder hat. Wodurch er sich in kürzester Zeit auf den Zylinder
  einschleift und Einfahrvorschriften entfallen können.
\end{enumerate}

(Kompressionsring, Ölabstreifring)

\begin{enumerate}
\def\labelenumi{(\arabic{enumi})}
\setcounter{enumi}{2}
\item
  \textbf{Trockensumpfschmierung} bezieht das Schmieröl nicht direkt aus
  der Ölwanne, sondern aus einem separaten Tank. Die Ölwanne enthält nur
  eine geringe Ölmenge, die durch eine Ölpumpe kontinuierlich in den
  Öltank abgeführt wird. Hierdurch wird ein Trockenlaufen durch große
  Fliehkräfte oder Schräglage entgegengewirkt.
\end{enumerate}

(Zwei oder zweistufige Ölpumpen: Saugpumpe, Druckpumpe;
Druckumlaufschmierung/Nasssumpf)
