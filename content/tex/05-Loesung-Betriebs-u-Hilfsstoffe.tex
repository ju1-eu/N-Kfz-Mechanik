%ju 05-Jun-22 05-Loesung-Betriebs-u-Hilfsstoffe.tex
\textbf{1) Was versteht man unter einer fraktionierenden Destillation
und welche Produkte fallen dabei an?}

Aufteilen von Rohöl nach Siedebereichen. Das Rohöl wird in einem
Röhrenofen auf ca. $360^\circ\text{C}$ erhitzt und anschließend in
einem Turm mit mehreren Ebenen geleitet. Die Kraftstoffdämpfe steigen
nach oben und kondensieren dabei nach und nach (Temperaturabnahme).
Zunächst wird Diesel von Petroleum, Schwer- und Leichtbenzin getrennt.
Propan und Butan wird zu LPG (verflüssigtes Petroleum Gas)
weiterverarbeitet.

Die Rohölbestandteile, die den Ofen flüssig verlassen, werden nach
nochmaligen erhitzen in einen weiteren Turm geleitet. Der Druckabfall
senkt den Siedebereich der Flüssigkeiten und es wird nach den gleichen
Verfahren Öle gewonnen, zum Beispiel Motoröl. Der Rest ist Bitumen.

\textbf{2) Was versteht man unter Viskosität?}

Es ist ein Maß für die Zähflüssigkeit des Öls und entspricht der inneren
Reibung.

Öl hat eine niedrige Viskosität, wenn es dünnflüssig ist und eine hohe
Viskosität, wenn es zähflüssig ist.

\textbf{3) Was versteht man unter Cracken und was wird dadurch erreicht?
Welche Arten des Crackens gibt es?}

Stoffumwandlung in der Erdölverarbeitung

\begin{enumerate}
\item
  langkettige Kohlenwasserstoffmoleküle (schwer siedend) werden unter
  Wärme und Druck oder mithilfe eines Katalysators in kurzkettige
  Kohlenwasserstoffmoleküle (leicht siedend, Benzin, Gas) zerteilt
\item
  Verfahren zur Erhöhung der Klopffestigkeit von Ottokraftstoffen
\item
  Thermisches Cracken und katalytisches Cracken. Diese beiden Gruppen
  unterscheiden sich im Wesentlichen dadurch, dass beim thermischen
  Cracken keine Katalysatoren eingesetzt werden.
\end{enumerate}

\textbf{4) Was gibt die Cetanzahl an?}

Zündwilligkeit von Dieselkraftstoff (Wie stark ein Kraftstoff zur
Selbstzündung neigt?)

\textbf{5) Welche Aufgabe haben Biozide als Additiv im
Dieselkraftstoff?}

Biozide vermeiden das Bakterienwachstum, verhindern ein Verstopfen der
Filtersysteme und ermöglichen eine Lagerzeit von bis zu 2 Jahren.

\textbf{6) Warum dürfen moderne Dieselmotoren keinesfalls mit
Ottokraftstoff betrieben werden?}

Die Schmierfähigkeit von Ottokraftstoff ist sehr gering. Durch die
reinigende Wirkung wäscht er den Schmierfilm von Dieselkraftstoff ab,
sodass die Bauteile des Einspritzsystems -- vorwiegend der Kolben der
Hochdruckpumpe -- nahezu trockenreiben, was zu sofortiger Spanbildung
führt. Die Späne werden durch den Rücklauf im kompletten
Kraftstoffsystem verteilt und lösen Erosionen, sowie im Bereich
beweglicher Teile weitere Spanbildung aus.

\textbf{7) Was versteht man unter E10-Kraftstoff?}

Superbenzin E10 mit einem Anteil Bioethanol von $10~\%$ und $90~\%$
Benzin.

Ethanol wird aus Biomasse hergestellt (Getreide, Mais, Zuckerrüben). Bei
der Verbrennung von Bioethanol wird nur so viel $CO_2$ ausgestoßen,
wie die Nutzpflanze während ihres Wachstums aufgenommen hat. Was zu
weniger Treibhausgasen führt und damit zum Klimaschutz beiträgt.

\textbf{8) Welche Anforderungen werden an Motoröl gestellt?}

\begin{itemize}
\item
  stabilen Schmierfilm bilden, um die Reibung zu vermindern
\item
  feinabdichten
\item
  Kühlen
\item
  vor Korrosion schützen
\item
  Schaumbildung verhindern
\item
  gegenüber hohen Temperaturen beständig sein
\end{itemize}

\textbf{9) Welche Gruppen von Fetten unterscheidet man?}

\textbf{Man unterscheidet die Fette nach der Art des verwendeten
Eindickers}

\begin{enumerate}
\item
  Lithiumseifen-Schmierfette
\item
  Natriumseifen-Schmierfette
\item
  Calziumseifen-Schmierfette
\end{enumerate}

\textbf{10) Welche Eigenschaften hat Bremsflüssigkeit (DOT 4)?}

\begin{itemize}
\item
  hygroskopisch
\item
  Mindestsiedepunkt $230~^\circ\text{C}$
\item
  giftig
\item
  greift Lacke an
\end{itemize}
