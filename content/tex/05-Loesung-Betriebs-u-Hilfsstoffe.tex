%ju 08-Jun-22 05-Loesung-Betriebs-u-Hilfsstoffe.tex
\textbf{1) Was versteht man unter einer fraktionierenden Destillation
und welche Produkte fallen dabei an?}

Aufteilen von Rohöl nach Siedebereichen. Das Rohöl wird in einem
Röhrenofen auf ca. $360^\circ\text{C}$ erhitzt und anschließend in
einem Turm mit mehreren Ebenen geleitet. Die Kraftstoffdämpfe steigen
nach oben und kondensieren dabei nach und nach (Temperaturabnahme).
Zunächst wird Diesel von Petroleum, Schwer- und Leichtbenzin getrennt.
Propan und Butan wird zu LPG (verflüssigtes Petroleum Gas)
weiterverarbeitet.

Die Rohölbestandteile, die den Ofen flüssig verlassen, werden nach
nochmaligen erhitzen in einen weiteren Turm geleitet. Der Druckabfall
senkt den Siedebereich der Flüssigkeiten und es wird nach den gleichen
Verfahren Öle gewonnen, zum Beispiel Motoröl. Der Rest ist Bitumen.

\textbf{2) Was versteht man unter Viskosität?}

\textbf{Viskosität} ist ein Maß für die Zähflüssigkeit von
Flüssigkeiten. Öl hat eine \emph{niedrige Viskosität}, wenn es
\emph{dünnflüssig} (fließfähiger) ist und eine \emph{hohe Viskosität},
wenn es \emph{dickflüssig} (zähflüssig, weniger fließfähig) ist. >>Je
nach Ölsorte ist die Viskosität verschieden groß, sie nimmt mit
steigender Temperatur ab.<<.

\textbf{3) Was versteht man unter Cracken und was wird dadurch erreicht?
Welche Arten des Crackens gibt es?}

\begin{enumerate}
\item
  langkettige Kohlenwasserstoffmoleküle (schwer siedend) werden unter
  Wärme und Druck (oder mithilfe eines Katalysators) in kurzkettige
  Kohlenwasserstoffmoleküle (leicht siedend) zerteilt
\item
  Verfahren zur Erhöhung der Klopffestigkeit von Ottokraftstoffen
\item
  Thermisches Cracken und katalytisches Cracken.
\end{enumerate}

\textbf{4) Was gibt die Cetanzahl an?}

Ist ein Maß für die Zündwilligkeit von Dieselkraftstoff (also wie stark
ein Kraftstoff zur Selbstzündung neigt).

\textbf{5) Welche Aufgabe haben Biozide als Additiv im
Dieselkraftstoff?}

\begin{itemize}
\item
  vermeiden von Bakterienwachstum (sonst wird Material der
  Hochdruckkomponenten abgetragen, ähnlich Sandstrahleneffekt
  (abgestorbene Bakterien werden mit hoher Geschwindigkeit durch das
  Einspritzsystem gefördert))
\item
  verhindern ein Verstopfen der Filtersysteme (durch hohe Anzahl von
  Bakterien)
\end{itemize}

\textbf{6) Warum dürfen moderne Dieselmotoren keinesfalls mit
Ottokraftstoff betrieben werden?}

Wegen der geringen Schmierfähigkeit von Ottokraftstoff. Durch die
reinigende Wirkung wird der Schmierfilm abgewaschen, sodass die Gefahr
des trockenlaufens besteht. Beispiel ist der Kolben der Hochdruckpumpe
vom Einspritzsystem.

\textbf{7) Was versteht man unter E10-Kraftstoff?}

Superbenzin E10 mit einer Beimischung von $10~\%$ Bioethanol.

\textbf{8) Welche Anforderungen werden an Motoröl gestellt?}

\begin{enumerate}
\item
  Schmieren (Lager, Gleitstellen von Kolben und Zylinder)
\item
  Kühlen (ableiten der Wärme vom Kolben)
\item
  Abdichten (Zwischen Kolbenringen und Zylinderlaufbuchsen,
  Feinabdichtung an Radialwellendichtringe)
\item
  Reinigen (Aufnehmen von Verbrennungsrückständen, Abrieb, Wasser,
  Säuren)
\item
  Geräusche dämpfen
\item
  Hohe thermische Stabilität (geringe temperaturabhängige
  Viskositätsänderung)
\item
  Geeignet für Katalysatoren, Dieselpartikelfilter und Ladermotoren
\item
  geringe Verdampfungsverluste (geringer Ölverbrauch,
  Ölkohleablagerungen)
\end{enumerate}

\textbf{9) Welche Gruppen von Fetten unterscheidet man?}

\textbf{Man unterscheidet die Fette nach der Art des verwendeten
Eindickers}

\begin{enumerate}
\item
  Lithiumseifen-Schmierfette
\item
  Natriumseifen-Schmierfette
\item
  Calziumseifen-Schmierfette
\end{enumerate}

\textbf{10) Welche Eigenschaften hat Bremsflüssigkeit (DOT 4)?}

\begin{itemize}
\item
  hygroskopisch
\item
  Mindestsiedepunkt $230~^\circ\text{C}$
\item
  giftig
\item
  greift Lacke an
\end{itemize}
