%ju 08-Jun-22 Motoroel.tex
\textbf{Aufgabe} Schmieren, Kühlen und vor Korrosion schützen, Reinigen,
Abdichten, Geräusche dämpfen

\textbf{Zusammensetzung}

\begin{enumerate}
\item
  Grundöl $80~\%$

  \begin{itemize}
  \item
    Mineralöl: \emph{Erdöl - Destillieren - schwersiedende Bestandteile
    - Raffinieren (reinigen/veredeln)} z. B. 15W-40, Hochtemperatur
    stabil bis: 130 °C
  \item
    Synthetisches Öl: \emph{Erdöl - Destillieren - Rohbenzin - Cracken}
    z. B. 5W-30, Hochtemperatur stabil bis: 180 °C
  \end{itemize}
\item
  Additive (Eigenschaften verbessern) $20~\%$
\end{enumerate}

\textbf{Viskosität}

\begin{itemize}
\item
  Grad der Zähflüssigkeit (Fließfähigkeit bei niedrigen und hohen
  Temperaturen)
\item
  niedrige Viskosität: dünnflüssig, hohe Viskosität: zähflüssig
\item
  SAE-Viskositätklasse, Mehrbereichsöl, z. B. 5W-30 (Tieftemperatur,
  Winter, Hochtemperatur)
\end{itemize}

\textbf{Hochtemperatur Querstabilität} Schmierfilm nicht abreißen
(Temperatur, Drehzahl abhängig)

\textbf{API - Leistungsklassen (Amerika)}

\textbf{ACEA - Spezifikation (Europa)}

Mindestanforderung an die Qualität von Motorölen

\textbf{Automobilhersteller Spezifikation}

\textbf{Anteil gering} Schwefel, Asche, Phosphor (Abgasnachbehandlung)

\textbf{Ölstand} >>Bei betriebswarmen Motor messen.<<

\begin{enumerate}
\item
  zu wenig
\item
  zu viel (Nachteil)

  \begin{itemize}
  \item
    Öl verbrennen > Motor durchgehen
  \item
    Öl aufschäumen > keine Schmierwirkung
  \end{itemize}
\end{enumerate}

>>Kurzstrecke<< (Wasser, Kraftstoff > Öl)

\begin{itemize}
\item
  Motor kalt: Ölstand hoch
\item
  Motor warm: niedrig
\end{itemize}

\textbf{Haltbarkeit}

\begin{itemize}
\item
  \emph{Altes Öl} ca. drei Jahre haltbar (wenn geöffnet)
\item
  \emph{Neues Öl} ca. fünf Jahre haltbar (original verschlossen)
\end{itemize}

\textbf{Hersteller}

Wolf Öl \footnote{\url{https://www.wolflubes.com/de_de/products/default.aspx}}
5W-30 C1 Produktcode: 65605

Castrol EDGE 5W-30 LL Motorenöl
