%ju 05-Jun-22 03-Fuellungsoptimierung-I.tex
\section{Downsizing (Prüfung)}\label{downsizing-pruefung}

Verkleinerung der Motoren (Hubraum und Zylinderzahl) bei gleicher
Leistung.

\section{LSPI - Low
Speed-Pre-Ignition}\label{lspi-low-speed-pre-ignition}

LSPI = vorzeitige Zündung, betrifft hoch aufgeladene Downsizing Motoren
\footnote{\url{https://www.autobild.de/artikel/lspi-vorzeitige-zuendung-16385077.html}}

\begin{itemize}
\item
  \textbf{Turbo aufgeladene Motoren}

  \begin{itemize}
  \item
    geringes Verdichtungsverhältnis (7-8:1)
  \item
    vor verdichtete Luft wird in den Zylinder eingeblasen und verdichtet
  \item
    Ladedruckregelung (Lastwunsch)
  \item
    vorgewärmte Luft (Ladeluftkühlung)
  \end{itemize}
\item
  vs.~\textbf{hoch verdichtete Saugmotoren}

  \begin{itemize}
  \item
    hohes Verdichtungsverhältnis (10-11:1), endet bei Klopfgrenze
  \end{itemize}
\end{itemize}

\textbf{Zwei Zündquellen, Ursache für die Selbstentzündung}

\begin{enumerate}
\item
  Niedergeschlagen Kraftstoff in Verbindung mit sehr niedrig Viskoses Öl

  \begin{itemize}
  \item
    $\to$ ein brennbares Gemisch entsteht, mit einer nicht ganz
    bekannten Selbstentzündungstemperatur
  \end{itemize}
\item
  Ölkohlerückstände (Kraftstoffreste) im Bereich der Einspritzdüsen
\end{enumerate}

Durch eine überhohe Verdichtung $\to$ steigt Verdichtungsenddruck und
damit Verdichtungstemperatur $\to$ dadurch hohe thermische Belastung.
Die Folge ist ein kapitaler Motorschaden.

\textbf{Körnerschlag} \footnote{\url{https://cdn.germanscooterforum.de/monthly_05_2009/post-24449-1241606436.jpg}}
Kolbenschäden $\to$ es entsteht eine Druckspritze bevor der Kolben OT
erreicht, eine zweite Flammenfront entsteht, wenn jetzt zwei
Flammfronten aufeinandertreffen, entstehen sehr hohe Druckspitzen, auch
wenn der Kolben nach UT geht.

\textbf{Kavitation} \footnote{\url{https://prozesstechnik.industrie.de/wp-content/uploads/4/0/40278086.jpg}}
Dampfblasenbildung \footnote{\url{https://www.youtube.com/watch?v=SEGTFbZ5RJ8}}
z. B. Bootsschraube saugt Flüssigkeiten an, Druck fällt ab durch
Unterdruck, wenn jetzt die Gasblasen implodieren, entstehen sog.
Mikrojets $\to$ Druckspitzen.

\section{Vorteile von Downsizing
Motoren}\label{vorteile-von-downsizing-motoren}

\begin{enumerate}
\item
  Geringere Pumpverluste (2 l vs.~1,2 l bei gleicher Leistung 150 PS)
\item
  geringere Reibungsverluste aufgrund der geringeren Größe
\item
  weniger Wärmeübertrag von Gasen zur Zylinderwandung
\end{enumerate}

\section{Mehrventiltechnik}\label{mehrventiltechnik}

Fachbuch (\textcite{respondeck:2019:servicetechniker} S. 141)

Um die Zylinderfüllung zu verbessern, werden drei oder mehr Ventile pro
Zylinder in Verbrennungsmotoren eingesetzt.

\textbf{Ziele von Mehrventiltechnik}

\begin{itemize}
\item
  Öffnungsquerschnitt der Ventile vergrößern, ohne die
  Drehzahlfestigkeit durch größere und damit trägere Ventile (mehr
  Masse) zu mindern.
\end{itemize}

\textbf{Vor- und Nachteile von Mehrventiltechnik}

\begin{itemize}
\item
  bessere Zylinderfüllung
\item
  Drehzahlfest
\item
  innere Reibung steigt
\item
  Abgaswärmeentzug

  \begin{itemize}
  \item
    Der Katalysator kommt schlechter auf Betriebstemperatur, da sich die
    Abgase an den Abgasrohren abkühlen können.
  \item
    Je mehr Auslassventile vorhanden sind, desto größer ist die
    Oberfläche der Abgasrohre und desto mehr kühlen die Abgase aus.
  \end{itemize}
\end{itemize}

Honda NR 750 - Ovalkolben \footnote{\url{https://de.wikipedia.org/wiki/Honda_NR_750}}

\textbf{Dreiventiltechnik} (Vorteile)

\begin{itemize}
\item
  Verbrennungsdruck steigt (kürzere Flammwege)
\item
  geringere Klopfneigung (weniger Zeit zur Gemischerwärmung vor
  Verbrennungsbeginn)
\item
  Ausstoß unverbrannter Kohlenwasserstoffe verringert sich (Zündkerze
  ist in der Nähe der Zylinderwand, wo das Kondensat lagert)
\item
  geringere NOx
\end{itemize}

Vgl. Kapitel >>\emph{Motorsteuerung / Dreiventiltechnik mit zwei
Zündkerzen}<<

\section{Nockenwellenverstellung - variable
Steuerzeiten}\label{nockenwellenverstellung-variable-steuerzeiten}

Fachbuch (\textcite{brand:2020:fachkundeKfz} S. 249)

Verdrehen der Einlassnockenwelle bzw. der Ein- und Auslassnockenwelle,
abhängig von der Motordrehzahl, Motorlast und Temperatur. Hierdurch
lässt sich die \emph{Länge der Ventilüberschneidung} anpassen.

\textbf{Warum machen wir eine Nockenwellenverstellung?} (Vorteile)

\begin{enumerate}
\item
  Optimale Zylinderfüllung in den unterschiedlichen Last- und
  Drehzahlbereichen zu ermöglichen
\item
  inneres AGR
\end{enumerate}

\textbf{Ziele der Nockenwellenverstellung}

\begin{itemize}
\item
  Wann das Ventil öffnet und schließt zu beeinflussen (variabel)
\item
  bei gleichbleibenden Nocken, Dauer und Öffnungswinkel (Hub) ändern
  sich nicht
\item
  Verdrehrichtung der Nockenwelle: Früh, Spät
\end{itemize}

\textbf{Verstellung der Einlassnockenwelle in Abhängigkeit vom
Betriebszustand}

\begin{table}[!ht]% hier: !ht 
\centering 
	\caption{}% \label{tab:}%% anpassen 
\begin{tabular}{@{}llll@{}}
\hline
\textbf{Betriebszustand} & \textbf{Leerlauf} & \textbf{Teillast} &
\textbf{Volllast} \\
\hline
Verstellrichtung NW & Spät & Früh & Spät \\
Ventilüberschneidung & klein & groß & klein \\
Abgas & CO sinkt & NOx sinkt & \\
EV schließt & weit nach UT & kurz nach UT & weit nach UT \\
\hline
\end{tabular} 
\end{table}

Merkmale (Vgl. Tabelle Verstellung der Einlassnockenwelle in
Abhängigkeit vom Betriebszustand)

\begin{itemize}
\item
  \textbf{Leerlauf} Kein Überströmen von Frischgasen und Abgasen,
  besserer Verbrennungsverlauf
\item
  \textbf{Teillast} Abgase strömen in den Einlasskanal und werden mit
  den Frischgasen angesaugt. Temperatur sinkt, NOx-Anteil sinkt
\item
  \textbf{Volllast} \emph{Nachladeeffekt} Frischgase strömen trotz
  aufwärts gehenden Kolben in den Zylinder nach
\end{itemize}

\textbf{Ausgangspunkt} $\to$ 90er-Jahre, erste Form des AGR (inneres
AGR), Drei-Wege-Katalysator, Ottomotor, Euro 2, Teillast (höchste
AGR-Rate, 80 km/h auf der Landstraße, keine Lastabfrage,
Spritspareffekt, NOx-Anteil senken)

\subsection{VarioCam - Verstellbarer Kettenspanner (Audi,
VW)}\label{variocam-verstellbarer-kettenspanner-audi-vw}

$\to$ Verändern der Ventilöffnungszeit der Einlassnockenwelle

\textbf{Wie?} Vgl. Tabelle Verstellung der Einlassnockenwelle in
Abhängigkeit vom Betriebszustand

\begin{itemize}
\item
  KW treibt Auslass-NW an und diese über einer Kette die Einlass-NW
\item
  \textbf{Kettenspanner} spannt \textbf{Kette nach oben} (federbelastet)
\item
  \textbf{NW} dreht sich \textbf{gegen UZS} (Uhrzeigersinn) in
  \textbf{Verstellposition} >>spät<< (Ausgangslage, Ventilüberschneidung
  klein)
\item
  SG bestromt Magnetventil, Motoröl fließt in Kettenspanner.
\item
  \textbf{Kettenspanner} spannt \textbf{Kette nach unten}
  (Hydraulikzylinder)
\item
  \textbf{NW} dreht sich \textbf{im UZS} in \textbf{Verstellposition}
  >>früh<<, (Ventilüberschneidung groß)
\end{itemize}

\subsection{Vanos - Variable Nockenwellensteuerung
(BMW)}\label{vanos-variable-nockenwellensteuerung-bmw}

\textbf{Wie?}

\begin{itemize}
\item
  \textbf{Nockenwellenrad und Nockenwelle} sind über ein \textbf{steiles
  Gewinde} miteinander verbunden.
\item
  \emph{Grundposition} NW steht in \textbf{Verstellposition} >>spät<<
\item
  SG bestromt ein Magnetventil (4/2-Wegeventil) $\to$ gibt den
  \textbf{Ölzufluss} zum Frühkanal frei
\item
  NW verdreht sich gegen Uhrzeigersinn in \textbf{Verstellposition}
  >>früh<<
\item
  Durch wechselseitigen Druckaufbau lässt sich die Position der
  Verstelleinheit halten.
\end{itemize}

\subsection{Flügelzellenversteller
(Mercedes)}\label{fluegelzellenversteller-mercedes}

$\to$ Verändern der Steuerzeiten

\textbf{Wie?}

\begin{itemize}
\item
  \textbf{Innenrotor} (fest mit NW) \textbf{und Außenrotor} (fest mit
  Kettenrad)
\item
  SG bestromt \textbf{Magnetventil} $\to$ die \textbf{Ölräume}
  zwischen den Rotorblättern können wechselseitig mit Öl befüllt werden
\item
  Die Kraftübertragung vom Nockenwellenrad auf die NW erfolgt immer über
  das Öl.
\item
  wird Ölraum rechts vom Innenrotorblatt mit Öl befüllt, kommt es zu
  einer \textbf{Verdrehung der NW gegen UZS} (Uhrzeigersinn) in Richtung
  >>spät<<
\item
  wird Ölraum links vom Innenrotorblatt mit Öl befüllt, kommt es zu
  einer \textbf{Verdrehung der NW im UZS} in Richtung >>früh<<
\item
  Durch wechselseitigen Druckaufbau lässt sich die Position der
  Verstelleinheit halten.
\end{itemize}

\section{Variabler Ventiltrieb}\label{variabler-ventiltrieb}

\subsection{Stufenweise variabler
Ventiltrieb}\label{stufenweise-variabler-ventiltrieb}

\textbf{Vorteile}

Bessere Zylinderfüllung durch zwei unterschiedliche Nockenprofile

\begin{itemize}
\item
  \emph{obere Drehzahlbereich} $\to$ steiler Nocken

  \begin{itemize}
  \item
    schnelles Öffnen, lange Öffnungsdauer, schnelles Schließen
  \end{itemize}
\item
  \emph{untere Drehzahlbereich} $\to$ spitzer Nocken

  \begin{itemize}
  \item
    Verhinderung von ungewollter Abgasrückführung durch zu lange
    Ventilüberschneidung
  \end{itemize}
\end{itemize}

\subsubsection{VTEC - Variable Valve Timing and Lift Electronic Control
(Honda)}\label{vtec-variable-valve-timing-and-lift-electronic-control-honda}

$\to$ Verändern von Ventilhub und Ventilöffnungszeit

\textbf{Wie?}

\begin{itemize}
\item
  Verstelleinheit liegt in den Schlepphebeln
\item
  \textbf{Umschaltung} zwischen dem Nockenprofilen erfolgt durch
  \textbf{Verblocken der Schlepphebel}
\item
  \textbf{Schlepphebel entriegelt}

  \begin{itemize}
  \item
    Die beiden äußeren Nocken öffnen mithilfe der äußeren Schlepphebel
    die Ventile.
  \item
    \textbf{Spitzer Nocken}

    \begin{itemize}
    \item
      kleiner Ventilhub
    \item
      kurze Ventilöffnungszeit
    \item
      \emph{niedrige Drehzahlen}
    \end{itemize}
  \end{itemize}
\item
  SG bestromt Elektromagnet, \textbf{Öldruck} verschiebt die
  \textbf{Sperrschieber} und verblockt die Schlepphebel untereinander.
\item
  \textbf{Schlepphebel verriegelt}

  \begin{itemize}
  \item
    wenn der steile Nocken auf den mittleren Schlepphebel aufläuft,
    nimmt dieser die beiden äußeren Schlepphebel mit und diese öffnen
    die Ventile.
  \item
    \textbf{Steiler Nocken}

    \begin{itemize}
    \item
      großer Ventilhub
    \item
      lange Ventilöffnungszeit
    \item
      \emph{hohe Drehzahlen}
    \end{itemize}
  \end{itemize}
\end{itemize}

\subsubsection{VarioCam Plus (Porsche)}\label{variocam-plus-porsche}

$\to$ Verändern von Ventilhub und Ventilöffnungswinkel

\textbf{Wie?}

\begin{itemize}
\item
  Verstelleinheit liegt im Tassenstößel
\item
  SG bestromt \textbf{Elektromagnet}, damit wird der
  \textbf{Tassenstößel mit Öldruck} gesteuert
\item
  Diese bestehen aus \textbf{zwei Stößeln}, die mithilfe eines
  \textbf{Bolzens} gegeneinander verriegelt werden können.
\item
  innere Stößel $\to$ kleinen Nocken
\item
  äußere Stößel $\to$ großen Nocken
\item
  \textbf{Stößel verriegelt} $\to$ große Ventilhub

  \begin{itemize}
  \item
    Innere und äußere Stößel wird durch einen Bolzen verriegelt
  \end{itemize}
\item
  \textbf{Stößel entriegelt} $\to$ kleiner Ventilhub

  \begin{itemize}
  \item
    sinkt der Öldruck, wird durch die Federkraft der Bolzen
    zurückgeschoben
  \end{itemize}
\end{itemize}

\subsubsection{Valvelift (Audi, +
Zylinderabschaltung)}\label{valvelift-audi-zylinderabschaltung}

\textbf{Wie?}

\begin{itemize}
\item
  Änderung des Nockenprofils durch Verschieben der Verstelleinheit
  (Nockenstück) auf der NW
\item
  SG bestromt \textbf{Elektromagnet} $\to$ \textbf{Metallstift} fährt
  aus \textbf{in eine Spiralnut} und verschiebt das \textbf{Nockenstück}
\item
  damit schalte ich zwischen \textbf{zwei Nockenprofilen} um
\item
  Arretierung des Nockenstücks erfolgt durch eine federbelastete Kugel.
\item
  \textbf{Zylinderabschaltung} (Teillast)

  \begin{itemize}
  \item
    Nockenprofil $\to$ Nockengrundkreis
  \item
    Die Ventile bleiben bei abgeschaltetem Zylinder geschlossen.
  \end{itemize}
\end{itemize}

\subsection{Stufenlos variabler
Ventiltrieb}\label{stufenlos-variabler-ventiltrieb}

\textbf{Vorteile}

$\to$ Verändern von Ventilhub in allen Drehzahlbereichen

\textbf{Ziel im unteren Drehzahlbereich}: Ein zündbares Gemisch zu
realisieren.

\textbf{Wie?}

\begin{itemize}
\item
  Durch geringe Ventilöffnung und damit Erhöhung der
  Strömungsgeschwindigkeit der Frischgase

  \begin{itemize}
  \item
    >>Venturi-Prinzip<< eine Verengung in einem Strömungskanal

    \begin{itemize}
    \item
      $\to$ höhere Strömungsgeschwindigkeit
    \item
      $\to$ bessere Verwirbelung
    \item
      $\to$ bessere Verteilung des Kraftstoff-Luftgemisches
    \end{itemize}
  \end{itemize}
\item
  Drosselklappe könnte wegfallen, wird aber weiterhin verbaut
\item
  \textbf{Wozu ist die Drosselklappe dann noch notwendig?}

  \begin{itemize}
  \item
    Schaltung des AGR (Abgasrückführung)

    \begin{itemize}
    \item
      Aufbau eines Druckgefälles/Druckdifferenz, durch Schließen der
      Drosselklappe wird ein Unterdruck erzeugt, was dazu führt, dass
      die Abgase in den Ansaugtrakt einströmen können
    \end{itemize}
  \end{itemize}
\item
  Notlauf
\end{itemize}

\subsubsection{Valvetronic}\label{valvetronic}

$\to$ Verändern von Ventilöffnungswinkel (Hub) und Ventilöffnungsdauer
(Nockenwellenverstellung)

\textbf{Wie?}

\begin{itemize}
\item
  SG verdreht mithilfe eines \textbf{Stellmotors} eine
  \textbf{Exzenterwelle} (Halbmondförmig)
\item
  Druck des Nockens wird zunächst auf einen \textbf{Zwischenhebel}
  übertragen
\item
  Der \textbf{Leerweg}, den der Zwischenhebel von der Betätigung durch
  den Nocken bis zur Übertragung auf das Ventil durchläuft, ist mittels
  einer Exzenterwelle einstellbar.
\item
  Je größer der Leerweg, desto kleiner der Ventilhub.
\item
  \textbf{Ventilhub} 0,3 mm und 9,85 mm
\end{itemize}

\subsubsection{Elektrohydraulischer Ventiltrieb
(MultiAir)}\label{elektrohydraulischer-ventiltrieb-multiair}

\textbf{Vorteil} Vollvariable Steuerzeiten

$\to$ stufenlose Veränderung von Ventilhub, Ventilöffnungsdauer und
die Anzahl der Ventilhübe der EV

\textbf{Wie?}

\begin{itemize}
\item
  auf der \textbf{Auslassnockenwelle} gibt es einen
  \textbf{Extranocken}, über Schlepphebel wird ein
  \textbf{Pumpenelement} betätigt

  \begin{itemize}
  \item
    $\to$ der erzeugt einen \textbf{Öldruck}, um die
    \textbf{Einlassseite} zu steuern,
  \end{itemize}
\item
  \textbf{Magnetventil geschlossen} Druck wird auf den Kolben
  übertragen, Ventil öffnen
\item
  \textbf{Magnetventil offen} Ventil schließen. Der Öldruck fließt in
  den Druckspeicher ab.
\item
  \emph{Vorteil \textbf{Druckspeicher}:} von der Nockenwelle
  unabhängiger Zeitpunkt, mit Öffnung eines Magnetventils (SG) ein
  Öldruck aus dem Druckspeicher nutzen, der das \textbf{Ventil
  öffnet/schließt}
\item
  \textbf{elektrohydraulisch-pneumatisch} (Ventile unabhängig von NW
  betätigen, noch nicht in der Großserie)
\item
  chinesische Hersteller Qoros und der schwedische
  Luxussportwagenhersteller Königsegg
\end{itemize}

\subsubsection{Elektromagnetischer Ventiltrieb (noch nicht zur
Serienreife
geschafft)}\label{elektromagnetischer-ventiltrieb-noch-nicht-zur-serienreife-geschafft}

\textbf{Vorteile}

\begin{itemize}
\item
  Vollvariable Steuerzeiten
\item
  Anzahl der geöffneten Ventile pro Zylinder frei wählbar
\item
  Zylinderabschaltung (ohne Gaswechselverluste möglich)
\item
  Wegfall von Nockenwellen (Gewichtseinsparung)
\end{itemize}

\textbf{Wie?}

\begin{itemize}
\item
  Unterstützung des Elektromagneten beim schnellen Öffnen und Schließen
  des Ventils.
\item
  Abbremsen des Ventils kurz vor den Endstellungen geöffnet und
  geschlossen
\item
  Ventile beim abgeschalteten oder defekten Systems in halbgeöffnete
  Stellung bringen, um Motorschäden durch Aufsetzen der Ventile zu
  verhindern.
\end{itemize}
