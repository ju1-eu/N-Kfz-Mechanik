% ju 13-5-22 Moderator-Karten-Vorlage.tex
\documentclass[a5paper,17pt,fleqn,parskip=half]{scrartcl}
\include{content/praeambel-artikel}  

%\usepackage[left=2cm,right=1cm,top=1cm,bottom=1cm,includeheadfoot]{geometry}
%\usepackage[left=4cm,right=2cm,top=1cm, bottom=1cm,includeheadfoot]{geometry}
\usepackage[landscape=true,left=6cm,right=1cm,top=1cm,bottom=0.5cm,includefoot]{geometry}%quer

% eigene Farbe definieren
% Adobe Prozessfarben: CMYK: 100,50,0,35 -> 1,0.5,0,0.35
\definecolor{orange}{cmyk}{0,0.55,0.61,0}   % 0,55,61,0
\definecolor{blau5}{cmyk}{1,0.77,0.1,0.01}  % 100,77,10,
\definecolor{rot5}{cmyk}{0.22,1,1,0.19}     % 22,100,100,19
\definecolor{grau2}{cmyk}{0,0,0,0.1}        % 0,0,0,40
\definecolor{blau}{cmyk}{0.93,0.66,0,0.21}% 

% Literatur
\bibliography{content/literatur}
\bibliography{content/literatur-kfz}
\bibliography{content/literatur-sport}

%%%%%%%%%%%%%%%%%%%%%%%%%%%%%%%%%%%%%%%%%%%%%%%%%%%%%%%
\newcommand{\name}{Jan Unger}% anpassen!!!!!
\newcommand{\thema}{Lerntools}% anpassen!!!!!
\newcommand{\quelle}{\name}
\newcommand{\website}{https://bw-ju.de/}
\newcommand{\github}{https://github.com/ju1-eu}
%%%%%%%%%%%%%%%%%%%%%%%%%%%%%%%%%%%%%%%%%%%%%%%%%%%%%%%

%\ihead{\textbf{Quelle:} \quelle}%{Kopfzeile innen}
%\ohead{\textbf{Datum:} \today}  %{Kopfzeile außen}
%\ifoot{\textbf{Thema:} \thema}  %{Fußzeile  innen}
\ofoot{Seite {\thepage} von {\pageref{LastPage}}}%{Fußzeile  außen}

\title{\thema}
\author{\name}
\date{\today}

\begin{document}
	%\thispagestyle{empty}
	%\maketitle
	%\newpage
	%\setcounter{page}{1}

	%%%%%%%%%%%%%%%%%%%%%%%%%%%%%%%%%%%%%%%
	%
	% \textcolor{rot5}{Folie 1}
	% \textbf{Fett}
	% \vspace{1em}%Abstand
	%
	%%%%%%%%%%%%%%%%%%%%%%%%%%%%%%%%%%%%%%%

	\subsection*{\textcolor{rot5}{Folie 1}}

	( Sehr geehrte Damen und Herren, )

	Hallo \textcolor{rot5}{Frau Kühr}, Hallo Leute!

	\vspace{1em}%Abstand
	Ich heiße Jan Unger und
	\textbf{mein Thema} ist Zeit.

	\vspace{1em}%Abstand
	Während ich hier \textcolor{rot5}{stehe} (oder sitze!) 

	und mit dem Reden beginne, vergeht sie bereits.

	\newpage
	\subsection*{\textcolor{rot5}{Folie 2}}

	\newpage
	\subsection*{\textcolor{rot5}{Folie 3}}

	\textbf{Zusammenfassend} möchte ich sagen:

	Die Zeit ist das, was wir aus ihr machen.

	\vspace{1em}%Abstand
	Meine \textbf{Schlussbemerkung} an Euch ist,

	>>Genießt den Augenblick!<<.

	\vspace{1em}%Abstand
	Vielen Dank.

	%\input{content/tex/.tex}

    % Bibliographie
    \printbibliography
\end{document}
